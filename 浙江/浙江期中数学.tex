\documentclass{exam-zh}

\begin{document}



\section{选择题:本题共 8 小题,每小题 5 分,共 40 分.在每小题给出的四个选项中,只有一项是符合题目要求的.}

\begin{question}
    已知集合 \( A = \{ x \mid 0 < x < 3 \} \),\( B = \{-1,0,1,2,3\} \),则 \( A \cap B = \)
    \begin{choices}
        \item \( \{1,2\} \)
        \item \( \{0,1,2\} \)
        \item \( \{-1,0,1,2\} \)
        \item \( \{-2,-1,0,1,2\} \)
    \end{choices}
\end{question}

\begin{question}
已知函数 \( f(x) = (m^2 + m - 1)x^{m^2 - 2} \) 是幂函数,且在 \( (0,+\infty) \) 上递增,则实数 \( m = \)
    \begin{choices}
        \item \( 2 \)
        \item \( -2 \)
        \item \( 1 \)
        \item \( 1 \) 或 \( -2 \)
    \end{choices}
\end{question}

\begin{question}
命题“\( \exists x \in \mathbf{R} \),\( x^2 - x + 3 \leq 0 \)”的否定是
    \begin{choices}
        \item \( \exists x \in \mathbf{R} \),\( x^2 - x + 3 \geq 0 \)
        \item \( \exists x \in \mathbf{R} \),\( x^2 - x + 3 > 0 \)
        \item \( \forall x \in \mathbf{R} \),\( x^2 - x + 3 \leq 0 \)
        \item \( \forall x \in \mathbf{R} \),\( x^2 - x + 3 > 0 \)
    \end{choices}
\end{question}

\begin{question}    
已知函数 \( f(x) = \begin{cases} x + 1, & x < 3 \\ f(x - 3), & x \geq 3 \end{cases} \),则 \( f(7) = \)
    \begin{choices}
        \item \( 16 \)
        \item \( 8 \)
        \item \( 2 \)
        \item \( -2 \)
    \end{choices}
\end{question}
    
\begin{question}
函数 \( f(x) = 2^{|x + 1|} \) 的图象大致为
%    \begin{choices}
%        \item 选项 A(图象:关于 \( x = -1 \) 对称,\( x = -1 \) 时 \( y = 1 \),\( x > -1 \) 时递增)
%        \item 选项 B(图象:关于 \( x = -1 \) 对称,\( x = -1 \) 时 \( y = 1 \),\( x > -1 \) 时递减)
%        \item 选项 C(图象:关于 \( x = 1 \) 对称,\( x = 1 \) 时 \( y = 1 \),\( x > 1 \) 时递减)
%        \item 选项 D(图象:关于 \( x = -1 \) 对称,\( x = -1 \) 时 \( y = 1 \),\( x > -1 \) 时递增且增长更快)
%    \end{choices}
	\begin{center}
	\includegraphics[width=\linewidth]{1-5.png}
	\end{center}
\end{question}

\begin{question}
使得不等式 \( 2^m > 2^n \) 成立的一个充分不必要条件是
    \begin{choices}
        \item \( \dfrac{1}{n} > \dfrac{1}{m} \)
        \item \( \sqrt{m} > \sqrt{n} \)
        \item \( m^2 > n^2 \)
        \item \( m^3 > n^3 \)
    \end{choices}
\end{question}

\begin{question}
已知函数 \( f(x) = \begin{cases} (4a - 1) \cdot 2^x, & x < 1 \\ x^2 - ax + 6, & x \geq 1 \end{cases} \) 是 \( \mathbf{R} \) 上的单调递增函数,则实数 \( a \) 的取值范围是
    \begin{choices}
        \item \( \left( \dfrac{1}{4}, 1 \right] \)
        \item \( \left( \dfrac{1}{4}, 2 \right] \)
        \item \( [2, +\infty) \)
        \item \( [1, 2] \)
    \end{choices}
\end{question}

\begin{question}
已知函数 \( f(x) = x^2 + 2x + m(e^{x + 1} + e^{-x - 1}) \) 的图象与 \( x \) 轴有且只有一个交点,则 \( m \) 的值为
    \begin{choices}
        \item \( -\dfrac{1}{2} \)
        \item \( \dfrac{1}{3} \)
        \item \( \dfrac{1}{2} \)
        \item \( 1 \)
    \end{choices}
\end{question}

\section{选择题:本题共 3 小题,每小题 6 分,共 18 分.在每小题给出的选项中,有多项符合题目要求.全部选对的得 6 分,部分选对的得部分分,有选错的得 0 分.}

\begin{question}
下列各选项给出的数学命题中,正确的是
    \begin{choices}
        \item 集合 \( A = \{ x \mid y = \sqrt{x^2 + 1} \} \),\( B = \{ y \mid y = \sqrt{x^2 + 1} \} \) 表示相等集合
        \item 若 \( y = f(x) \) 是一次函数,满足 \( f(f(x)) = x + 2 \),则 \( f(x) = x + 1 \)
        \item 函数 \( y = \dfrac{x - 2}{x + 1} (x \geq 1) \) 的值域为 \( \left[ -\dfrac{1}{2}, 1 \right) \)
        \item 若关于 \( x \) 的不等式 \( ax^2 + bx + c > 0 \) 的解集为 \( (-2, 3) \),则不等式 \( cx^2 - bx + a < 0 \) 的解集为 \( \left( -\dfrac{1}{3}, \dfrac{1}{2} \right) \)
    \end{choices}
\end{question}
    
\begin{question}
已知 \( a > 0, b > 0 \),且 \( \dfrac{1}{a} + \dfrac{4}{b} = 4 \),则
    \begin{choices}
        \item \( b > 1 \)
        \item \( a + b \geq \dfrac{9}{4} \)
        \item \( ab \leq 1 \)
        \item \( \dfrac{1}{a^2} + \dfrac{16}{b^2} \geq 8 \)
    \end{choices}
\end{question}
    
\begin{question}
已知定义域为 \( \mathbf{R} \) 的函数 \( f(x) \),对任意实数 \( x, y \) 都有 \( f(x + y) + f(x - y) = f(x)f(y) \),且 \( f(2) = -2 \),则以下结论一定正确的有
    \begin{choices}
        \item \( f(0) = 1 \)
        \item \( f(x) \) 是奇函数
        \item \( f(x) \) 关于点 \( (1, 0) \) 中心对称
        \item \( f(1) + f(2) + \cdots + f(2025) = 0 \)
    \end{choices}
\end{question}    
\begin{center}
\vspace{0.5cm}
\Large
\textbf{第Ⅱ卷}
\vspace{0.5cm}
\end{center}

\section{填空题:本题共 3 小题,每小题 5 分,共 15 分.}

\begin{question}
    计算 \( 8^{\dfrac{2}{3}} + 2\lg 2 + \lg 25 = \) \(\underline{\quad\quad\quad\quad\quad}\)
\end{question}    

    
\begin{question}
已知函数 \( f(x) = \begin{cases} x + 2, & x < 0 \\ x^2 - 2x + 2, & x \geq 0 \end{cases} \),若当 \( x \in [a, b] \) 时,\( 1 \leq f(x) \leq 10 \),则 \( b - a \) 的最大值是 \(\underline{\quad\quad\quad\quad\quad}\)
\end{question}    

\begin{question}
已知函数 \( f(x) = ax^2 - bx - a + b \),\( x \in [0, m] \),对任意 \( 2b \geq a > 0 \),不等式 \( f(x) \leq (2b - a)(x + 1) \) 恒成立,则 \( m \) 的最大值为 \(\underline{\quad\quad\quad\quad\quad}\)
\end{question}    


\section{解答题:本题共 5 小题,共 77 分.解答应写出文字说明、证明过程或演算步骤.}

\begin{problem}
    (13 分) 设 \( U = \mathbf{R} \),\( A = \{ x \mid 2 < x < 4 \} \),\( B = \{ x \mid x^2 - 4x + a \leq 0, a \in \mathbf{R} \} \).
    \begin{enumerate}
        \item 当 \( a = 3 \) 时,求 \( A \cap B \),\( (\complement_U A) \cup B \);
        \item 若 \( A \cap B = A \),求实数 \( a \) 的取值范围.
    \end{enumerate}
\end{problem}
\begin{problem}
    (15 分) 已知函数 \( f(x) = (a^2 - 2a - 2)a^x (a > 0 \text{ 且 } a \neq 1) \) 过点 \( (0, 1) \).
    \begin{enumerate}
        \item 求 \( a \) 的值,并写出 \( f(x) \) 的解析式;
        \item 判断 \( F(x) = f(x) - \dfrac{1}{f(x)} \) 的奇偶性,并用定义证明;
        \item 若 \( f(x) = h(x) + t(x) \),且 \( h(x) \) 为奇函数,\( t(x) \) 为偶函数,写出 \( h(x) \) 的解析式(无需证明).
    \end{enumerate}
\end{problem}

\begin{problem}
(15 分) 为加快落实新旧动能转换,某单位在国家科研部门的支持下,进行技术攻关,新上了一个项目,该项目可以把二氧化碳处理转化为一种可利用的化工产品.经测算,该项目月处理成本 \( y \)(元)与月处理量 \( x \)(吨)之间的函数关系可近似地表示为 \( y = \begin{cases} \dfrac{1}{3}x^3 - 80x^2 + 5040x, & x \in [120, 144) \\ \dfrac{1}{2}x^2 - 200x + 80000, & x \in [144, 500] \end{cases} \),且每处理一吨二氧化碳得到可利用的化工产品价值为 200 元,若该项目不盈利,国家将给予补偿.
    \begin{enumerate}
        \item 当 \( x \in [200, 300] \) 时,判断该项目能否盈利.如果盈利,求出最大利润;如果该项目不盈利,要使该单位不亏损,则国家需要补偿资金的范围是多少元?
        \item 该项目每月处理量为多少吨时,才能使每吨的平均处理成本最低?
    \end{enumerate}
\end{problem}    
\begin{problem}
    (17 分) 已知函数 \( f(x) = \dfrac{ax + b}{x^2 + 1} \),若 \( f(1) = \dfrac{1}{2} \),且当 \( x \neq 0 \) 时 \( f(x) = f\left( \dfrac{1}{x} \right) \).
    \begin{enumerate}
        \item 求 \( a, b \) 的值,并写出 \( f(x) \) 的解析式;
        \item 判断函数 \( f(x) \) 在 \( [1, +\infty) \) 上的单调性,并用定义证明;
        \item 若对任意的 \( x \in \left[ \dfrac{1}{3}, \dfrac{3}{4} \right] \) 都有 \( f(x) + f(3x - a) \leq 0 \) 恒成立,求实数 \( a \) 的取值范围.
    \end{enumerate}
\end{problem}

\begin{problem}
    (17 分) 俄国数学家切比雪夫(1821—1894)是研究直线逼近函数的理论先驱. 设 \( f(x) \) 是定义在 \( [m, n] \) 上的连续函数,称 \( E = \max\limits_{m \leq x \leq n} |f(x) - (ax + b)| \) 为 \( f(x) \) 与直线 \( g(x) = ax + b \) 的偏差. 若存在 \( x_0 \in [m, n] \) 使得 \( |f(x_0) - g(x_0)| = E \),则称 \( x_0 \) 为直线 \( g(x) \) 的偏差点. 记 \( A = \{ g(x) = ax + b \mid a, b \in \mathbf{R} \} \),若存在 \( g_0(x) \in A \) 使得 \( \max\limits_{m \leq x \leq n} |f(x) - g_0(x)| = \min\limits_{a, b \in \mathbf{R}} \max\limits_{m \leq x \leq n} |f(x) - g(x)| \) 则称 \( g_0(x) \) 为 \( f(x) \) 在切比雪夫意义下的最佳逼近直线.
    \begin{enumerate}
        \item 函数 \( f(x) = x^2, x \in [-1, 2] \),\( g(x) = x + 1 \),求 \( f(x), g(x) \) 的偏差以及偏差点;
        \item 函数 \( f(x) = 3x + \dfrac{4}{x}, x \in [1, 4], g(x) = 2x + b \),求 \( f(x), g(x) \) 的偏差的最小值,并求出取得最小值时 \( b \) 的值;
        \item 证明:直线 \( g(x) = \dfrac{1}{2}x + \dfrac{1}{4} \) 是函数 \( f(x) = \sqrt{x} \) 在 \( x \in [0, 4] \) 上的最佳逼近直线.
    \end{enumerate}
\end{problem}

\end{document}