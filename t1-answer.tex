\documentclass[CJKmath,a4paper,10pt]{ctexart}
\usepackage{geometry}
\geometry{inner=1.5cm,outer=2.5cm,top=2.45cm,bottom=3cm}
\usepackage[dvipsnames,svgnames,x11names,table]{xcolor}


\RequirePackage{amssymb} % Must be loaded before unicode-math
\RequirePackage{unicode-math} % Math fonts in xetexorluatex
%\setmathfont{texgyrepagella-math.otf}
%\setmainfont{STIXTwoText}[
%	Path=./fonts/STIXTwoText/,
%  Extension = .otf,
%  UprightFont = STIXTwoMath-Regular,
%  BoldFont = STIXTwoText-SemiBold,
%  ItalicFont = STIXTwoText-Italic,
%  BoldItalicFont = STIXTwoText-BoldItalic,
%]
%\setmathfont{STIXTwoMath-Regular.otf}

\usepackage{tikz,calc}
\usetikzlibrary{calc,arrows,shadows.blur,tikzmark}
% By default all math in TikZ nodes are set in inline mode. Change this to
% displaystyle so that we don't get small fractions.
\everymath{\displaystyle}

\usepackage[most]{tcolorbox}
\tcbuselibrary{breakable, skins,theorems}


\usepackage{xkeyval}
\makeatletter 
\usepackage{amsmath,amssymb,amsthm}
\usepackage{xkeyval} 
\usepackage{zhlipsum}
\tcbset{
    common/.style={
      fontupper=\rmfamily,
      lower separated=true,
      coltitle=black,
      colback=gray!5,
      boxrule=0.5pt,
      fonttitle=\bfseries,
      enhanced,
      breakable,
      top=8pt,
      before skip=8pt,
      attach boxed title to top left={yshift=-0.05in,xshift=0.15in},
      boxed title style={
      	boxrule=0pt,
        colframe=black,
        arc=0pt,
        outer arc=0pt,
        drop fuzzy shadow
     	},
      separator sign={.},
      drop fuzzy shadow
     },
     litistyle/.style={  %例题风格
     	common,
      colframe=gray,% 
      colback=white,
      colbacktitle=Gold,
      sharp corners,rounded corners=southeast,
      overlay unbroken and last={
      \node[anchor=south east, outer sep=0pt] at (\linewidth-width,0){\textcolor{black!60}{$\blacksquare$}};}
         },
% -----------------------------------------------------------------------------
% Explanation: BreakableBox@title key
% The following defines a pgfkeys/tcolorbox key named
%   "BreakableBox@title" which accepts 2 arguments (see 
%   "/.code n args={2}"). When the key is used like
%   "BreakableBox@title={<env>}{<optTitle>}" the code below is executed
%   with #1=<env> and #2=<optTitle>.
%
% Purpose:
% - If the second argument (#2) is empty, set the tcolorbox title to
%     \csname <env>name\endcsname~\thetcbcounter
%   which expands to the environment name (e.g. "liti" -> \litiname)
%   followed by the current counter value.
% - If the second argument is provided, append " (#2)" to the title.
%
% Notes:
% - This relies on \makeatletter being active so that keys with '@'
%   are allowed in control sequence names.
% - \ifblank is used to test whether #2 is empty; ensure the package
%   that provides \ifblank (for example, etoolbox or xparse) is loaded
%   earlier if necessary.
% - Example usage inside the code: \tcbset{title={...}} sets the
%   tcolorbox title key accordingly.
%
         BreakableBox@title/.code n args={2}
              {
                \ifblank{#2}
                  {\tcbset{title={\csname #1name\endcsname~\thetcbcounter}}}
                  {\tcbset{title={\csname #1name\endcsname~\thetcbcounter\ (#2)}}}
              },
}      
  % define an internal control sequence \newbreakablebox for fancy mode's newtheorem
  % #1 is the environment name, #2 is the prefix of label, #3 is the style
  % style: thmstyle, defstyle, prostyle
  % e.g. \newbreakablebox{theorem}{thm}{thmstyle}
  % will define two environments: numbered ``theorem'' and no-numbered ``theorem*''
  % WARNING FOR MULTILINGUAL: this cs will automatically find \theoremname's definition,
  % WARNING FOR MULTILINGUAL: it should be defined in language settings.
  \newcommand{\newbreakablebox}[3]{
    \ifcsundef{#1name}{%
      % define a default name if not defined before
      \tcbset{BreakableBox@title/.code n args={2}{\tcbset{title={use newcommand define #1name}}} }
    }{\relax}
    \DeclareTColorBox[auto counter,number within=section]{#1}{ g o t\label g }{
        common, % use common style
        #3, % use the style passed in
        IfValueTF={##1}
          {BreakableBox@title={#1}{##1}}
          {
            IfValueTF={##2}
            {BreakableBox@title={#1}{##2}}
            {BreakableBox@title={#1}{}}
          },
        IfValueT={##4}
          {
            IfBooleanTF={##3}
              {label={##4}}
              {VIVID@label={#2}{##4}}
          }
      }
    \DeclareTColorBox{#1*}{ g o }{
        common,#3,
        IfValueTF={##1}
          {BreakableBox@title={#1}{##1}}
          {
            IfValueTF={##2}
            {BreakableBox@title={#1}{##2}}
            {BreakableBox@title={#1}{}}
          },
      }
  }
  % define several environment 
  % we define headers like \definitionname before
  \newcommand{\litiname}{例题}
  \newbreakablebox{liti}{def}{litistyle}
  
  
%补充内容
%%设置新字体
%%定义带圈数字命令
\newfontfamily{\nmfont}{circlenumber}
[%
Extension=.otf,
Path=./fonts/]

\newcommand{\quan}[1]{{\nmfont \symbol{#1}}}
\newcommand{\kk}[1]{\quan{\numexpr32+#1}}%\kk{<参数范围1-95>}96、97、98、99分别用\quan{196} \quan{197} \quan{199} \quan{201}

%脚注使用带圈数字
\newcommand*\kkctr[1]{%
  \protect\kk{\number\numexpr\value{#1}\relax}}
\renewcommand*\thefootnote{\textcolor{black}{\kkctr{footnote}}}

%%无悬挂脚注格式
\renewcommand\@makefntext[1]{%
  \setlength\parindent{2\ccwd}\selectfont
  \@thefnmark\ #1}

%修改\part,使其不分页
\def\@endpart{%
	\thispagestyle{empty}
  \vskip40\p@%
   \@afterheading}



\RequirePackage{enumitem}
%\newenvironment{myenum}{\begin{enumerate}[label=\protect\kk{\arabic*}]\small}{\end{enumerate}}%
\setlist{noitemsep}
\setlist[enumerate, 1]{label=\protect\kk{\arabic*},itemsep=0.5ex}  
\makeatother



%%%marker环境
\newtcolorbox{marker}[1][]{enhanced,before skip=2mm,
	after skip=3mm,fontupper=\rmfamily,
	boxrule=0.4pt,left=5mm,right=2mm,top=1mm,bottom=1mm,
	colback=yellow!50,colframe=yellow!20!black,
	sharp corners,rounded corners=southeast,
	arc is angular,arc=3mm,underlay={%
		\path[fill=tcbcolback!80!black] ([yshift=3mm]interior.south east)--++(-0.4,-0.1)--++(0.1,-0.2);
		\path[draw=tcbcolframe,shorten <=-0.05mm,shorten >=-0.05mm] ([yshift=3mm]interior.south east)--++(-0.4,-0.1)--++(0.1,-0.2);
		\path[fill=yellow!50!black,draw=none] (interior.south west) rectangle node[white]{\Huge\bfseries !} ([xshift=4mm]interior.north west);
	},
	drop fuzzy shadow,#1
}
\begin{document}
\section{单项选择题(本题共8小题,每小题5分,共40分)}

\begin{enumerate}
    \item 答案:D
    命题透析:本题考查集合的交运算。
    解析:根据交集的定义可得 \( A \cap B = \{ x \mid -1 < x < 1 \text{ 或 } 2 < x < 3 \} \)。

    \item 答案:A
    命题透析:本题考查充分条件与必要条件的判断。
    解析:当 \( x = 2 \) 时,\( x^2 - 3x + 2 = 0 \),满足充分性;当 \( x^2 - 3x + 2 = 0 \) 时,\( x = 2 \) 或 \( 1 \),不一定有 \( x = 2 \),故必要性不成立。

    \item 答案:A
    命题透析:本题考查一元二次不等式与一元二次方程。
    解析:因为 \( x^2 + bx + c < 0 \) 的解集为 \( (-3, 4) \),所以 \( -3, 4 \) 是方程 \( x^2 + bx + c = 0 \) 的两根,由根与系数的关系可得
    \[
    \begin{cases}
    -3 + 4 = -b \\
    -3 \times 4 = c
    \end{cases}
    \]
    解得 \( \begin{cases} b = -1 \\ c = -12 \end{cases} \),所以 \( b + c = -13 \)。

    \item 答案:B
    命题透析:本题考查指数与对数的运算。
    解析:因为 \( p = 2^{1.1} > 2 \),\( q = \log_4 8 = \log_4 4^{\dfrac{3}{2}} = \dfrac{3}{2} \),\( r = 0.3^2 = 0.09 \),所以 \( p > q > r \)。

    \item 答案:D
    命题透析:本题考查函数的奇偶性与单调性。
    解析:\( f(x) = -x \) 为减函数;\( f(x) = -\dfrac{1}{x} \) 为奇函数,在 \( (-\infty, 0) \),\( (0, +\infty) \) 上均单调递增,但在定义域上不单调;\( f(x) = x^{\dfrac{1}{2}} \) 的定义域为 \( [0, +\infty) \),\( f(x) \) 是非奇非偶函数;\( f(x) = \begin{cases} -x^2, x < -1 \\ x, -1 \leq x \leq 1 \\ x^2, x > 1 \end{cases} \) 的定义域为 \( \mathbb{R} \),易知此函数符合题意。

    \item 答案:D
    命题透析:本题考查基本不等式的应用。
    解析:方法一:因为 \( a + b \geq 2\sqrt{ab} \),所以 \( ab \leq \left( \dfrac{a + b}{2} \right)^2 = \left( \dfrac{4}{2} \right)^2 = 4 \)(当且仅当 \( a = b = 2 \) 时等号成立),所以 \( (1 + a)(1 + b) = 1 + (a + b) + ab \leq 1 + 4 + 4 = 9 \)。
    方法二:\( (1 + a)(1 + b) \leq \left[ \dfrac{(1 + a) + (1 + b)}{2} \right]^2 = 3^2 = 9 \)(当且仅当 \( a = b = 2 \) 时等号成立)。

    \item 答案:B
    命题透析:本题考查“对勾函数”的性质。
    解析:当 \( b < 0 \) 时,\( f(x) = x + \dfrac{b}{x} \) 在 \( (0, +\infty) \) 上单调递增,所以在 \( (1, +\infty) \) 上单调递增,符合题意;
    当 \( b = 0 \) 时,\( f(x) = x \) 在 \( (-\infty, +\infty) \) 上单调递增,所以在 \( (1, +\infty) \) 上单调递增,符合题意;
    当 \( b > 0 \) 时,\( f(x) = x + \dfrac{b}{x} \) 为对勾函数,在 \( (\sqrt{b}, +\infty) \) 上单调递增,因为 \( f(x) \) 在 \( (1, +\infty) \) 上单调递增,所以 \( \sqrt{b} \leq 1 \),所以 \( 0 < b \leq 1 \)。
    综上所述:\( b \leq 1 \)。

    \item 答案:C
    命题透析:本题考查利用函数的单调性解不等式。
    解析:不妨设 \( 0 < x_1 < x_2 \),因为 \( \dfrac{x_2 f(x_1) - x_1 f(x_2)}{x_1 - x_2} < 0 \),所以 \( x_2 f(x_1) - x_1 f(x_2) > 0 \),所以有 \( \dfrac{f(x_1)}{x_1} > \dfrac{f(x_2)}{x_2} \),所以函数 \( g(x) = \dfrac{f(x)}{x} \) 是 \( (0, +\infty) \) 上的减函数,由 \( f(x) \) 的定义域为 \( (0, +\infty) \),知在 \( f(x + 5) > \dfrac{f(x^2 - 25)}{x - 5} \) 中,应满足
    \[
    \begin{cases}
    x + 5 > 0, \\
    x^2 - 25 > 0, \\
    x - 5 \neq 0,
    \end{cases}
    \]
    解得 \( x > 5 \)。当 \( x > 5 \) 时,\( f(x + 5) > \dfrac{f(x^2 - 25)}{x - 5} \implies \dfrac{f(x + 5)}{x + 5} > \dfrac{f(x^2 - 25)}{x^2 - 25} \),则 \( g(x + 5) > g(x^2 - 25) \),所以 \( 0 < x + 5 < x^2 - 25 \),解得 \( x > 6 \),故不等式 \( f(x + 5) > \dfrac{f(x^2 - 25)}{x - 5} \) 的解集为 \( (6, +\infty) \)。
\end{enumerate}

\section{多项选择题(本题共3小题,每小题6分,共18分。每小题全部选对的得6分,部分选对的得部分分,有选错的得0分)\\ }

\begin{enumerate}
    \item 答案:BD
    命题透析:本题考查幂函数、指数函数的性质。
    解析:对于A,取 \( a = 1, b = -2 \),则 \( a^2 < b^2 \),故A错误;
    对于B,函数 \( y = x^3 \) 在 \( \mathbb{R} \) 上单调递增,所以 \( a > b \iff a^3 > b^3 \),故B正确;
    对于C,取 \( a = 1, b = -2 \),则 \( \dfrac{1}{a} > \dfrac{1}{b} \),故C错误;
    对于D,函数 \( y = 2^x \) 在 \( \mathbb{R} \) 上单调递增,所以 \( a > b \iff 2^a > 2^b \),故D正确。

    \item 答案:ABC
    命题透析:本题考查指数函数与一次函数的图象。
    解析:对于A,指数函数图象对应 \( a > 1, b > 0 \),一次函数图象也符合,故A正确;
    对于B,指数函数图象对应 \( 0 < a < 1, b > 0 \),一次函数图象也符合,故B正确;
    对于C,由指数函数图象可知 \( a > 1, b < 0 \),一次函数图象也符合,故C正确;
    对于D,指数函数图象对应 \( 0 < a < 1, b < 0 \),一次函数图象对应 \( b > 0 \),矛盾,故D错误。

    \item 答案:ACD
    命题透析:本题考查指数函数的性质。
    解析:对于A,\( f(0) = a^0 - a^0 = 1 - 1 = 0 \),即 \( f(x) \) 的图象过定点 \( (0, 0) \),故A正确;
    对于B,当 \( a > 1 \) 时,\( y = a^x \) 为增函数,\( y = a^{-x} \) 为减函数,所以 \( f(x) = a^x - a^{-x} \) 在 \( \mathbb{R} \) 上是增函数,当 \( 0 < a < 1 \) 时,\( y = a^x \) 为减函数,\( y = a^{-x} \) 为增函数,所以 \( f(x) = a^x - a^{-x} \) 在 \( \mathbb{R} \) 上为减函数,故B错误;
    对于C,由题可知函数 \( y = \dfrac{f(2x)}{2f(x)} \) 的定义域为 \( \{ x \mid x \neq 0 \} \),关于原点对称,又因为 \( \dfrac{f(2x)}{2f(x)} = \dfrac{a^{2x} - a^{-2x}}{2(a^x - a^{-x})} = \dfrac{a^x + a^{-x}}{2} \),\( \dfrac{f(-2x)}{2f(-x)} = \dfrac{a^{-2x} - a^{2x}}{2(a^{-x} - a^x)} = \dfrac{a^{-x} + a^x}{2} \),所以 \( \dfrac{f(-2x)}{2f(-x)} = \dfrac{f(2x)}{2f(x)} \),即 \( y = \dfrac{f(2x)}{2f(x)} \) 为偶函数,故C正确;
    对于D,当 \( a > 1 \) 时,\( f(|x|) = \begin{cases} a^{-x} - a^x, x < 0, \\ 0, x = 0, \\ a^x - a^{-x}, x > 0, \end{cases} \) 故 \( f(|x|) \) 在 \( (-\infty, 0] \) 上为减函数,在 \( [0, +\infty) \) 上为增函数,所以当 \( x = 0 \) 时,\( f(|x|) \) 取得最小值0,故D正确。
\end{enumerate}

\section{填空题(本题共3小题,每小题5分,共15分)\\ }

\begin{enumerate}
    \item 答案:1
    命题透析:本题考查分段函数的性质。
    解析:由题意得 \( f(-1) = -1 + a = 0 \),所以 \( a = 1 \),当 \( x < 0 \) 时,\( f(x) = x + 1 < 1 \),当 \( x \geq 0 \) 时,\( f(x) = \left( \dfrac{1}{2} \right)^x \) 单调递减,所以 \( f(x)_{\max} = f(0) = 1 \),综上,\( f(x) \) 的最大值为1。

    \item 答案:197
    命题透析:本题考查集合的表示与运算,集合中的元素个数问题。
    解析:因为 \( A = \{ x \in \mathbb{N}^* \mid x \leq 100 \} \),所以 \( \text{card}(A) = 100 \),当 \( x > 0 \) 时,\( y = x^3 + 3 \) 单调递增,所以 \( \text{card}(B) = 100 \),又 \( 1^3 + 3 = 4, 2^3 + 3 = 17, 3^3 + 3 = 54, 4^3 + 3 = 145 \),所以 \( \text{card}(A \cap B) = 3 \)。所以 \( \text{card}(A \cup B) = \text{card}(A) + \text{card}(B) - \text{card}(A \cap B) = 100 + 100 - 3 = 197 \)。

    \item 答案:\( \sqrt{3} \)
    命题透析:本题考查函数图象的应用。
    解析:由 \( (|x| - a)(3 - x^2) \leq 0 \),得
    \[
    \begin{cases}
    |x| - a \geq 0, \\
    3 - x^2 \leq 0,
    \end{cases}
    \text{ 或 }
    \begin{cases}
    |x| - a \leq 0, \\
    3 - x^2 \geq 0,
    \end{cases}
    \]
    结合函数 \( y = |x| - a \) 和 \( y = 3 - x^2 \) 的图象可知,两图象与 \( x \) 轴的交点相同,由图可知 \( a = \sqrt{3} \)。
\end{enumerate}

\section{解答题(本题共5小题,共77分。解答应写出文字说明、证明过程或演算步骤)}

\begin{enumerate}
    \item 命题透析:本题考查指数与对数的综合运算。
    解析:
    \begin{enumerate}
        \item 原式 \( = \dfrac{5}{2} - \dfrac{1}{2} - \dfrac{3}{2} + \dfrac{1}{2} = 1 \)。 \hfill(4分)\\ 
        \item 原式 \( = 3 + \log_2 4 - 0 + 2 = 7 \)。 \hfill(8分)\\ 
        \item 因为 \( \left( x^{\dfrac{1}{2}} + x^{-\dfrac{1}{2}} \right)^2 = x + x^{-1} + 2 = 5 \),又 \( x^{\dfrac{1}{2}} + x^{-\dfrac{1}{2}} > 0 \),所以 \( x^{\dfrac{1}{2}} + x^{-\dfrac{1}{2}} = \sqrt{5} \), \hfill(10分)\\ 
        因为 \( x^2 + x^{-2} = (x + x^{-1})^2 - 2 = 7 \),所以 \( x^2 + x^{-2} - 2 = 5 \), \hfill(12分)\\ 
        故 \( \dfrac{x^{\dfrac{1}{2}} + x^{-\dfrac{1}{2}}}{x^2 + x^{-2} - 2} = \dfrac{\sqrt{5}}{5} \)。 \hfill(13分)\\ 
    \end{enumerate}

    \item 命题透析:本题考查集合的表示与运算。
    解析:
    \begin{enumerate}
        \item 当 \( m = \dfrac{4}{3} \) 时,\( M = \left\{ x \mid \dfrac{5}{3} < x < \dfrac{7}{3} \right\} \), \hfill(1分)\\ 
        \( N = \{ x \mid 3^x \geq 9 \} = \{ x \mid x \geq 2 \} \), \hfill(3分)\\ 
        所以 \( \complement_{\mathbb{R}} N = \{ x \mid x < 2 \} \), \hfill(5分)\\ 
        所以 \( M \cap (\complement_{\mathbb{R}} N) = \left\{ x \mid \dfrac{5}{3} < x < 2 \right\} \)。 \hfill(7分)\\ 
        \item 若 \( M = \varnothing \),则 \( 2m - 1 \geq m + 1 \),得 \( m \geq 2 \)。 \hfill(10分)\\ 
        若 \( M \neq \varnothing \),则 \( \begin{cases} 2m - 1 < m + 1, \\ 2m - 1 \geq 2, \end{cases} \) 解得 \( \dfrac{3}{2} \leq m < 2 \)。 \hfill(13分)\\ 
        综上,实数 \( m \) 的取值范围为 \( \left[ \dfrac{3}{2}, +\infty \right) \)。 \hfill(15分)\\ 
    \end{enumerate}

    \item 命题透析:本题考查奇函数的性质、函数的图象与性质。
    解析:
    \begin{enumerate}
        \item 当 \( x \leq 0 \) 时,\( f(x) = x^2 + 2x \),
        当 \( x > 0 \) 时,\( -x < 0 \),所以 \( f(-x) = (-x)^2 + 2(-x) = x^2 - 2x \),
        因为 \( f(x) \) 是奇函数,所以当 \( x > 0 \) 时,\( f(x) = -f(-x) = -x^2 + 2x \),
        故 \( f(x) = \begin{cases} -x^2 + 2x, x > 0, \\ x^2 + 2x, x \leq 0. \end{cases} \) \hfill(5分)\\ 
        \item 根据 \( f(x) \) 的解析式,可作出其图象(图略)。
        由图可知 \( f(x) \) 的单调递增区间为 \( [-1, 1] \)。 \hfill(10分)\\ 
        \item \( f(x) = t \) 有3个不相等的实数根,等价于 \( f(x) \) 与 \( y = t \) 的图象有3个交点。 \hfill(12分)\\ 
        由图象可知,当 \( t \in (-1, 1) \) 时,\( f(x) \) 与 \( y = t \) 的图象有3个交点,
        所以 \( t \) 的取值范围是 \( (-1, 1) \)。 \hfill(15分)\\ 
    \end{enumerate}

    \item 命题透析:本题考查二次函数的解析式及最值问题。
    解析:
    \begin{enumerate}
        \item 由题可设 \( f(x) = a(x + 1)(x - 3) \),又由 \( f(0) = a \times (-3) = -3 \),得 \( a = 1 \),
        所以 \( f(x) = (x + 1)(x - 3) = x^2 - 2x - 3 \)。 \hfill(4分)\\ 
        \item 由题有 \( x^2 - 2x - 3 > -x + 2m - 1 \),即 \( x^2 - x - 2 > 2m \) 对任意的 \( x \in [-2, 2] \) 恒成立,
        设 \( h(x) = x^2 - x - 2 = \left( x - \dfrac{1}{2} \right)^2 - \dfrac{9}{4}, x \in [-2, 2] \),则 \( h(x)_{\min} = h\left( \dfrac{1}{2} \right) = -\dfrac{9}{4} \)。 \hfill(7分)\\ 
        所以 \( 2m < -\dfrac{9}{4} \),解得 \( m < -\dfrac{9}{8} \),
        即 \( m \) 的取值范围是 \( \left( -\infty, -\dfrac{9}{8} \right] \)。 \hfill(9分)\\ 
        \item 由(1)可知 \( f(x) = (x - 1)^2 - 4 \),其图象的对称轴为直线 \( x = 1 \)。
        当 \( t \leq 1 \) 时,\( f(x) \) 在 \( [t - 1, t] \) 上单调递减,则 \( g(t) = f(t) = t^2 - 2t - 3 \); \hfill(11分)\\ 
        当 \( t - 1 < 1 < t \),即 \( 1 < t < 2 \) 时,\( f(x) \) 在 \( [t - 1, 1] \) 上单调递减,在 \( [1, t] \) 上单调递增,
        此时 \( g(t) = f(1) = -4 \); \hfill(14分)\\ 
        当 \( t - 1 \geq 1 \),即 \( t \geq 2 \) 时,\( f(x) \) 在 \( [t - 1, t] \) 上单调递增,
        此时 \( g(t) = f(t - 1) = t^2 - 4t \)。 \hfill(16分)\\ 
        综上,\( g(t) = \begin{cases} t^2 - 2t - 3, t \leq 1, \\ -4, 1 < t < 2, \\ t^2 - 4t, t \geq 2. \end{cases} \) \hfill(17分)\\ 
    \end{enumerate}

    \item 命题透析:本题考查抽象函数的性质,函数与不等式的综合。
    解析:
    \begin{enumerate}
        \item 在 \( f(a + b) = f(a) + f(b) - 1 \) 中,
        令 \( a = b = 0 \),得 \( f(0) = f(0) + f(0) - 1 \),所以 \( f(0) = 1 \)。 \hfill(3分)\\ 
        \item 设 \( x_1, x_2 \in \mathbb{R} \) 且 \( x_1 > x_2 \),取 \( a = x_1 - x_2, b = x_2 \),
        则 \( f(x_1) = f(x_1 - x_2) + f(x_2) - 1 \),即 \( f(x_1) - f(x_2) = f(x_1 - x_2) - 1 \)。 \hfill(5分)\\ 
        由于当 \( x > 0 \) 时,\( f(x) > 1 \),而 \( x_1 - x_2 > 0 \),所以 \( f(x_1 - x_2) > 1 \),
        即 \( f(x_1) - f(x_2) = f(x_1 - x_2) - 1 > 0 \),则 \( f(x_1) > f(x_2) \), \hfill(7分)\\ 
        故 \( f(x) \) 是 \( \mathbb{R} \) 上的增函数。 \hfill(8分)\\ 
        \item 不等式 \( f(t \cdot 4^x) + f(3 \times 2^x - 1) > 2 \) 等价于 \( f(t \cdot 4^x + 3 \times 2^x - 1) > 1 = f(0) \), \hfill(10分)\\ 
        由(2)可知 \( f(x) \) 是 \( \mathbb{R} \) 上的增函数,
        所以 \( t \cdot 4^x + 3 \times 2^x - 1 > 0 \iff t > \left( \dfrac{1}{4} \right)^x - 3 \cdot \left( \dfrac{1}{2} \right)^x \) 在 \( \mathbb{R} \) 上能成立。 \hfill(12分)\\ 
        下面求函数 \( y = \left( \dfrac{1}{4} \right)^x - 3 \cdot \left( \dfrac{1}{2} \right)^x \) 的最小值:
        令 \( m = \left( \dfrac{1}{2} \right)^x \),则 \( m > 0 \),\( y = m^2 - 3m = \left( m - \dfrac{3}{2} \right)^2 - \dfrac{9}{4} \),
        所以当 \( m = \dfrac{3}{2} \) 时,\( y_{\min} = -\dfrac{9}{4} \),
        所以 \( t > -\dfrac{9}{4} \),即 \( t \) 的取值范围是 \( \left( -\dfrac{9}{4}, +\infty \right) \)。 \hfill(17分)\\ 
    \end{enumerate}

\end{enumerate}

\end{document}