\section{2025年10月数学联考}

\begin{liti}
不等式$ax^2+ax-1<0$对一切实数$x$恒成立的$a$的取值集合为$\mathbb{A}$,集合$\mathbb{B}=\{x|x^2+mx-2m-2<0\}$.
\begin{enumerate}
\item 求集合$\mathbb{A}$;
\item 若“$x\in\mathbb{A}$”是“$x\in\mathbb{B}$”的充分条件,求$m$的取值范围.
\end{enumerate}

\tcblower
\begin{enumerate}
\item 设函数$f(x)=ax^2+ax-1$。要使不等式$f(x)<0$对一切实数$x$恒成立, 必须分情况讨论$a$的取值.
\begin{itemize}
    \item 若$a>0$, 则二次函数开口向上, 当$x\to\infty$时, $f(x)\to+\infty$, 不可能恒小于0.
    \item 若$a=0$, 不等式变为$-1<0$, 恒成立. 所以$a=0$是解的一部分.
    \item 若$a<0$, 二次函数开口向下. 为使$f(x)<0$恒成立, 其图像必须完全在$x$轴下方, 这意味着它与$x$轴没有交点, 即判别式必须小于0.
    $\Delta = a^2 - 4(a)(-1) = a^2+4a < 0$.
    解得$a(a+4)<0$, 即$-4 < a < 0$.
\end{itemize}
综合以上情况, $a$的取值范围是$(-4, 0]$. 因此, $\mathbb{A}=\{a|-4<a\le 0\}=(-4,0]$.

\item 题目条件“$x\in\mathbb{A}$”是“$x\in\mathbb{B}$”的充分条件, 即$\mathbb{A}\subseteq\mathbb{B}$.\\
将第(1)问求出的集合$\mathbb{A}$代入, 有$(-4,0]\subseteq\{x|x^2+mx-2m-2<0\}$.\\
设$g(x)=x^2+mx-2m-2$. 要使$(-4,0]\subseteq\{x|g(x)<0\}$, 需要$g(x)<0$在区间$(-4,0]$上恒成立.\\
由于$g(x)$是一个开口向上的抛物线, 要使其在闭区间上小于0, 只需保证在区间端点处函数值小于等于0即可.\\
即$g(-4)\le 0$且$g(0)\le 0$.
\begin{itemize}
    \item $g(-4) = (-4)^2 + m(-4) - 2m - 2 = 16 - 4m - 2m - 2 = 14 - 6m \le 0 \implies m \ge \frac{14}{6} = \frac{7}{3}$.
    \item $g(0) = 0^2 + m(0) - 2m - 2 = -2m - 2 \le 0 \implies -2m \le 2 \implies m \ge -1$.
\end{itemize}
要同时满足两个条件, 必须取它们的交集, 即$m\ge\frac{7}{3}$.
因此, $m$的取值范围是$[\frac{7}{3}, +\infty)$.
\end{enumerate}
\end{liti}

\begin{liti}
\begin{enumerate}
\item 若$ax^2-ax+1>0$的解集为$\{x|x \neq b\}$,求$a,b$的值;
\item 当$x\geq 1$时,$ax^2-ax+1>-a+2$恒成立,求$a$的取值范围;
\item 求关于$x$的不等式$ax^2-ax+1>3x-2$的解集。
\end{enumerate}
\tcblower

\begin{enumerate}
\item 因为不等式$ax^2-ax+1>0$的解集为$\{x|x \neq b\}$, 说明抛物线$y=ax^2-ax+1$开口向上, 且与$x$轴只有一个交点.\\
所以$a>0$且判别式$\Delta = (-a)^2 - 4a = a^2-4a=0$.\\
解得$a=4$.\\
此时不等式为$4x^2-4x+1>0$, 即$(2x-1)^2>0$, 解集为$\{x|x\neq \frac{1}{2}\}$.\\
所以$b=\frac{1}{2}$.\\
综上, $a=4, b=\frac{1}{2}$.

\item 当$x\geq 1$时, $ax^2-ax+1>-a+2$恒成立, 即$a(x^2-x+1)-1>0$恒成立.\\
设$g(x)=a(x^2-x+1)-1$.\\
令$h(x)=x^2-x+1=(x-\frac{1}{2})^2+\frac{3}{4}$.\\
当$x\ge 1$时, $h(x)$是增函数, 所以$h(x)_{\min}=h(1)=1$.\\
\begin{itemize}
    \item 若$a>0$, $g(x)$的最小值为$a \cdot h(x)_{\min}-1 = a-1$.
    要使$g(x)>0$恒成立, 只需$a-1>0$, 解得$a>1$.
    \item 若$a=0$, 不等式变为$1>2$, 不成立.
    \item 若$a<0$, $g(x)$随$h(x)$的增大而减小. 由于$h(x)$在$[1, +\infty)$上无上界, $g(x)$无下界, 不可能恒大于0.
\end{itemize}
所以$a$的取值范围是$(1, +\infty)$.

\item 不等式$ax^2-ax+1>3x-2$可化为$ax^2-(a+3)x+3>0$.
因式分解得$(ax-3)(x-1)>0$.
\begin{itemize}
    \item 当$a>0$时, 抛物线开口向上. 两根为$x=1$和$x=3/a$.
    \begin{itemize}
        \item 若$a>3$, 则$3/a < 1$. 解集为$x<3/a$或$x>1$.
        \item 若$a=3$, 则$3/a=1$. 不等式为$3(x-1)^2>0$, 解集为$x\neq 1$.
        \item 若$0<a<3$, 则$3/a > 1$. 解集为$x<1$或$x>3/a$.
    \end{itemize}
    \item 当$a=0$时, 不等式为$-3x+3>0$, 解集为$x<1$.
    \item 当$a<0$时, 抛物线开口向下. 两根$x=1$和$x=3/a$. 因为$a<0$, 所以$3/a<0<1$.
    解集为$3/a < x < 1$.
\end{itemize}
\end{enumerate}

\end{liti}