\documentclass[CJKmath,a4paper,10pt]{ctexart}
\usepackage{geometry}
\geometry{inner=1.5cm,outer=1.5cm,top=2.45cm,bottom=1cm}
\usepackage[dvipsnames,svgnames,x11names,table]{xcolor}


\RequirePackage{amssymb} % Must be loaded before unicode-math
\RequirePackage{unicode-math} % Math fonts in xetexorluatex
%\setmathfont{texgyrepagella-math.otf}
\setmainfont{STIXTwoText}[
	Path=./fonts/STIXTwoText/,
  Extension = .otf,
  UprightFont = STIXTwoMath-Regular,
  BoldFont = STIXTwoText-SemiBold,
  ItalicFont = STIXTwoText-Italic,
  BoldItalicFont = STIXTwoText-BoldItalic,
]
\setmathfont{STIXTwoMath-Regular.otf}

\usepackage{tikz,calc}
\usetikzlibrary{calc,arrows,shadows.blur,tikzmark}
% By default all math in TikZ nodes are set in inline mode. Change this to
% displaystyle so that we don't get small fractions.
\everymath{\displaystyle}

\usepackage[most]{tcolorbox}
\tcbuselibrary{breakable, skins,theorems}


\usepackage{xkeyval}
\makeatletter 
\usepackage{amsmath,amssymb,amsthm}
\usepackage{xkeyval} 
\usepackage{zhlipsum}
\tcbset{
    common/.style={
      fontupper=\rmfamily,
      lower separated=true,
      coltitle=black,
      colback=gray!5,
      boxrule=0.5pt,
      fonttitle=\bfseries,
      enhanced,
      breakable,
      top=8pt,
      before skip=8pt,
      attach boxed title to top left={yshift=-0.05in,xshift=0.15in},
      boxed title style={
      	boxrule=0pt,
        colframe=black,
        arc=0pt,
        outer arc=0pt,
        drop fuzzy shadow
     	},
      separator sign={.},
      drop fuzzy shadow
     },
     litistyle/.style={  %例题风格
     	common,
      colframe=gray,% 
      colback=white,
      colbacktitle=Gold,
      sharp corners,rounded corners=southeast,
      overlay unbroken and last={
      \node[anchor=south east, outer sep=0pt] at (\linewidth-width,0){\textcolor{black!60}{$\blacksquare$}};}
         },
% -----------------------------------------------------------------------------
% Explanation: BreakableBox@title key
% The following defines a pgfkeys/tcolorbox key named
%   "BreakableBox@title" which accepts 2 arguments (see 
%   "/.code n args={2}"). When the key is used like
%   "BreakableBox@title={<env>}{<optTitle>}" the code below is executed
%   with #1=<env> and #2=<optTitle>.
%
% Purpose:
% - If the second argument (#2) is empty, set the tcolorbox title to
%     \csname <env>name\endcsname~\thetcbcounter
%   which expands to the environment name (e.g. "liti" -> \litiname)
%   followed by the current counter value.
% - If the second argument is provided, append " (#2)" to the title.
%
% Notes:
% - This relies on \makeatletter being active so that keys with '@'
%   are allowed in control sequence names.
% - \ifblank is used to test whether #2 is empty; ensure the package
%   that provides \ifblank (for example, etoolbox or xparse) is loaded
%   earlier if necessary.
% - Example usage inside the code: \tcbset{title={...}} sets the
%   tcolorbox title key accordingly.
%
         BreakableBox@title/.code n args={2}
              {
                \ifblank{#2}
                  {\tcbset{title={\csname #1name\endcsname~\thetcbcounter}}}
                  {\tcbset{title={\csname #1name\endcsname~\thetcbcounter\ (#2)}}}
              },
}      
  % define an internal control sequence \newbreakablebox for fancy mode's newtheorem
  % #1 is the environment name, #2 is the prefix of label, #3 is the style
  % style: thmstyle, defstyle, prostyle
  % e.g. \newbreakablebox{theorem}{thm}{thmstyle}
  % will define two environments: numbered ``theorem'' and no-numbered ``theorem*''
  % WARNING FOR MULTILINGUAL: this cs will automatically find \theoremname's definition,
  % WARNING FOR MULTILINGUAL: it should be defined in language settings.
  \newcommand{\newbreakablebox}[3]{
    \ifcsundef{#1name}{%
      % define a default name if not defined before
      \tcbset{BreakableBox@title/.code n args={2}{\tcbset{title={use newcommand define #1name}}} }
    }{\relax}
    \DeclareTColorBox[auto counter,number within=section]{#1}{ g o t\label g }{
        common, % use common style
        #3, % use the style passed in
        IfValueTF={##1}
          {BreakableBox@title={#1}{##1}}
          {
            IfValueTF={##2}
            {BreakableBox@title={#1}{##2}}
            {BreakableBox@title={#1}{}}
          },
        IfValueT={##4}
          {
            IfBooleanTF={##3}
              {label={##4}}
              {VIVID@label={#2}{##4}}
          }
      }
    \DeclareTColorBox{#1*}{ g o }{
        common,#3,
        IfValueTF={##1}
          {BreakableBox@title={#1}{##1}}
          {
            IfValueTF={##2}
            {BreakableBox@title={#1}{##2}}
            {BreakableBox@title={#1}{}}
          },
      }
  }
  % define several environment 
  % we define headers like \definitionname before
  \newcommand{\litiname}{例题}
  \newbreakablebox{liti}{def}{litistyle}
  
  
%补充内容
%%设置新字体
%%定义带圈数字命令
\newfontfamily{\nmfont}{circlenumber}
[%
Extension=.otf,
Path=./fonts/]

\newcommand{\quan}[1]{{\nmfont \symbol{#1}}}
\newcommand{\kk}[1]{\quan{\numexpr32+#1}}%\kk{<参数范围1-95>}96、97、98、99分别用\quan{196} \quan{197} \quan{199} \quan{201}

%脚注使用带圈数字
\newcommand*\kkctr[1]{%
  \protect\kk{\number\numexpr\value{#1}\relax}}
\renewcommand*\thefootnote{\textcolor{black}{\kkctr{footnote}}}

%%无悬挂脚注格式
\renewcommand\@makefntext[1]{%
  \setlength\parindent{2\ccwd}\selectfont
  \@thefnmark\ #1}

%修改\part,使其不分页
\def\@endpart{%
	\thispagestyle{empty}
  \vskip40\p@%
   \@afterheading}



\RequirePackage{enumitem}
%\newenvironment{myenum}{\begin{enumerate}[label=\protect\kk{\arabic*}]\small}{\end{enumerate}}%
\setlist{noitemsep}
\setlist[enumerate, 1]{label=\protect\kk{\arabic*},itemsep=0.5ex}  
\makeatother



%%%marker环境
\newtcolorbox{marker}[1][]{enhanced,before skip=2mm,
	after skip=3mm,fontupper=\rmfamily,
	boxrule=0.4pt,left=5mm,right=2mm,top=1mm,bottom=1mm,
	colback=yellow!50,colframe=yellow!20!black,
	sharp corners,rounded corners=southeast,
	arc is angular,arc=3mm,underlay={%
		\path[fill=tcbcolback!80!black] ([yshift=3mm]interior.south east)--++(-0.4,-0.1)--++(0.1,-0.2);
		\path[draw=tcbcolframe,shorten <=-0.05mm,shorten >=-0.05mm] ([yshift=3mm]interior.south east)--++(-0.4,-0.1)--++(0.1,-0.2);
		\path[fill=yellow!50!black,draw=none] (interior.south west) rectangle node[white]{\Huge\bfseries !} ([xshift=4mm]interior.north west);
	},
	drop fuzzy shadow,#1
}

\usepackage{exam-zh-choices}
\usepackage{fontawesome5}
\begin{document}



\section{选择题}

\begin{liti}
已知函数\( f(x) \)的定义域为\( (-\infty,0)\cup(0,+\infty) \),且\( f(x)=f(-x) \),当\( x_1,x_2\in(0,+\infty) \)时,\( (x_1 - x_2)\left(\dfrac{f(x_1)}{x_1} - \dfrac{f(x_2)}{x_2}\right) < 0 \)恒成立.若\( f(1)=0 \),则不等式\( \dfrac{f(x)}{x} > 0 \)的解集为
    \begin{choices}
			\item \( (-1,0)\cup(1,+\infty) \)
			\item \( (-\infty,-1)\cup(1,+\infty) \)
			\item \( (-1,0)\cup(0,1) \)
			\item \( (-\infty,-1)\cup(0,1) \)
    \end{choices}
\tcblower
\begin{itemize}
\item  \( g(x) = \dfrac{f(x)}{x} \)是奇函数,且在\( (-\infty, 0) \)和\( (0, +\infty) \)上均为减函数,同时\( g(1) = g(-1) = 0 \)。

\item 分区间讨论\( g(x) > 0 \):
   \begin{itemize}
   \item 当\( x \in (0, +\infty) \)时,\( g(x) \)递减且\( g(1) = 0 \),故\( 0 < x < 1 \)时\( g(x) > 0 \);
   \item  当\( x \in (-\infty, 0) \)时,\( g(x) \)递减且\( g(-1) = 0 \),故\( x < -1 \)时\( g(x) > 0 \)。
   \end{itemize}

\item  综上,\( g(x) > 0 \)的解集为\( (-\infty, -1) \cup (0, 1) \),即不等式\( \dfrac{f(x)}{x} > 0 \)的解集为\( \underline{(-\infty, -1) \cup (0, 1)} \)(对应选项D)。
\end{itemize}
\end{liti}

\begin{liti}
已知\( a > 0 \),\( b > 0 \),且\( ab = a + b + 8 \),下列说法正确的是
\begin{choices}
\item \( a + b \)的最小值为8
\item \( a^2 + b^2 \)的最大值为32
\item \( \dfrac{1}{1 - a} + \dfrac{1}{1 - b} \)的最大值为\(-\dfrac{2}{3}\)
\item \( 3a + b \)的最小值为\( 6\sqrt{3} + 4 \)
\end{choices}
\tcblower
\begin{itemize}
\item 
由\( a > 0, b > 0 \),根据基本不等式\( ab \leq \left( \dfrac{a + b}{2} \right)^2 \),结合\( ab = a + b + 8 \),令\( t = a + b (t > 0) \),则:\\
$
t + 8 \leq \left( \dfrac{t}{2} \right)^2
$ 
整理得\( t^2 - 4t - 32 \geq 0 \),即\( (t - 8)(t + 4) \geq 0 \)。
因为\( t > 0 \),所以\( t \geq 8 \),当且仅当\( a = b = 4 \)时取等号。
故\( a + b \)的最小值为\( 8 \),\faCheck


\item 
由\( a^2 + b^2 = (a + b)^2 - 2ab \),结合\( ab = a + b + 8 \)和\( a + b \geq 8 \),令\( t = a + b (t \geq 8) \),则\( ab = t + 8 \),因此:
$
a^2 + b^2 = t^2 - 2(t + 8) = t^2 - 2t - 16
$
函数\( y = t^2 - 2t - 16 \)在\( t \geq 8 \)时单调递增,当\( t = 8 \)时,\( y = 8^2 - 2 \times 8 - 16 = 32 \),且\( t \)越大,\( y \)越大,无最大值。
故\( a^2 + b^2 \)无最大值,\faTimes

\item 
$ \dfrac{1}{1 - a} + \dfrac{1}{1 - b} \iff
\dfrac{1}{1 - a} + \dfrac{1}{1 - b} = \dfrac{(1 - b) + (1 - a)}{(1 - a)(1 - b)} = \dfrac{2 - (a + b)}{1 - (a + b) + ab}
$

由\( ab = a + b + 8 \),代入得:
$
\dfrac{2 - (a + b)}{1 - (a + b) + (a + b + 8)} = \dfrac{2 - t}{9} \quad (t = a + b \geq 8)
$\\
\(\because t \geq 8 ,\therefore 2 - t \leq -6 \),则\( \dfrac{2 - t}{9} \leq -\dfrac{2}{3} \),当且仅当\( a = b = 4 \)时取等号。
故\( \dfrac{1}{1 - a} + \dfrac{1}{1 - b} \)的最大值为\( -\dfrac{2}{3} \),\faCheck


\item 
\( ab = a + b + 8 \iff b = \dfrac{a + 8}{a - 1} (a > 1) \),则\( 3a + b = 3a + \dfrac{a + 8}{a - 1} \)。\\
令\( a - 1 = m (m > 0) \),则\( a = m + 1 \),代入得:
$
3(m + 1) + \dfrac{m + 1 + 8}{m} = 3m + 3 + \dfrac{m + 9}{m} = 4m + \dfrac{9}{m} + 4
$\\
根据基本不等式\( 4m + \dfrac{9}{m} \geq 2\sqrt{4m \cdot \dfrac{9}{m}} = 12 \),当且仅当\( 4m = \dfrac{9}{m} \)即\( m = \dfrac{3}{2} \)时取等号,此时\( a = \dfrac{5}{2} \),\( b = 7 \)。
故\( 3a + b \)的最小值为\( 12 + 4 = 16 \),而非\( 6\sqrt{3} + 4 \),\faTimes。
\end{itemize}


综上,正确答案为\(\boxed{AC}\)。
\end{liti}

\begin{liti}
已知命题$p:\forall x\in \mathbb{R},x^2−mx+m>0$,命题$q: 集合A=\{x\big|mx^2 −3x+1=0,m\in\mathbb{R}\}$中至多有一个元素.
\begin{enumerate}
\item 若p为真命题,求实数m的取值范围;
\item 若q为真命题,求实数m的取值范围.
\end{enumerate}
\tcblower
\begin{itemize}
\item  若\( p \)为真命题,求实数\( m \)的取值范围
命题\( p \): \( \forall x \in \mathbf{R} \),\( x^2 - mx + m > 0 \)为真,说明二次函数\( y = x^2 - mx + m \)的图象在\( \mathbf{R} \)上恒在\( x \)轴上方。
对于二次函数\( y = ax^2 + bx + c (a \neq 0) \),当\( a > 0 \)且判别式\( \Delta < 0 \)时,函数图象恒在\( x \)轴上方。
此处\( a = 1 > 0 \),判别式\( \Delta = (-m)^2 - 4 \times 1 \times m = m^2 - 4m \),令\( \Delta < 0 \),即:
\[
m^2 - 4m < 0 \implies m(m - 4) < 0
\]
解得\( 0 < m < 4 \)。

因此,当\( p \)为真命题时,实数\( m \)的取值范围是\( \boxed{(0, 4)} \)。


\item  若\( q \)为真命题,求实数\( m \)的取值范围
命题\( q \): 集合\( A = \{ x \mid mx^2 - 3x + 1 = 0, m \in \mathbf{R} \} \)中至多有一个元素,即方程\( mx^2 - 3x + 1 = 0 \)\textbf{无实根}或\textbf{有且仅有一个实根}。

分情况讨论:

\begin{itemize}
\item  \textbf{当\( m = 0 \)时},方程化为\( -3x + 1 = 0 \),解得\( x = \frac{1}{3} \),此时集合\( A \)有一个元素,符合“至多一个元素”的条件。

\item  \textbf{当\( m \neq 0 \)时},方程为二次方程,判别式\( \Delta = (-3)^2 - 4 \times m \times 1 = 9 - 4m \)。
  \begin{itemize}
  \item  若方程无实根,则\( \Delta < 0 \),即\( 9 - 4m < 0 \implies m > \frac{9}{4} \);
  \item  若方程有且仅有一个实根,则\( \Delta = 0 \),即\( 9 - 4m = 0 \implies m = \frac{9}{4} \)。
  \end{itemize}

\end{itemize}
\end{itemize}
综上,当\( m = 0 \)或\( m \geq \frac{9}{4} \)时,命题\( q \)为真。
因此,实数\( m \)的取值范围是\( \boxed{\{ 0 \} \cup \left[ \frac{9}{4}, +\infty \right)} \)。
\end{liti}
\end{document}