\documentclass[CJKmath,a4paper,10pt]{ctexart}
\usepackage{geometry}
\geometry{inner=1.5cm,outer=1.5cm,top=2.45cm,bottom=1cm}
\usepackage[dvipsnames,svgnames,x11names,table]{xcolor}


\RequirePackage{amssymb} % Must be loaded before unicode-math
\RequirePackage{unicode-math} % Math fonts in xetexorluatex
%\setmathfont{texgyrepagella-math.otf}
\setmainfont{STIXTwoText}[
	Path=./fonts/STIXTwoText/,
  Extension = .otf,
  UprightFont = STIXTwoMath-Regular,
  BoldFont = STIXTwoText-SemiBold,
  ItalicFont = STIXTwoText-Italic,
  BoldItalicFont = STIXTwoText-BoldItalic,
]
\setmathfont{STIXTwoMath-Regular.otf}

\usepackage{tikz,calc}
\usetikzlibrary{calc,arrows,shadows.blur,tikzmark}
% By default all math in TikZ nodes are set in inline mode. Change this to
% displaystyle so that we don't get small fractions.
\everymath{\displaystyle}

\usepackage[most]{tcolorbox}
\tcbuselibrary{breakable, skins,theorems}


\usepackage{xkeyval}
\makeatletter 
\usepackage{amsmath,amssymb,amsthm}
\usepackage{xkeyval} 
\usepackage{zhlipsum}
\tcbset{
    common/.style={
      fontupper=\rmfamily,
      lower separated=true,
      coltitle=black,
      colback=gray!5,
      boxrule=0.5pt,
      fonttitle=\bfseries,
      enhanced,
      breakable,
      top=8pt,
      before skip=8pt,
      attach boxed title to top left={yshift=-0.05in,xshift=0.15in},
      boxed title style={
      	boxrule=0pt,
        colframe=black,
        arc=0pt,
        outer arc=0pt,
        drop fuzzy shadow
     	},
      separator sign={.},
      drop fuzzy shadow
     },
     litistyle/.style={  %例题风格
     	common,
      colframe=gray,% 
      colback=white,
      colbacktitle=Gold,
      sharp corners,rounded corners=southeast,
      overlay unbroken and last={
      \node[anchor=south east, outer sep=0pt] at (\linewidth-width,0){\textcolor{black!60}{$\blacksquare$}};}
         },
% -----------------------------------------------------------------------------
% Explanation: BreakableBox@title key
% The following defines a pgfkeys/tcolorbox key named
%   "BreakableBox@title" which accepts 2 arguments (see 
%   "/.code n args={2}"). When the key is used like
%   "BreakableBox@title={<env>}{<optTitle>}" the code below is executed
%   with #1=<env> and #2=<optTitle>.
%
% Purpose:
% - If the second argument (#2) is empty, set the tcolorbox title to
%     \csname <env>name\endcsname~\thetcbcounter
%   which expands to the environment name (e.g. "liti" -> \litiname)
%   followed by the current counter value.
% - If the second argument is provided, append " (#2)" to the title.
%
% Notes:
% - This relies on \makeatletter being active so that keys with '@'
%   are allowed in control sequence names.
% - \ifblank is used to test whether #2 is empty; ensure the package
%   that provides \ifblank (for example, etoolbox or xparse) is loaded
%   earlier if necessary.
% - Example usage inside the code: \tcbset{title={...}} sets the
%   tcolorbox title key accordingly.
%
         BreakableBox@title/.code n args={2}
              {
                \ifblank{#2}
                  {\tcbset{title={\csname #1name\endcsname~\thetcbcounter}}}
                  {\tcbset{title={\csname #1name\endcsname~\thetcbcounter\ (#2)}}}
              },
}      
  % define an internal control sequence \newbreakablebox for fancy mode's newtheorem
  % #1 is the environment name, #2 is the prefix of label, #3 is the style
  % style: thmstyle, defstyle, prostyle
  % e.g. \newbreakablebox{theorem}{thm}{thmstyle}
  % will define two environments: numbered ``theorem'' and no-numbered ``theorem*''
  % WARNING FOR MULTILINGUAL: this cs will automatically find \theoremname's definition,
  % WARNING FOR MULTILINGUAL: it should be defined in language settings.
  \newcommand{\newbreakablebox}[3]{
    \ifcsundef{#1name}{%
      % define a default name if not defined before
      \tcbset{BreakableBox@title/.code n args={2}{\tcbset{title={use newcommand define #1name}}} }
    }{\relax}
    \DeclareTColorBox[auto counter,number within=section]{#1}{ g o t\label g }{
        common, % use common style
        #3, % use the style passed in
        IfValueTF={##1}
          {BreakableBox@title={#1}{##1}}
          {
            IfValueTF={##2}
            {BreakableBox@title={#1}{##2}}
            {BreakableBox@title={#1}{}}
          },
        IfValueT={##4}
          {
            IfBooleanTF={##3}
              {label={##4}}
              {VIVID@label={#2}{##4}}
          }
      }
    \DeclareTColorBox{#1*}{ g o }{
        common,#3,
        IfValueTF={##1}
          {BreakableBox@title={#1}{##1}}
          {
            IfValueTF={##2}
            {BreakableBox@title={#1}{##2}}
            {BreakableBox@title={#1}{}}
          },
      }
  }
  % define several environment 
  % we define headers like \definitionname before
  \newcommand{\litiname}{例题}
  \newbreakablebox{liti}{def}{litistyle}
  
  
%补充内容
%%设置新字体
%%定义带圈数字命令
\newfontfamily{\nmfont}{circlenumber}
[%
Extension=.otf,
Path=./fonts/]

\newcommand{\quan}[1]{{\nmfont \symbol{#1}}}
\newcommand{\kk}[1]{\quan{\numexpr32+#1}}%\kk{<参数范围1-95>}96、97、98、99分别用\quan{196} \quan{197} \quan{199} \quan{201}

%脚注使用带圈数字
\newcommand*\kkctr[1]{%
  \protect\kk{\number\numexpr\value{#1}\relax}}
\renewcommand*\thefootnote{\textcolor{black}{\kkctr{footnote}}}

%%无悬挂脚注格式
\renewcommand\@makefntext[1]{%
  \setlength\parindent{2\ccwd}\selectfont
  \@thefnmark\ #1}

%修改\part,使其不分页
\def\@endpart{%
	\thispagestyle{empty}
  \vskip40\p@%
   \@afterheading}



\RequirePackage{enumitem}
%\newenvironment{myenum}{\begin{enumerate}[label=\protect\kk{\arabic*}]\small}{\end{enumerate}}%
\setlist{noitemsep}
\setlist[enumerate, 1]{label=\protect\kk{\arabic*},itemsep=0.5ex}  
\makeatother



%%%marker环境
\newtcolorbox{marker}[1][]{enhanced,before skip=2mm,
	after skip=3mm,fontupper=\rmfamily,
	boxrule=0.4pt,left=5mm,right=2mm,top=1mm,bottom=1mm,
	colback=yellow!50,colframe=yellow!20!black,
	sharp corners,rounded corners=southeast,
	arc is angular,arc=3mm,underlay={%
		\path[fill=tcbcolback!80!black] ([yshift=3mm]interior.south east)--++(-0.4,-0.1)--++(0.1,-0.2);
		\path[draw=tcbcolframe,shorten <=-0.05mm,shorten >=-0.05mm] ([yshift=3mm]interior.south east)--++(-0.4,-0.1)--++(0.1,-0.2);
		\path[fill=yellow!50!black,draw=none] (interior.south west) rectangle node[white]{\Huge\bfseries !} ([xshift=4mm]interior.north west);
	},
	drop fuzzy shadow,#1
}

\usepackage{exam-zh-choices}
\usepackage{fontawesome5}
\begin{document}
\section{2025年10月数学联考}

\begin{liti}
不等式$ax^2+ax-1<0$对一切实数$x$恒成立的$a$的取值集合为$\mathbb{A}$,集合$\mathbb{B}=\{x|x^2+mx-2m-2<0\}$.
\begin{enumerate}
\item 求集合$\mathbb{A}$;
\item 若“$x\in\mathbb{A}$”是“$x\in\mathbb{B}$”的充分条件,求$m$的取值范围.
\end{enumerate}

\tcblower
\begin{enumerate}
\item 设函数$f(x)=ax^2+ax-1$。要使不等式$f(x)<0$对一切实数$x$恒成立, 必须分情况讨论$a$的取值.
\begin{itemize}
    \item 若$a>0$, 则二次函数开口向上, 当$x\to\infty$时, $f(x)\to+\infty$, 不可能恒小于0.
    \item 若$a=0$, 不等式变为$-1<0$, 恒成立. 所以$a=0$是解的一部分.
    \item 若$a<0$, 二次函数开口向下. 为使$f(x)<0$恒成立, 其图像必须完全在$x$轴下方, 这意味着它与$x$轴没有交点, 即判别式必须小于0.
    $\Delta = a^2 - 4(a)(-1) = a^2+4a < 0$.
    解得$a(a+4)<0$, 即$-4 < a < 0$.
\end{itemize}
综合以上情况, $a$的取值范围是$(-4, 0]$. 因此, $\mathbb{A}=\{a|-4<a\le 0\}=(-4,0]$.

\item 题目条件“$x\in\mathbb{A}$”是“$x\in\mathbb{B}$”的充分条件, 即$\mathbb{A}\subseteq\mathbb{B}$.\\
将第(1)问求出的集合$\mathbb{A}$代入, 有$(-4,0]\subseteq\{x|x^2+mx-2m-2<0\}$.\\
设$g(x)=x^2+mx-2m-2$. 要使$(-4,0]\subseteq\{x|g(x)<0\}$, 需要$g(x)<0$在区间$(-4,0]$上恒成立.\\
由于$g(x)$是一个开口向上的抛物线, 要使其在闭区间上小于0, 只需保证在区间端点处函数值小于等于0即可.\\
即$g(-4)\le 0$且$g(0)\le 0$.
\begin{itemize}
    \item $g(-4) = (-4)^2 + m(-4) - 2m - 2 = 16 - 4m - 2m - 2 = 14 - 6m \le 0 \implies m \ge \frac{14}{6} = \frac{7}{3}$.
    \item $g(0) = 0^2 + m(0) - 2m - 2 = -2m - 2 \le 0 \implies -2m \le 2 \implies m \ge -1$.
\end{itemize}
要同时满足两个条件, 必须取它们的交集, 即$m\ge\frac{7}{3}$.
因此, $m$的取值范围是$[\frac{7}{3}, +\infty)$.
\end{enumerate}
\end{liti}

\begin{liti}
\begin{enumerate}
\item 若$ax^2-ax+1>0$的解集为$\{x|x \neq b\}$,求$a,b$的值;
\item 当$x\geq 1$时,$ax^2-ax+1>-a+2$恒成立,求$a$的取值范围;
\item 求关于$x$的不等式$ax^2-ax+1>3x-2$的解集。
\end{enumerate}
\tcblower

\begin{enumerate}
\item 因为不等式$ax^2-ax+1>0$的解集为$\{x|x \neq b\}$, 说明抛物线$y=ax^2-ax+1$开口向上, 且与$x$轴只有一个交点.\\
所以$a>0$且判别式$\Delta = (-a)^2 - 4a = a^2-4a=0$.\\
解得$a=4$.\\
此时不等式为$4x^2-4x+1>0$, 即$(2x-1)^2>0$, 解集为$\{x|x\neq \frac{1}{2}\}$.\\
所以$b=\frac{1}{2}$.\\
综上, $a=4, b=\frac{1}{2}$.

\item 当$x\geq 1$时, $ax^2-ax+1>-a+2$恒成立, 即$a(x^2-x+1)-1>0$恒成立.\\
设$g(x)=a(x^2-x+1)-1$.\\
令$h(x)=x^2-x+1=(x-\frac{1}{2})^2+\frac{3}{4}$.\\
当$x\ge 1$时, $h(x)$是增函数, 所以$h(x)_{\min}=h(1)=1$.\\
\begin{itemize}
    \item 若$a>0$, $g(x)$的最小值为$a \cdot h(x)_{\min}-1 = a-1$.
    要使$g(x)>0$恒成立, 只需$a-1>0$, 解得$a>1$.
    \item 若$a=0$, 不等式变为$1>2$, 不成立.
    \item 若$a<0$, $g(x)$随$h(x)$的增大而减小. 由于$h(x)$在$[1, +\infty)$上无上界, $g(x)$无下界, 不可能恒大于0.
\end{itemize}
所以$a$的取值范围是$(1, +\infty)$.

\item 不等式$ax^2-ax+1>3x-2$可化为$ax^2-(a+3)x+3>0$.
因式分解得$(ax-3)(x-1)>0$.
\begin{itemize}
    \item 当$a>0$时, 抛物线开口向上. 两根为$x=1$和$x=3/a$.
    \begin{itemize}
        \item 若$a>3$, 则$3/a < 1$. 解集为$x<3/a$或$x>1$.
        \item 若$a=3$, 则$3/a=1$. 不等式为$3(x-1)^2>0$, 解集为$x\neq 1$.
        \item 若$0<a<3$, 则$3/a > 1$. 解集为$x<1$或$x>3/a$.
    \end{itemize}
    \item 当$a=0$时, 不等式为$-3x+3>0$, 解集为$x<1$.
    \item 当$a<0$时, 抛物线开口向下. 两根$x=1$和$x=3/a$. 因为$a<0$, 所以$3/a<0<1$.
    解集为$3/a < x < 1$.
\end{itemize}
\end{enumerate}

\end{liti}
\end{document}
