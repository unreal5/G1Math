\documentclass[CJKmath,a4paper,10pt]{ctexart}
\usepackage{geometry}
\geometry{inner=1.5cm,outer=1.5cm,top=2.45cm,bottom=1cm}
\usepackage[dvipsnames,svgnames,x11names,table]{xcolor}


\RequirePackage{amssymb} % Must be loaded before unicode-math
\RequirePackage{unicode-math} % Math fonts in xetexorluatex
%\setmathfont{texgyrepagella-math.otf}
\setmainfont{STIXTwoText}[
	Path=./fonts/STIXTwoText/,
  Extension = .otf,
  UprightFont = STIXTwoMath-Regular,
  BoldFont = STIXTwoText-SemiBold,
  ItalicFont = STIXTwoText-Italic,
  BoldItalicFont = STIXTwoText-BoldItalic,
]
\setmathfont{STIXTwoMath-Regular.otf}

\usepackage{tikz,calc}
\usetikzlibrary{calc,arrows,shadows.blur,tikzmark}
% By default all math in TikZ nodes are set in inline mode. Change this to
% displaystyle so that we don't get small fractions.
\everymath{\displaystyle}

\usepackage[most]{tcolorbox}
\tcbuselibrary{breakable, skins,theorems}


\usepackage{xkeyval}
\makeatletter 
\usepackage{amsmath,amssymb,amsthm}
\usepackage{xkeyval} 
\usepackage{zhlipsum}
\tcbset{
    common/.style={
      fontupper=\rmfamily,
      lower separated=true,
      coltitle=black,
      colback=gray!5,
      boxrule=0.5pt,
      fonttitle=\bfseries,
      enhanced,
      breakable,
      top=8pt,
      before skip=8pt,
      attach boxed title to top left={yshift=-0.05in,xshift=0.15in},
      boxed title style={
      	boxrule=0pt,
        colframe=black,
        arc=0pt,
        outer arc=0pt,
        drop fuzzy shadow
     	},
      separator sign={.},
      drop fuzzy shadow
     },
     litistyle/.style={  %例题风格
     	common,
      colframe=gray,% 
      colback=white,
      colbacktitle=Gold,
      sharp corners,rounded corners=southeast,
      overlay unbroken and last={
      \node[anchor=south east, outer sep=0pt] at (\linewidth-width,0){\textcolor{black!60}{$\blacksquare$}};}
         },
% -----------------------------------------------------------------------------
% Explanation: BreakableBox@title key
% The following defines a pgfkeys/tcolorbox key named
%   "BreakableBox@title" which accepts 2 arguments (see 
%   "/.code n args={2}"). When the key is used like
%   "BreakableBox@title={<env>}{<optTitle>}" the code below is executed
%   with #1=<env> and #2=<optTitle>.
%
% Purpose:
% - If the second argument (#2) is empty, set the tcolorbox title to
%     \csname <env>name\endcsname~\thetcbcounter
%   which expands to the environment name (e.g. "liti" -> \litiname)
%   followed by the current counter value.
% - If the second argument is provided, append " (#2)" to the title.
%
% Notes:
% - This relies on \makeatletter being active so that keys with '@'
%   are allowed in control sequence names.
% - \ifblank is used to test whether #2 is empty; ensure the package
%   that provides \ifblank (for example, etoolbox or xparse) is loaded
%   earlier if necessary.
% - Example usage inside the code: \tcbset{title={...}} sets the
%   tcolorbox title key accordingly.
%
         BreakableBox@title/.code n args={2}
              {
                \ifblank{#2}
                  {\tcbset{title={\csname #1name\endcsname~\thetcbcounter}}}
                  {\tcbset{title={\csname #1name\endcsname~\thetcbcounter\ (#2)}}}
              },
}      
  % define an internal control sequence \newbreakablebox for fancy mode's newtheorem
  % #1 is the environment name, #2 is the prefix of label, #3 is the style
  % style: thmstyle, defstyle, prostyle
  % e.g. \newbreakablebox{theorem}{thm}{thmstyle}
  % will define two environments: numbered ``theorem'' and no-numbered ``theorem*''
  % WARNING FOR MULTILINGUAL: this cs will automatically find \theoremname's definition,
  % WARNING FOR MULTILINGUAL: it should be defined in language settings.
  \newcommand{\newbreakablebox}[3]{
    \ifcsundef{#1name}{%
      % define a default name if not defined before
      \tcbset{BreakableBox@title/.code n args={2}{\tcbset{title={use newcommand define #1name}}} }
    }{\relax}
    \DeclareTColorBox[auto counter,number within=section]{#1}{ g o t\label g }{
        common, % use common style
        #3, % use the style passed in
        IfValueTF={##1}
          {BreakableBox@title={#1}{##1}}
          {
            IfValueTF={##2}
            {BreakableBox@title={#1}{##2}}
            {BreakableBox@title={#1}{}}
          },
        IfValueT={##4}
          {
            IfBooleanTF={##3}
              {label={##4}}
              {VIVID@label={#2}{##4}}
          }
      }
    \DeclareTColorBox{#1*}{ g o }{
        common,#3,
        IfValueTF={##1}
          {BreakableBox@title={#1}{##1}}
          {
            IfValueTF={##2}
            {BreakableBox@title={#1}{##2}}
            {BreakableBox@title={#1}{}}
          },
      }
  }
  % define several environment 
  % we define headers like \definitionname before
  \newcommand{\litiname}{例题}
  \newbreakablebox{liti}{def}{litistyle}
  
  
%补充内容
%%设置新字体
%%定义带圈数字命令
\newfontfamily{\nmfont}{circlenumber}
[%
Extension=.otf,
Path=./fonts/]

\newcommand{\quan}[1]{{\nmfont \symbol{#1}}}
\newcommand{\kk}[1]{\quan{\numexpr32+#1}}%\kk{<参数范围1-95>}96、97、98、99分别用\quan{196} \quan{197} \quan{199} \quan{201}

%脚注使用带圈数字
\newcommand*\kkctr[1]{%
  \protect\kk{\number\numexpr\value{#1}\relax}}
\renewcommand*\thefootnote{\textcolor{black}{\kkctr{footnote}}}

%%无悬挂脚注格式
\renewcommand\@makefntext[1]{%
  \setlength\parindent{2\ccwd}\selectfont
  \@thefnmark\ #1}

%修改\part,使其不分页
\def\@endpart{%
	\thispagestyle{empty}
  \vskip40\p@%
   \@afterheading}



\RequirePackage{enumitem}
%\newenvironment{myenum}{\begin{enumerate}[label=\protect\kk{\arabic*}]\small}{\end{enumerate}}%
\setlist{noitemsep}
\setlist[enumerate, 1]{label=\protect\kk{\arabic*},itemsep=0.5ex}  
\makeatother



%%%marker环境
\newtcolorbox{marker}[1][]{enhanced,before skip=2mm,
	after skip=3mm,fontupper=\rmfamily,
	boxrule=0.4pt,left=5mm,right=2mm,top=1mm,bottom=1mm,
	colback=yellow!50,colframe=yellow!20!black,
	sharp corners,rounded corners=southeast,
	arc is angular,arc=3mm,underlay={%
		\path[fill=tcbcolback!80!black] ([yshift=3mm]interior.south east)--++(-0.4,-0.1)--++(0.1,-0.2);
		\path[draw=tcbcolframe,shorten <=-0.05mm,shorten >=-0.05mm] ([yshift=3mm]interior.south east)--++(-0.4,-0.1)--++(0.1,-0.2);
		\path[fill=yellow!50!black,draw=none] (interior.south west) rectangle node[white]{\Huge\bfseries !} ([xshift=4mm]interior.north west);
	},
	drop fuzzy shadow,#1
}

\usepackage{exam-zh-choices}
\usepackage{fontawesome5}
\begin{document}
\section{灵宝一高数学竞赛}
%\begin{liti}
%(2024河南月考)f(x)=x|x|,满足f(2x-1)+f(x)$\leq$0
%\end{liti}
%\begin{liti}
%(2024甘肃月考)$函数y=f(x)是定义在[-2,2]上的单调减函数,且f(a+1)<f(2a),实数a的范围是$
%
%\end{liti}

%\begin{liti}{第16题}
%已知函数$f(x)=ax^2-(a+2)x+b$.
%\begin{enumerate}
%\item 若$f(x)\leq 0$的解集为$\{x|1\leq x \leq 2\}$,求$a,b$的值;
%\item 若$b=2$,求不等式$f(x)\leq 0$的解集;
%\item 在\kk{1}的条件下,若对任意$x>1$,不等式$\dfrac{f(x)+1}{ax-1}\leq 2k^2+k$恒成立,求实数$k$的取值范围。
%\end{enumerate}
%
%\tcblower\small
%\textbf{解:}
%\begin{enumerate}
%\item \textbf{求a,b的值}:因为$f(x) \leq 0$的解集为$\{x|1 \leq x \leq 2\}$,所以二次方程$ax^2 - (a+2)x + b = 0$的两根为1和2,且抛物线开口向上(即a > 0)。
%
%根据韦达定理:
%		\begin{itemize}
%				\item 根的和:$1 + 2 = \dfrac{a+2}{a}$,解得$3a = a + 2$,即$a = 1$;
%				\item 根的积:$1 \times 2 = \dfrac{b}{a}$,代入$a = 1$,得$b = 2$。
%		\end{itemize}
%\item 当$b=2$时,求不等式$f(x) \leq 0$的解集
%
%此时$f(x) = ax^2 - (a+2)x + 2 = (ax - 2)(x - 1)$,分情况讨论a的取值:
%\begin{itemize}
%\item 当a = 0时:$f(x) = -2x + 2$,解$-2x + 2 \leq 0$,得$x \geq 1$,解集为$\{x|x \geq 1\}$。
%
%\item 当a > 0时:方程$(ax - 2)(x - 1) = 0$的根为$x = \dfrac{2}{a}$和$x = 1$。
%	\begin{itemize}
%	\item 若$\dfrac{2}{a} = 1(即a = 2):f(x) = 2(x - 1)^2 \leq 0$,仅当$x = 1$时成立,解集为$\{x|x = 1\}$。
%
%	\item 若$\dfrac{2}{a} > 1(即0 < a < 2)$:抛物线开口向上,解集为$\{x|1 \leq x \leq \dfrac{2}{a}\}$。
%
%	\item 若$\dfrac{2}{a} < 1(即a > 2)$:抛物线开口向上,解集为$\{x|\dfrac{2}{a} \leq x \leq 1\}$。
%	\end{itemize}
%\item  当$a < 0$时:$\dfrac{2}{a} < 0 < 1$,抛物线开口向下,解集为$\{x|x \leq \dfrac{2}{a}或x \geq 1\}$。
%\end{itemize}
%\item 求实数k的取值范围
%
%由\kk{1}知$a = 1,b = 2,故f(x) = x^2 - 3x + 2$,不等式变为:$\dfrac{x^2 - 3x + 2 + 1}{x - 1} \leq 2k^2 + k \quad (x > 1)$
%
%步骤1:\textbf{化简}:
%令$t = x - 1(t > 0,因为x > 1),则x = t + 1$,代入分子:$(t + 1)^2 - 3(t + 1) + 3 = t^2 - t + 1$
%
%因此左边变为:$\dfrac{t^2 - t + 1}{t} = t + \dfrac{1}{t} - 1$
%
%步骤2:\textbf{求左边的最小值}
%
%由基本不等式:当$t > 0$时,$t + \dfrac{1}{t} \geq 2\sqrt{t \cdot \dfrac{1}{t}} = 2$(当且仅当t = 1,即x = 2时取等号)。
%因此$t + \dfrac{1}{t} - 1 \geq 2 - 1 = 1$,即左边的最小值为1。
%
%步骤3:\textbf{解关于k的不等式}
%
%要使不等式对任意x > 1恒成立,需$2k^2 + k \geq 1$,即:$2k^2 + k - 1 \geq 0$,即:$(2k - 1)(k + 1) \geq 0$。
%
%解得$k \leq -1$或$k \geq \dfrac{1}{2}$。
%\end{enumerate}
%最终答案:
%\begin{enumerate}
%\item a = 1,b = 2;
%\item 分情况讨论(见上述过程);
%\item k的取值范围为$(-\infty, -1] \cup [\dfrac{1}{2}, +\infty)$。
%\end{enumerate}
%\end{liti}
\section*{一、单选题(本题共8小题,每题5分,共40分。在每小题给出的选项中,仅一项符合题目要求)}
\begin{enumerate}
    \item 下列函数中最小值为4的是( )
    \begin{align*}
        A. &\ y=4x+\frac{1}{x} \\
        C. &\ \text{当 } x<\frac{3}{2} \text{ 时,} y=2x-1+\frac{1}{2x-3} \\
        D. &\ y=\sqrt{x^2+5}+\frac{4}{\sqrt{x^2+5}}
    \end{align*}

    \item 购买同一种物品,可采用两种策略:第一种不考虑价格升降,每次购买数量固定;第二种不考虑价格升降,每次购买花费固定。假设连续两天购买,第一天价格为 \(p_1\),第二天价格为 \(p_2\)(\(p_1 \neq p_2\)),则下列选项正确的是( )
    \begin{align*}
        A. &\ \text{第一种方式购买单价为 } \frac{p_1p_2}{p_1+p_2} \\
        B. &\ \text{第一种方式购买单价为 } \frac{p_1+p_2}{2} \\
        C. &\ \text{第一种方式购买单价更低} \\
        D. &\ \text{第二种方式购买单价更低}
    \end{align*}

    \item 函数 \(y=[x]\) 称为高斯函数(取整函数),表示不大于 \(x\) 的最大整数(如 \([1.5]=1,[-2.3]=-3,[3]=3\))。则不等式 \(4[x]^2 - 12[x] + 5 \leq 0\) 成立的充分不必要条件是( )
    \begin{align*}
        A. &\ \left[\frac{1}{2}, \frac{5}{2}\right] \\
        B. &\ [1,2] \\
        C. &\ [1,3) \\
        D. &\ [1,3]
    \end{align*}

    \item 已知函数 \(f(x)=-x^2+4x+1\) 在区间 \([0,m]\) 上的值域为 \([1,5]\),则 \(m\) 的取值范围是( )
    \begin{align*}
        A. &\ (0,2] \\
        B. &\ (0,4] \\
        C. &\ [2,4]
    \end{align*}

    \item 若函数 \(y=f(x)\) 的大致图象如图所示(图略),则 \(f(x)\) 的解析式可能是( )
    \begin{align*}
        A. &\ f(x)=\frac{x}{|x|-1} \\
        B. &\ f(x)=\frac{x}{1-|x|} \\
        C. &\ f(x)=\frac{x}{x^2-1} \\
        D. &\ f(x)=\frac{x}{1-x^2}
    \end{align*}

    \item 已知函数 \(f(x)=\begin{cases}
        x+\frac{1}{2}, & x \in [0, \frac{1}{2}) \\
        3x^2, & x \in [\frac{1}{2}, 1]
    \end{cases}\),若存在 \(x_1<x_2\) 使得 \(f(x_1)=f(x_2)\),则 \(x_1 \cdot f(x_2)\) 的取值范围为( )
    \begin{align*}
        A. &\ \left[\frac{3}{4}, 1\right) \\
        B. &\ \left[\frac{1}{8}, \frac{\sqrt{3}}{6}\right) \\
        D. &\ \left[\frac{3}{8}, 3\right) \\
        C. &\ \left[\frac{3}{16}, \frac{1}{2}\right)
    \end{align*}

    \item 已知函数 \(f(x)\) 的定义域为 \(\mathbb{R}\),对任意 \(x_1 \neq x_2\) 都有 \([f(x_1)-f(x_2)](x_1-x_2)>0\)。若 \(f(x^2-3x+a)>f(x-2a^2-6a)\) 对任意 \(x \in \mathbb{R}\) 恒成立,则实数 \(a\) 的取值范围是( )
    \begin{align*}
        A. &\ \left(-\infty, -\frac{1}{2}\right) \cup (4, +\infty) \\
        B. &\ \left(-\frac{1}{4}, \frac{1}{2}\right) \\
        D. &\ \left[-\frac{1}{2}, 4\right]
    \end{align*}

    \item 定义 \(min\ M\) 表示数集 \(M\) 中的最小值,已知不全为0的实数 \(x,y\),二元函数 \(f(x,y)=min\left\{x, \frac{x}{x^2+y^2}\right\}\),则 \(f(x,y)\) 的最大值为( )
    \begin{align*}
        A. &\ 0 \\
        C. &\ 1 \\
        D. &\ 2
    \end{align*}
\end{enumerate}

\section*{二、多项选择题(本题共3小题,每题6分,共18分。全部选对得6分,部分选对得部分分,有选错得0分)}
\begin{enumerate}
    \item[9.] 下列命题为假命题的是( )
    \begin{align*}
        A. &\ \text{命题“}\exists x_0>0, x_0^2-5x_0+6>0\text{”的否定是“}\forall x \leq 0, x^2-5x+6 \leq 0\text{”} \\
        B. &\ \text{若函数 } f(x+1) \text{ 的定义域为 } [1,4] \text{,则函数 } f(x) \text{ 的定义域为 } [2,5] \\
        C. &\ \text{二次函数 } y=x^2-x-6 \text{ 的零点为 } (-2,0) \text{ 和 } (3,0) \\
        D. &\ \text{“}a^2=b^2\text{”是“}a=b\text{”的必要不充分条件}
    \end{align*}

    \item[10.] 已知 \(a,b\) 为正实数,且 \(ab+2a+b=16\),则( )
    \begin{align*}
        A. &\ 2a+b \text{ 的最小值为8} \\
        B. &\ \frac{1}{a+1}+\frac{1}{b+2} \text{ 的最小值为 } \frac{\sqrt{2}}{2} \\
        C. &\ ab \text{ 的最大值为8} \\
        D. &\ b+\frac{1}{9-a} \text{ 的最小值为 } \frac{6\sqrt{2}-1}{10}
    \end{align*}

    \item[11.] 给出下列四个命题,其中真命题的是( )
    \begin{align*}
        A. &\ \text{已知函数 } f(x)=x^2-(2a+1)x+5 \text{,若对任意 } x_1,x_2 \in (4,+\infty) \text{,当 } x_1>x_2 \text{ 时} \\
        &\ \text{总有 } f(x_1)-f(x_2)>x_2-x_1 \text{,则实数 } a \text{ 的取值范围是 } (-\infty,4] \\
        B. &\ \text{存在不同的实数 } k \text{,使得关于 } x \text{ 的方程 } (x^2-1)^2-|x^2-1|+k=0 \text{ 分别恰有2个和4个不同实根} \\
        C. &\ \text{存在不同的实数 } k \text{,使得关于 } x \text{ 的方程 } (x^2-1)^2-|x^2-1|+k=0 \text{ 分别恰有5个和8个不同实根} \\
        D. &\ \text{存在不同的实数 } k \text{,使得关于 } x \text{ 的方程 } (x^2-1)^2-|x^2-1|+k=0 \text{ 分别恰有0个和3个不同实根}
    \end{align*}
\end{enumerate}

\section*{三、填空题(本题共3小题,每题5分,共15分)}
\begin{enumerate}
    \item[12.] 不等式 \(\frac{2x-1}{x+2} \leq 1\) 的解集为 \(\underline{\qquad\qquad}\)。

    \item[13.] 下列几个命题:
    \begin{enumerate}
        \item ① 函数 \(f(x)=\frac{\sqrt{2-x}}{x}\) 的定义域为 \((-\infty,0) \cup (0,2]\);
        \item ② 函数 \(f(x)=2|x-2|+5|x+1|\) 的最小值是6;
        \item ③ 函数 \(y=x+\sqrt{x-1}\) 的值域为 \([0,+\infty)\);
        \item ④ 已知 \(f(x)+2f(-x)=3x+x^2\),则 \(f(1)=-\frac{8}{3}\);
        \item ⑤ 命题 \(p: y \geq 3\) 是命题 \(q: y\) 的最小值是3的充要条件;
        \item ⑥ 若函数 \(f(x)=\begin{cases}
            (2+3a)x-1, & x>1 \\
            4ax-x^2, & x \leq 1
        \end{cases}\) 在 \(\mathbb{R}\) 上单调递增,则实数 \(a\) 的取值范围是 \([\frac{1}{2},2]\)。
    \end{enumerate}
    其中正确命题的序号分别是 \(\underline{\qquad\qquad}\)。

    \item[14.] 已知 \(a,b\) 均为正数,且 \(a+b=1\),\(c>1\),则 \(\left(\frac{a^2+1}{2ab}-1\right)c+\frac{\sqrt{2}}{c-1}\) 的最小值为 \(\underline{\qquad\qquad}\)。
\end{enumerate}

\section*{四、解答题(本题共5小题,共77分。解答应写出文字说明、证明过程或演算步骤)}
\begin{enumerate}
    \item[15.] 已知命题 \(p\):实数 \(x\) 满足 \(x^2-10x+16 \leq 0\),命题 \(q\):实数 \(x\) 满足 \(x^2-4mx+3m^2 \leq 0\)(其中 \(m>0\))。
    \begin{enumerate}
        \item[(1)] 若 \(m=1\),且命题 \(p\) 和 \(q\) 中至少有一个为真命题,求实数 \(x\) 的取值范围;
        \item[(2)] 若 \(q\) 是 \(p\) 的充分条件,求实数 \(m\) 的取值范围。
    \end{enumerate}

    \item[16.] 已知函数 \(f(x)=ax^2-(a+2)x+b\)。
    \begin{enumerate}
        \item[(1)] 若 \(f(x) \leq 0\) 的解集为 \(\{x \mid 1 \leq x \leq 2\}\),求 \(a,b\) 的值;
        \item[(2)] 若 \(b=2\),求不等式 \(f(x) \leq 0\) 的解集;
        \item[(3)] 在(1)的条件下,若对任意 \(x>1\),不等式 \(\frac{f(x)+1}{ax-1} \geq 2k^2+k\) 恒成立,求实数 \(k\) 的取值范围。
    \end{enumerate}

    \item[17.] 某学校要建造一个长方体形体育馆,其地面面积为 \(240\ \text{m}^2\),体育馆高 \(5\ \text{m}\)。甲工程队报价为:馆顶每平方米造价100元,前后两侧墙壁平均造价每平方米150元,左右两侧墙壁平均造价每平方米250元,设体育馆前墙长为 \(x\) 米。
    \begin{enumerate}
        \item[(1)] 当 \(x \in (0,50)\) 时,体育馆前墙长度为多少时,甲工程队报价最低?
        \item[(2)] 当 \(x \in (0,t]\)(\(0<<t<50\))时,乙工程队给出的整体报价为 \(12000+500\left(\frac{a+1152}{x}+a\right)\) 元(\(a>0\))。若无论体育馆前墙长度 \(x\) 为多少米,乙工程队都能中标,试求 \(a\) 的取值范围。
    \end{enumerate}

    \item[18.] 问题:已知 \(a,b,c\) 均为正实数,且 \(\frac{1}{a}+\frac{1}{b}+\frac{1}{c}=1\),求证:\(a+b+c \geq 9\)(当且仅当 \(a=b=c=3\) 时,等号成立)。\\
    证明:\(a+b+c=(a+b+c)\left(\frac{1}{a}+\frac{1}{b}+\frac{1}{c}\right)=3+\left(\frac{b}{a}+\frac{a}{b}\right)+\left(\frac{c}{a}+\frac{a}{c}\right)+\left(\frac{c}{b}+\frac{b}{c}\right) \geq 3+2+2+2=9\)。\\
    学习上述解法并解决下列问题:
    \begin{enumerate}
        \item[(1)] 已知 \(a,b,c\) 均为正实数,且 \(a+b+c=4\),求 \(\frac{1}{a}+\frac{4}{b}+\frac{9}{c}\) 的最小值;
        \item[(2)] 已知 \(a,b,x,y\) 均为正实数,且 \(\frac{x^2}{a^2}+\frac{y^2}{b^2}=1\),求证:\(a^2+b^2 \geq (x+y)^2\);
        \item[(3)] 求 \(T=\sqrt{3-2t}+\sqrt{t-1}\) 的最大值,并求出使得 \(T\) 取得最大值时 \(t\) 的值。
    \end{enumerate}

    \item[19.] 已知函数 \(f(x)=|3x^2-ax|+x^2-x+1\)(\(x>0\)),\(g(x)=\frac{f(x)}{x}\)。
    \begin{enumerate}
        \item[(1)] 若 \(a=1\),求函数 \(f(x)\) 的值域;
        \item[(2)] 若 \(a \leq 0\),试判断 \(g(x)\) 的单调性并证明;
        \item[(3)] 对 \(\forall t \in [3,4]\),\(\exists x_1,x_2 \in [\frac{1}{8},2]\)(\(x_1 \neq x_2\)),使得 \(t=g(x_1)=g(x_2)\),求实数 \(a\) 的取值范围。
    \end{enumerate}
\end{enumerate}
\end{document}
