\documentclass[CJKmath,a4paper,10pt]{ctexart}
\usepackage{geometry}
\geometry{inner=1.5cm,outer=1.5cm,top=2.45cm,bottom=1cm}
\usepackage[dvipsnames,svgnames,x11names,table]{xcolor}

\usepackage{tikz,calc}
\usetikzlibrary{calc,shadows.blur}
\usepackage[most]{tcolorbox}
\tcbuselibrary{breakable, skins,theorems}


\usepackage{xkeyval}
\makeatletter 
\usepackage{amsmath,amssymb,amsthm}
\usepackage{xkeyval} 
\usepackage{zhlipsum}
\tcbset{
    common/.style={
      fontupper=\rmfamily,
      lower separated=true,
      coltitle=black,
      colback=gray!5,
      boxrule=0.5pt,
      fonttitle=\bfseries,
      enhanced,
      breakable,
      top=8pt,
      before skip=8pt,
      attach boxed title to top left={yshift=-0.11in,xshift=0.15in},
      boxed title style={
      	boxrule=0pt,
        colframe=white,
        arc=0pt,
        outer arc=0pt
     	},
      separator sign={.},
      drop fuzzy shadow
     },
     litistyle/.style={  %例题风格
     	common,
      colframe=gray,%  
      colback=gray!5,
      colbacktitle=yellow, 
      overlay unbroken and last={
      \node[anchor=south east, outer sep=0pt] at (\linewidth-width,0){\textcolor{black!60}{$\blacksquare$}};}
		 },
		 BreakableBox@title/.code n args={2}
		      {
		        \ifblank{#2}
		          {\tcbset{title={\csname #1name\endcsname~\thetcbcounter}}}
		          {\tcbset{title={\csname #1name\endcsname~\thetcbcounter\ (#2)}}}
		      },
}      
  % define an internal control sequence \newbreakablebox for fancy mode's newtheorem
  % #1 is the environment name, #2 is the prefix of label, #3 is the style
  % style: thmstyle, defstyle, prostyle
  % e.g. \newbreakablebox{theorem}{thm}{thmstyle}
  % will define two environments: numbered ``theorem'' and no-numbered ``theorem*''
  % WARNING FOR MULTILINGUAL: this cs will automatically find \theoremname's definition,
  % WARNING FOR MULTILINGUAL: it should be defined in language settings.
  \newcommand{\newbreakablebox}[3]{
    \ifcsundef{#1name}{%
      % define a default name if not defined before
      \tcbset{BreakableBox@title/.code n args={2}{\tcbset{title={use newcommand define #1name}}} }
    }{\relax}
    \DeclareTColorBox[auto counter,number within=section]{#1}{ g o t\label g }{
        common,#3,
        IfValueTF={##1}
          {BreakableBox@title={#1}{##1}}
          {
            IfValueTF={##2}
            {BreakableBox@title={#1}{##2}}
            {BreakableBox@title={#1}{}}
          },
        IfValueT={##4}
          {
            IfBooleanTF={##3}
              {label={##4}}
              {VIVID@label={#2}{##4}}
          }
      }
    \DeclareTColorBox{#1*}{ g o }{
        common,#3,
        IfValueTF={##1}
          {BreakableBox@title={#1}{##1}}
          {
            IfValueTF={##2}
            {BreakableBox@title={#1}{##2}}
            {BreakableBox@title={#1}{}}
          },
      }
  }
  % define several environment 
  % we define headers like \definitionname before
  \newcommand{\litiname}{例题}
  \newbreakablebox{liti}{def}{litistyle}
\makeatother



%%%marker环境
\newtcolorbox{marker}[1][]{enhanced,before skip=2mm,
	after skip=3mm,fontupper=\rmfamily,
	boxrule=0.4pt,left=5mm,right=2mm,top=1mm,bottom=1mm,
	colback=yellow!50,colframe=yellow!20!black,
	sharp corners,rounded corners=southeast,
	arc is angular,arc=3mm,underlay={%
		\path[fill=tcbcolback!80!black] ([yshift=3mm]interior.south east)--++(-0.4,-0.1)--++(0.1,-0.2);
		\path[draw=tcbcolframe,shorten <=-0.05mm,shorten >=-0.05mm] ([yshift=3mm]interior.south east)--++(-0.4,-0.1)--++(0.1,-0.2);
		\path[fill=yellow!50!black,draw=none] (interior.south west) rectangle node[white]{\Huge\bfseries !} ([xshift=4mm]interior.north west);
	},
	drop fuzzy shadow,#1
}

\usepackage{exam-zh-choices}
\begin{document}
\section{数学作业}




\begin{liti}
若正实数$a,b$满足$a+b=1$,则$ax^2+\big(3+\dfrac{1}{b}\big)x-a=0$的根的最大值为?
\tcblower\kaishu
设方程两根分别为$\alpha,\beta$,其中$\alpha\le\beta$。
由韦达定理,$\alpha\times\beta=\dfrac{-a}{a}=-1$。
因为$a$是正实数,所以方程必有一正根和一负根。因此,较大根 $\beta > 0$。

将$\beta$代入原方程,我们得到:$a\beta^2 + \left(3+\dfrac{1}{b}\right)\beta - a = 0$

整理后可得:$a(1-\beta^2) = \left(3+\dfrac{1}{b}\right)\beta$

因为 $a>0, b>0, \beta>0$,所以 $\left(3+\dfrac{1}{b}\right)\beta > 0$。
因此,$a(1-\beta^2) > 0$,可推得 $1-\beta^2 > 0$,即 $0 < \beta < 1$。

又因为 $a,b$为正实数且 $a+b=1$,所以 $0<a<1$ 且 $0<b<1$。

将 $a=1-b$ 代入整理后的方程:
$(1-b)(1-\beta^2) = \left(3+\dfrac{1}{b}\right)\beta = \dfrac{3b+1}{b}\beta$

化简后得:$b-b^2-b\beta^2+b^2\beta^2 = 3b\beta+\beta$

将上式整理为关于$b$的二次方程:
$(\beta^2-1)b^2 + (1-3\beta-\beta^2)b - \beta = 0$

再两边同时$\times -1$得:$(1-\beta^2)b^2 + (\beta^2+3\beta-1)b + \beta = 0$

因为$b$是实数,所以该二次方程的判别式 $\Delta \ge 0$。

$\Delta = (\beta^2+3\beta-1)^2 - 4(1-\beta^2)\beta \ge 0$

$\beta^4+9\beta^2+1+6\beta^3-2\beta^2-6\beta - 4\beta+4\beta^3 \ge 0$

$\beta^4+10\beta^3+7\beta^2-10\beta+1 \ge 0$

由于 $\beta \ne 0$,不等式两边同除以 $\beta^2$:$\beta^2+10\beta+7-\dfrac{10}{\beta}+\dfrac{1}{\beta^2} \ge 0$

$\left(\beta^2+\dfrac{1}{\beta^2}\right) + 10\left(\beta-\dfrac{1}{\beta}\right) + 7 \ge 0$

令 $y=\beta-\dfrac{1}{\beta}$,则 $\beta^2+\dfrac{1}{\beta^2} = y^2+2$。

代入不等式:$y^2+2 + 10y + 7 \ge 0$

$y^2+10y+9 \ge 0$

$(y+1)(y+9) \ge 0$

解得 $y \ge -1$ 或 $y \le -9$。

即 $\beta-\dfrac{1}{\beta} \ge -1$ 或 $\beta-\dfrac{1}{\beta} \le -9$。
因为 $0 < \beta < 1$,所以 $\beta-\dfrac{1}{\beta} < 0$。

\begin{itemize}
    \item 对于 $\beta-\dfrac{1}{\beta} \ge -1$:\\
    因为 $\beta>0$,不等式两边同乘 $\beta$ 得 $\beta^2-1 \ge -\beta$,即 $\beta^2+\beta-1 \ge 0$。\\
    解得 $\beta \ge \dfrac{-1+\sqrt{5}}{2}$ 或 $\beta \le \dfrac{-1-\sqrt{5}}{2}$。\\
    此解与 $0 < \beta < 1$ 的区间没有交集,故此情况不成立。
    \item 对于 $\beta-\dfrac{1}{\beta} \le -9$:\\
    因为 $\beta>0$,不等式两边同乘 $\beta$ 得 $\beta^2-1 \le -9\beta$,即 $\beta^2+9\beta-1 \le 0$。\\
    解得 $\dfrac{-9-\sqrt{85}}{2} \le \beta \le \dfrac{-9+\sqrt{85}}{2}$。
\end{itemize}

综合 $0 < \beta < 1$ 和上述解集,我们得到 $\beta$ 的取值范围是 $0 < \beta \le \dfrac{-9+\sqrt{85}}{2}$。

因此,根 $\beta$ 的最大值为 $\dfrac{-9+\sqrt{85}}{2}$。
\newpage
\textbf{\rmfamily 解法二:}

设方程 $ax^2+\left(3+\dfrac{1}{b}\right)x-a=0$的两根为$x_1,x_2$。

由韦达定理得: $x_1x_2 = \dfrac{-a}{a}=-1$。这表明方程必有一正根和一负根。我们要求的是根的最大值,即正根的最大值。

设正根为 $x(x>0)$,则另一根为 $-\dfrac{1}{x}$。 

$两根之和为 x_1+x_2 = x - \dfrac{1}{x} = -\dfrac{3+\dfrac{1}{b}}{a} = -\dfrac{3b+1}{ab}$。
\vskip1.2ex

利用“1的代换”得:$x - \dfrac{1}{x} = -\dfrac{3b+1}{ab} = -\dfrac{3b+a+b}{ab} = -\left(\dfrac{4b+a}{ab}\right) = -\left(\dfrac{4}{a}+\dfrac{1}{b}\right)$

我们要求 $x$ 的最大值:

$\because x>0,f(x)=x-\dfrac{1}{x}$ 是一个增函数。

$\therefore$要使 $x$ 最大,就需要使 $x-\dfrac{1}{x} 最大$。

这等价于求 $-\left(\dfrac{4}{a}+\dfrac{1}{b}\right)$ 的最大值,也就是求 $\left(\dfrac{4}{a}+\dfrac{1}{b}\right)$ 的最小值。

\textbf{\rmfamily 权方和}:$\dfrac{4}{a}+\dfrac{1}{b}  \ge \dfrac{(2+1)^2}{a+b} = 9$。当且仅当 $\dfrac{2}{a}=\dfrac{1}{b}$,即 $a=2b$ 时取等。

又$\because a+b=1 \therefore 2b+b=1 \implies b=\dfrac{1}{3}, a=\dfrac{2}{3}。$

$\therefore \big(x-\dfrac{1}{x}\big)_{max} = -9$。即 $x^2+9x-1=0。$

解得 $x=\dfrac{-9\pm\sqrt{81+4}}{2}$。

又$\because x>0$,$\therefore x=\dfrac{-9+\sqrt{85}}{2}$。

\end{liti}


\end{document}
