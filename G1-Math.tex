\documentclass[CJKmath,a4paper,10pt]{ctexart}
\usepackage{geometry}
\geometry{inner=1.5cm,outer=1.5cm,top=2.45cm,bottom=1cm}
\RequirePackage[dvipsnames,svgnames,x11names,table]{xcolor}

\usepackage{xeboiboites}
% /* --------------------------------- 模板默认颜色 --------------------------------- */
\colorlet{outermarginfgcolor}{DarkCyan} % foregroundcolor 较深的前景色
\colorlet{outermarginbgcolor}{DarkCyan!30} % backgroundcolor 较浅的背景色
% /* -------------------------------------------------------------------------- */

\RequirePackage{tikz,calc} %%页面样式设计核心包 %提供\pgfonlayer命令
\usetikzlibrary{cd,calc,shadows,hobby,intersections, decorations.markings, decorations.pathreplacing,spy,arrows,shapes,fadings,trees,mindmap,patterns,shapes.arrows,shapes.symbols,tikzmark,shapes.geometric,graphs, quotes, angles,decorations.pathmorphing,through,shadings,backgrounds,positioning,fit,arrows.meta,shapes.misc,decorations.shapes}

% 绘图:函数图像
\usepackage{pgfplots}
\pgfplotsset{compat=1.18}

\RequirePackage[most]{tcolorbox}
\tcbuselibrary{breakable, skins,theorems}


\usepackage{amsmath,amssymb}
\usepackage{xparse}
\usepackage{etoolbox}
\RequirePackage[explicit]{titlesec}
\usepackage{varwidth}
% /* -------------------------------- Section样式 ------------------------------- */
%\titleformat{\section}
%{\thispagestyle{empty}}
%{}
%{-.5em} %左右移动\thesection标签位置
%{\mysectionformat{#1}}
%
%\titleformat{name=\section,numberless}{}{}{-.5em}{\mysectionnonumformat{#1}}
%
%\newcommand{\mysectionformat}[1]{%
%\makebox[0pt][l]{\def\rad{7pt}%
%\begin{tikzpicture}[remember picture]
%%  \path[fill=outermarginfgcolor,drop shadow={opacity=0.3,shadow xshift=.05cm,shadow yshift=-.05cm}]node[append after command={
%%  (sec.north west) -- (sec.south west) -- ([xshift=-\rad]sec.south east) to[out=0,in=180,looseness=1] ([xshift=3*\rad]sec.north east) --cycle},
%%  text=white,font=\sffamily\large\bfseries,align=center,inner ysep=2mm] (sec) at (0,0) {\thesection};
%	\path[fill=outermarginfgcolor,drop shadow={opacity=0.2,shadow xshift=.05cm,shadow yshift=-.05cm}]node[append after command={
%  (sec.north west) -- (sec.south west) -- ([xshift=-\rad]sec.south east) to[out=0,in=180,looseness=0] ([xshift=1*\rad]sec.north east) --cycle},
%  text=white,font=\sffamily\large\bfseries,align=center,inner ysep=2mm] (sec) at (0,0) {\thesection};
%  % 标题
%  \path[fill=outermarginfgcolor,drop shadow={opacity=0.2,shadow xshift=.05cm,shadow yshift=-.05cm}]node[append after command={
%      ([xshift=1*\rad]secnum.north west) to[out=180,in=0,looseness=0] ([xshift=-1*\rad]secnum.south west) --(secnum.south east) to (secnum.north east) --cycle},text=black,font=\sffamily\large\bfseries,below right](secnum) at ([shift={(.5*\rad,-0.5mm)}]sec.north east) {\begin{varwidth}{.85\linewidth}\setlength\baselineskip{18pt}\hspace{.5cm}#1\end{varwidth}};
%\end{tikzpicture}}}%最后一个选项为 [<after code>]
%
%\newcommand{\mysectionnonumformat}[1]{%
%\makebox[0pt][l]{\def\rad{7pt}%
%\begin{tikzpicture}[remember picture]
%  \path[fill=outermarginfgcolor,drop shadow={opacity=0.3,shadow xshift=.05cm,shadow yshift=-.05cm}]node[append after command={
%  % 去掉左侧曲线,改为直线;右侧保持原先弧线造型
%  (sec.north west) -- (sec.south west) -- ([xshift=-\rad]sec.south east) to[out=0,in=180,looseness=1] ([xshift=3*\rad]sec.north east) --cycle},
%  text=white,font=\sffamily\large\bfseries,align=center,inner ysep=2mm] (sec) at (0,0) {Sec};
%  \node[text=black,font=\large,below right] (secnum) at ([shift={(0,-1mm)}]sec.north east) {\begin{varwidth}{.85\linewidth}\setlength\baselineskip{18pt}\hspace{.5cm}#1\end{varwidth}};
%\end{tikzpicture}}}%最后一个选项为 [<after code>]
%
%\titlespacing*{\section}{0pt}{3.5ex plus 1ex minus .2ex}{2.3ex plus .2ex}


% ==== ElegantBook fancy-mode exercise (standalone) ====
\makeatletter
\newcommand{\exercisename}{}
\definecolor{main}{RGB}{0,120,2}
\colorlet{outermarginfgcolor}{DarkCyan} % foregroundcolor 较深的前景色
\colorlet{outermarginbgcolor}{white} % backgroundcolor 较浅的背景色
% 设定主编号上级计数器(tcolorbox auto counter 使用)
\def\ELEGANT@thmcnt{section}

\tcbset{new/usesamecnt/.style={}}

\tcbset{
  common/.style={
    fontupper=\itshape,
    lower separated=false,
    coltitle=white,
    colback=gray!5,
    boxrule=0.5pt,
    fonttitle=\bfseries,
    enhanced,
    breakable,
    top=1em,
    before skip=8pt,
    attach boxed title to top left={
      yshift=-0.11in,
      xshift=0.15in},
    boxed title style={
      boxrule=0pt,
      colframe=white,
      arc=0pt,
      outer arc=0pt},
    separator sign={.},
  },
  defstyle/.style={
    common,
    colframe=outermarginfgcolor,
    colback=outermarginbgcolor,
    arc=0pt,
    colbacktitle=outermarginfgcolor,
    overlay unbroken and last={
      % 更稳健的定位,用 frame.south east 避免 width 变量依赖
      \node[anchor=south east, outer sep=0pt] at (frame.south east) {\textcolor{outermarginfgcolor}{$\blacksquare$}};},
  },
  ELEGANT@title/.code n args={2}{
    \tcbset{title={\csname #1name\endcsname~%
      %\ifdef{\thetcbcounter}{\thetcbcounter}{}%不需要计数器
      \ifblank{#2}{}{ (#2)}}}
  },
  ELEGANT@label/.code n args={2}{\ifblank{#2}{}{\tcbset{label={#1:#2}}}},
}

\NewDocumentCommand\DefineNewEnvironment{ m m m O{} }{%
  \ifcsundef{#1name}{}{\relax}%
  \expandafter\ifblank\expandafter{#4}{\tcbset{new/usecnt/.style={}}}{\tcbset{new/usecnt/.style={use counter from = #4}}}%
  \DeclareTColorBox[auto counter,number within=\ELEGANT@thmcnt,usesamecnt,usecnt]{#1}{ g o t\label g }{%
    common,#3,
    IfValueTF={##1}{ELEGANT@title={#1}{##1}}{IfValueTF={##2}{ELEGANT@title={#1}{##2}}{ELEGANT@title={#1}{}}},%
    IfValueT={##4}{%
      IfBooleanTF={##3}{label={##4}}{ELEGANT@label={#2}{##4}}%
    }%
  }
  \DeclareTColorBox{#1*}{ g o }{%
    common,#3,
    IfValueTF={##1}{ELEGANT@title={#1}{##1}}{IfValueTF={##2}{ELEGANT@title={#1}{##2}}{ELEGANT@title={#1}{}}},%
  }
}

\DefineNewEnvironment{exercise}{def}{defstyle}



% ---- example 环境定义(仿照 exercise) ----
% 名称显示文字
\newcommand{\examplename}{例}
% 配色与样式:参考 defstyle,略作颜色区分,正文不用斜体
	\tcbset{
  examplestyle/.style={
    common,
    fontupper=\normalfont, % 与 exercise 区分:正文常规体
    colframe=RoyalBlue4,
    colback=RoyalBlue1!5,
    colbacktitle=RoyalBlue4,
    arc=0pt,
    overlay unbroken and last={
      \node[anchor=south east, outer sep=0pt] at (frame.south east) {\textcolor{RoyalBlue4}{$\blacksquare$}};},
  },
}
% 新 example 环境:内部计数共享同一 section 层级(与 exercise 相同)
\DefineNewEnvironment{example}{ex}{examplestyle}
% 使用方式:\begin{example}[可选标题]\label{ex:xxx} ... \end{example}
\makeatother
% ==== End exercise block ====

%%%marker环境
\newtcolorbox{marker}[1][]{enhanced,before skip=2mm,
	after skip=3mm,fontupper=\rmfamily,
	boxrule=0.4pt,left=5mm,right=2mm,top=1mm,bottom=1mm,
	colback=yellow!50,colframe=yellow!20!black,
	sharp corners,rounded corners=southeast,
	arc is angular,arc=3mm,underlay={%
		\path[fill=tcbcolback!80!black] ([yshift=3mm]interior.south east)--++(-0.4,-0.1)--++(0.1,-0.2);
		\path[draw=tcbcolframe,shorten <=-0.05mm,shorten >=-0.05mm] ([yshift=3mm]interior.south east)--++(-0.4,-0.1)--++(0.1,-0.2);
		\path[fill=yellow!50!black,draw=none] (interior.south west) rectangle node[white]{\Huge\bfseries !} ([xshift=4mm]interior.north west);
	},
	drop fuzzy shadow,#1
}
\usepackage{float}
\usepackage{fontawesome5}
\usepackage{pifont}

\newbreakabletheorem[small box style={draw=orange!30!black!20,rounded corners,%
 fill=orange!10!black!2,decoration=penciline, decorate, thick},
 headfont=\bfseries\large,
 big box style={color=orange!30!black!20,fill=yellow!5,thick},
 broken edges={draw=orange!30!black!20,thick,fill=orange!20!black!5,
 decoration={random steps, segment length=.5cm,%
 amplitude=1.3mm},decorate},%
 other edges={decoration=penciline,decorate,thick}]%
 {exer}{例题}{exer}
 
 
\newboxedtheorem[small box style={fill=blue!20, draw=black, line width=.7pt,
decoration={penciline},decorate},
big box style={draw=black,thin,
decoration={penciline},decorate},
headfont=\bfseries]%
{solution}{解:}{}
 
 

\begin{document}
\section{习题}
\begin{exer}
$\exists x >0,使得\dfrac{4x^2+1}{2x}-3m+m^2<0成立,求实数m的取值范围$
\end{exer}
\begin{solution}
\begin{flalign*}
&原不等式可化为:m^2-3m < -(\dfrac{4x^2+1}{2x}) = -(2x+\dfrac{1}{2x})&\\
&问题转化为:m^2-3m < -(2x+\dfrac{1}{2x})_{max}。&\\
&\because x>0,\therefore 2x+\dfrac{1}{2x}\ge 2\sqrt{2x\cdot\dfrac{1}{2x}}=2。当且仅当x=\dfrac{1}{2}时取等。&\\
&\therefore -(2x+\dfrac{1}{2x})的最大值为-2。&\\
&\therefore m^2-3m < -2 \iff m^2-3m+2<0 \iff (m-1)(m-2)<0&\\
&解得:1<m<2&
\end{flalign*}
\end{solution}

\begin{exer}
$若正实数a,b满足a+b=1,则ax^2+\big(3+\dfrac{1}{b}\big)x-a=0的根的最大值为?$
\end{exer}
\begin{solution}\small
\begin{flalign*}
&解:设方程为 ax^2+\left(3+\dfrac{1}{b}\right)x-a=0。由韦达定理可知,两根之积 x_1x_2 = \dfrac{-a}{a}=-1。&\\
&这表明方程必有一正根和一负根。我们要求的是根的最大值,即正根的最大值。&\\
&设正根为 x(x>0),则另一根为 -\dfrac{1}{x}。&\\
&两根之和为 x_1+x_2 = x - \dfrac{1}{x} = -\dfrac{3+\frac{1}{b}}{a} = -\dfrac{3b+1}{ab}。&\\
&x - \dfrac{1}{x} = -\dfrac{3b+1}{ab} = -\dfrac{3b+a+b}{ab} = -\left(\dfrac{4b+a}{ab}\right) = -\left(\dfrac{4}{a}+\dfrac{1}{b}\right)&\\
&我们要求 x 的最大值。因为 x>0,f(x)=x-\dfrac{1}{x} 是一个增函数。因此,要使 x 最大,就需要使 x-\dfrac{1}{x} 最大。&\\
&这等价于求 -\left(\dfrac{4}{a}+\dfrac{1}{b}\right) 的最大值,也就是求 \left(\dfrac{4}{a}+\dfrac{1}{b}\right) 的最小值。&\\
&\textbf{权方和}:\dfrac{4}{a}+\dfrac{1}{b}  \ge \dfrac{(2+1)^2}{a+b} = 9。当且仅当 \dfrac{2}{a}=\dfrac{1}{b},即 a=2b 时取等。&\\
&又\because a+b=1 \therefore 2b+b=1 \implies b=\dfrac{1}{3}, a=\dfrac{2}{3}。&\\
&所以 x-\dfrac{1}{x} 的最大值为 -9。即 x^2+9x-1=0。&\\
&解得 x=\dfrac{-9\pm\sqrt{81+4}}{2}。因为 x>0,所以 x=\dfrac{-9+\sqrt{85}}{2}。&
\end{flalign*}
\end{solution}

\begin{exer}
$已知集合A=\{x|0\le x \le a\},集合B=\{x|m^2+5\le x \le m^2+6\}。若命题“\exists m\in\mathbb{R},A\cap B=\emptyset”为假命题,则实数a的取值范围?$
\end{exer}
\begin{solution}\small
\begin{flalign*}
&解:设命题 p 为:“\exists m\in\mathbb{R},A\cap B=\emptyset”。题目指出命题 p 为假命题,则其否定形式 \neg p 必为真命题。&\\
&\neg p:“\forall m\in\mathbb{R},A\cap B \neq \emptyset”。这意味着对于任意实数 m,集合 A 和集合 B 的交集都非空。&\\
&集合 A = [0, a],集合 B = [m^2+5, m^2+6]。&\\
&\because m^2 \ge 0, \therefore m^2+5 \ge 5。集合 B 是一个始终在区间 [5, +\infty) 内的、长度为1的闭区间。&\\
&要使 A 与所有可能的 B 都有交集,A必须覆盖 B 可能出现的所有范围,即 [5, +\infty)。&\\
&因此 A=[0,a] 必须是 [0, +\infty),这意味着 a\to+\infty,这在实数 a 的取值范围中是不可能的。&\\
&我们\textbf{怀疑}原命题可能存在印刷错误,正确的应该是:“\forall m\in\mathbb{R},A\cap B=\emptyset”为假命题。&\\
&那么其真命题形式是:“\exists m\in\mathbb{R},A\cap B \neq \emptyset”。&\\
&这意味着,存在至少一个实数 m,使得集合 A 和 B 的交集不为空。&\\
&要使 A=[0, a] 和 B=[m^2+5, m^2+6] 有交集,需要满足 a \ge m^2+5。&\\
&我们需要存在至少一个 m 使得 a \ge m^2+5 成立。&\\
&函数 f(m) = m^2+5 的最小值为 5(当 m=0 时)。&\\
&为了让不等式 a \ge m^2+5 有解,a 必须大于或等于 m^2+5 的最小值。&\\
&所以,a \ge 5。&\\
&因此,实数a的取值范围是 [5, +\infty)。&
\end{flalign*}
\end{solution}

\begin{exer}[19题]
若集合$\{x|px^2+8x-4=0\}=\{q\},则p+q的取值可以为?$
\end{exer}
\begin{solution}\small
\begin{flalign*}
&若集合 \{x | px^2+8x-4=0\} = \{q\},这意味着方程 px^2+8x-4=0 只有一个解,这个解就是 q。&\\
&一个方程只有一个解,需要分两种情况讨论:&\\
&\textbf{情况一:该方程为一元一次方程} &\\
&当 p=0 时,方程变为一个一元一次方程:8x - 4 = 0。解得 x = \dfrac{1}{2}。&\\
&此时,集合为 \{\dfrac{1}{2}\},所以 q=\dfrac{1}{2}。p+q = 0 + \dfrac{1}{2} = \dfrac{1}{2}。&\\
&\textbf{情况二:该方程为一元二次方程且判别式为0} &\\
&当 p \neq 0 时,该方程为一元二次方程。要使其只有一个解,判别式 \Delta 必须为0。&\\
&\Delta = b^2 - 4ac = 8^2 - 4(p)(-4) = 64 + 16p。&\\
&令 \Delta = 0,则:64 + 16p = 0,解得 p = -4。&\\
&此时,方程为 -4x^2+8x-4=0。唯一的解 q = \dfrac{-b}{2a} = \dfrac{-8}{2(-4)} = 1。&\\
&此时,p+q = -4 + 1 = -3。&\\
&综上所述,p+q 的取值可以为 \dfrac{1}{2} 或 -3。&
\end{flalign*}
\end{solution}

\begin{exer}[20题]
$已知实数m,n,p且m>1,mn-n-2m^2=0,m^2+n+4=4m+p,则下列结论正确的是:\\A.\quad n>m\quad B.\quad p>m \quad C.\quad p<n\quad D.\quad n<p$
\end{exer}
\begin{solution}\small
\begin{flalign*}
&1. \textbf{分析第一个方程}: mn - n - 2m^2 = 0&\\
&n(m-1) = 2m^2&\\
&\text{因为 } m>1, \text{所以 } m-1>0, \text{我们可以得到:}&\\
&n = \dfrac{2m^2}{m-1}&\\
\\
&2. \textbf{比较 n 和 m (选项A)}&\\
&n-m = \dfrac{2m^2}{m-1} - m = \dfrac{2m^2 - m(m-1)}{m-1} = \dfrac{m^2+m}{m-1}&\\
&\text{因为 } m>1, \text{所以分子 } m^2+m > 0, \text{分母 } m-1 > 0。&\\
&\therefore n-m > 0 \implies n > m. \text{ 所以选项A正确。}&\\
\\
&3. \textbf{分析第二个方程}: m^2+n+4=4m+p&\\
&p = m^2-4m+4+n = (m-2)^2+n&\\
\\
&4. \textbf{比较 p 和 n (选项C, D)}&\\
&p-n = (m-2)^2 \ge 0&\\
&\text{这意味着 } p \ge n. \text{ 仅当 m=2 时 p=n。}&\\
&\text{所以选项C (p<n) 错误,选项D (n<p) 不总是成立。}&\\
\\
&5. \textbf{比较 p 和 m (选项B)}&\\
&\text{我们已经证明了 } p \ge n \text{ 且 } n > m。&\\
&\text{根据不等式的传递性,} p \ge n > m \implies p > m.&\\
&\text{所以选项B也正确。}&\\
\\
&\textbf{结论}: \text{如果这是单选题,题目可能存在问题,因为A和B都正确。如果这是多选题,则选A和B。}&
\end{flalign*}
\end{solution}


\end{document}
