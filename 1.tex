\section{常见习题}
%\begin{example}[2023广东月考]\label{ex1}
%已知命题$p:\forall x \in R,ax^2+2x+3>0$。若命题$p$为假命题,则实数a的取值范围是多少?
%\end{example}
%\begin{flalign*}
%&\because p假\iff \textlnot p 为真。&\\
%&\therefore 原命题变成: \exists x \in R,ax^2+2x+3\le 0成立。&\\
%&\begin{cases}
%&a=0,原式=2x+3,成立\\
%&a < 0,成立\\
%&a>0,只要\Delta\ge 0即可。解得:a\in(-\infty,\frac{1}{3}]。前提是a>0,因此a\in(0,\frac{1}{3}]
%\end{cases}\\
%&综上,a的取值范围为:\big\{a|a\le\frac{1}{3}\big\}
%\end{flalign*}
%\begin{marker}
%二次函数先看最高次项的系数是否为0。
%\end{marker}
%
%\begin{example}[2024孝感开学考]
%已知$a>0,b>0,且a+2b=2,则(\qquad)$
%\begin{table}[H]
%\begin{tabular}{cc}
%A.$\quad\frac{1}{a+1}+\frac{2}{b}$的最小值是3$\quad$&B.$\quad \frac{ab}{2a+b}$的最大值为$\frac{2}{9}$\\
%C.$\quad\frac{(a+2)(b+1)}{ab}$的最大值为9$\quad$&D.$\quad\frac{b}{a}+\frac{1}{ab}$的最小值为$1+\sqrt{2}$
%\end{tabular}
%\end{table}
%\end{example}
%\begin{flalign*}
%&\begin{cases}
%&A:\frac{1}{a+1}+\frac{2}{b}=\frac{1}{a+1}+\frac{2\times 2}{2\times b}\ge\frac{(1+2)^2}{a+1+2b}=3,\checkmark\\
%&B:\frac{ab}{2a+b}=\frac{1}{\frac{2}{b}+\frac{1}{a}}。分母“1”的代换:(\frac{2}{b}+\frac{1}{a})\times(a+2b)\times\frac{1}{2}=(\frac{2a}{b}+\frac{2b}{a}+5)\times\frac{1}{2}\ge\frac{9}{2}\quad\therefore \le\frac{2}{9} \checkmark\\
%&C:\frac{ab+a+2b+2}{ab}=\frac{ab+4}{ab}=1+\frac{4}{ab}。\because a+2b=2\therefore a\times(2b)\le(\frac{a+2b}{2})^2=1\therefore ab\le\frac{1}{2}。原式\ge 1+8=9 错误。\\
%&D:齐次化,把条件两边同时平方得:\frac{2b}{a}+\frac{a}{4b}+1\ge 1+\sqrt{2} \checkmark
%\end{cases}&
%\end{flalign*}
\begin{exer}
已知 $x>0, y>0$ 且 $x^2+y^2=x-y$,求 $\displaystyle \frac{x+y+1}{x+2y}$ 的最小值。
%\tcblower
\\
\itshape\small
\begin{flalign*}
解:&\because x^2+y^2=x-y\iff \dfrac{x^2+y^2}{x-y}=1&\\
&\therefore \frac{x+y+1}{x+2y}=\frac{x+y+ \dfrac{x^2+y^2}{x-y}}{x+2y}= \dfrac{\dfrac{(x+y)\times(x-y)}{x-y}+\dfrac{x^2+y^2}{x-y}}{x+2y}=\dfrac{\dfrac{2x^2}{x-y}}{x+2y}=\dfrac{2x^2}{(x-y)(x+2y)}=\dfrac{2x^2}{x^2+xy-2y^2}\\
&分子分母同除以x^2得:\dfrac{2}{1+\dfrac{y}{x}-2(\dfrac{y}{x})^2}。\\
&令\dfrac{y}{x}=t(t > 0),则分母变为:1+t-2t^2,整理得:-2t^2+t+1\\
&当t=-\dfrac{1}{2\times(-2)},即t=\dfrac{1}{4}时取得最大值\dfrac{9}{8}。\\
&\therefore \dfrac{2}{1+\dfrac{y}{x}-2(\dfrac{y}{x})^2}\ge\dfrac{2}{\frac{9}{8}}=\dfrac{16}{9}
\end{flalign*}

\begin{center}
\begin{tikzpicture}[scale=.75]
  \begin{axis}[
    width=9cm,height=5.5cm,
    xmin=0,xmax=2.2,
    ymin=-4,ymax=2,
    domain=0:2.2,
    samples=200,
    xlabel={$t$},ylabel={$y$},
    xtick={0,0.5,1,1.5,2},
    ytick={-4,-3,-2,-1,0,1,2},
    grid=both,
    grid style={line width=.2pt, draw=gray!30},
    major grid style={line width=.3pt,draw=gray!55},
    smooth
  ]
    % 曲线 y=1+t-2t^2
    \addplot[very thick,outermarginfgcolor] {1+x-2*x^2};
    % 标注顶点
    \addplot[mark=*,only marks,mark size=1.6pt,color=red] coordinates {(0.25,1.125)};
    \node[anchor=south west,font=\small] at (axis cs:0.25,1.125) {$\left(\tfrac{1}{4},\tfrac{9}{8}\right)$};
    % 零点 (求解 1+t-2t^2=0 => t=(1\pm\sqrt{1+8})/4=(1\pm3)/4 -> t=1,-1/2)
    \addplot[mark=*,only marks,mark size=1.6pt,color=blue] coordinates {(1,0)};
    \node[anchor=north east,font=\small] at (axis cs:1,0) {$t=1$};
    % 标注函数表达式
    \node[xshift=1cm,anchor=north east,font=\small,fill=white,inner sep=1pt] at (axis cs:1.2,1.2) {$y=-2t^{2}+t+1$};
  \end{axis}
\end{tikzpicture}
\end{center}
\end{exer}%\newpage
\begin{exer}
已知正实数 $a,b$ 满足 $\dfrac{1}{(2a+b)b}+\dfrac{2}{(2b+a)a}=1$,求 $ab$ 的最大值。

\itshape\small
解一:
\begin{flalign*}
&左右同乘以ab得:\dfrac{a}{2a+b}+\dfrac{2b}{2b+a}=ab&\\
&令\begin{cases}
2a+b=m\\
2b+a=n
\end{cases}则:\begin{cases}
a=\dfrac{2m-n}{3}\\
b=\dfrac{2n-m}{3}
\end{cases}\\
&\therefore ab=\dfrac{\dfrac{2m-n}{3}}{m}+\dfrac{2\times\dfrac{2n-m}{3}}{n}=\dfrac{2m-n}{3m}+\dfrac{4n-2m}{3n}=\dfrac{2}{3}+\dfrac{4}{3}-\big(\dfrac{n}{3m}+\dfrac{2m}{3n}\big)\\
&\because \dfrac{n}{3m}+\dfrac{2m}{3n}\ge 2\sqrt{\dfrac{n}{3m}\cdot\dfrac{2m}{3n}}=\dfrac{2\sqrt{2}}{3}\quad(当3n^2=6m^2,即n=\sqrt{2}m,取等)\\
&\therefore ab_{max}=2-\dfrac{2\sqrt{2}}{3}
\end{flalign*}
解二:
\itshape\small\kaishu
\begin{flalign*}
&\therefore ab=\dfrac{1}{2+\dfrac{b}{a}}+\dfrac{2}{2+\dfrac{a}{b}}&\\
&令\dfrac{b}{a}为t(t>0),则原式变为:ab=\dfrac{1}{2+t}+\dfrac{2}{2+\dfrac{1}{t}},显然ab<\dfrac{1}{2}+\dfrac{2}{2}<2\\
&即求\dfrac{1}{2+t}+\dfrac{2}{2+\dfrac{1}{t}}的最大值,为了方便设为K(K=ab,k\in(0,2)),则:\\
&K=\dfrac{1}{2+t}+\dfrac{2}{2+\dfrac{1}{t}}=\dfrac{1}{2+t}+\dfrac{2t}{2t+1}\\
&两边同乘以 (2+t)(2t+1),得\\
& (2+t) (2t+1)=(2t+1)+2t (t+2)\\
&化简,得:(2K-2)t^2+(5K-6)t+2K-1=0\\
&\because 等式成立,方程一定有解\\
&\therefore \Delta=(5K-6)^2-4(2K-2)(2K-1)=9K^2-36K+28\ge 0\\
&\therefore K\ge 2+\dfrac{2\sqrt{2}}{3} 或 K\le 2-\dfrac{2\sqrt{2}}{3}\\
&又\because K\in(0,2)\\
&\therefore K\in(0,2-\dfrac{2\sqrt{2}}{3})\\
&ab的最大值为,2-\dfrac{2\sqrt{2}}{3}
\end{flalign*}
\end{exer}