\documentclass[CJKmath]{exam-zh}
\examsetup{
	page/size=a4paper,
	paren/show-answer=true,
	solution/show-solution=show-stay,
}
\begin{document}
\section{恒成立、存在性问题}

\begin{question}
已知关于$x$的不等式$2x-1 > m(x^2-1)$
\begin{enumerate}
    \item 是否存在实数$m$,使不等式对任意$x \in \mathbb{R}$恒成立?
    \item 若不等式对于$x \in (1, +\infty)$恒成立,求$m$的取值范围.
    \item 若不等式对于$m \in [-2, 2]$恒成立,求实数$x$的取值范围.
\end{enumerate}
\begin{solution}
原不等式等价于$-mx^2+2x+(m-1)>0$。

1) 若对$\forall x\in\mathbb R$恒成立,则当$|x|\to\infty$,左侧$\sim -m x^2$必须不趋于$-\infty$,故需$m\le 0$;又代入$x=0$得$-1>-m$,即$m>1$,矛盾,故\textbf{不存在}这样的$m$。

2) 对于$x\in(1,\infty)$恒成立,当$x>1$时$x^2-1>0$,可写成
$$m<\frac{2x-1}{x^2-1}=:\varphi(x).$$
对$x>1$,有$\varphi'(x)=-\dfrac{2(x^2-x+1)}{(x^2-1)^2}<0$,故$\varphi$在$(1,\infty)$单调递减,
$\inf\varphi(x)=\lim\limits_{x\to\infty}\varphi(x)=0$,且$\varphi(x)>0$。因此$\,m\le 0$。

3) 要对所有$m\in[-2,2]$恒成立,相当于
$$2x-1>\max_{m\in[-2,2]} \{m(x^2-1)\}=\begin{cases}2(x^2-1),&|x|>1,\\-2(x^2-1),&|x|<1,\\0,&|x|=1.\end{cases}$$
分别求解:
- $|x|>1$: $2x-1>2(x^2-1)\iff 2x^2-2x-1<0\iff \dfrac{1-\sqrt3}{2}<x<\dfrac{1+\sqrt3}{2}$,与$|x|>1$取交,得$x\in\big(1,\tfrac{1+\sqrt3}{2}\big)$;
- $|x|<1$: $2x-1>-2(x^2-1)\iff 2x^2+2x-3>0$,两根为$\dfrac{-1\pm\sqrt7}{2}$,在$|x|<1$内取$x>\tfrac{-1+\sqrt7}{2}$,即$x\in\big(\tfrac{-1+\sqrt7}{2},1\big)$;
- $|x|=1$: $x=1$成立,$x=-1$不成立。

综上,$x\in\big(\tfrac{-1+\sqrt7}{2},\tfrac{1+\sqrt3}{2}\big)$。

答案:
(1) 不存在;(2) $m\le 0$;(3) $x\in\big(\tfrac{-1+\sqrt7}{2},\tfrac{1+\sqrt3}{2}\big)$。
\end{solution}
\end{question}

\begin{question}
$f(x) = \sqrt{(m^2 - m - 6)x^2 + (m + 2)x + 8}$
\begin{enumerate}
    \item 若$f(x)$的定义域为$[-1, 2]$,求$m = \underline{\qquad\qquad}$
    \item 若$f(x)$的定义域为$\mathbb{R}$,求实数$m$的取值范围.
    \item 若$f(x)$的值域为$[0, +\infty)$,求实数$m$的取值范围.
    \item 若$f(x)$的值域为$[1, +\infty)$,则$m = \underline{\quad}$
\end{enumerate}
\end{question}

\begin{question}
\begin{enumerate}
    \item 若不等式$\dfrac{1}{x} + \dfrac{1}{1 - 4x} - m \geq 0$对$x \in (0, \dfrac{1}{4})$恒成立,则$m$的最大值为$\underline{\qquad\qquad}$
    \item 若存在$x \in \mathbb{R}$使$|x - 1| + |x + 2| \leq m$,求$m$的取值范围.
    \item 若存在$x \in [1, 3]$使$x^2 - 2mx + m^2 - 1 < 0$成立,求$m$的取值范围.
\end{enumerate}
\end{question}

\begin{question}{(口诀):“\textbf{任意}”即“扛住最坏情况”,“\textbf{存在}”即“只要有一个即可”}
\begin{enumerate}
    \item 已知$f(x) = x^2 - x + 3$,$g(x) = 2x - a$,若对$\forall x_1 \in [0, 2]$,$\exists x_2 \in [0, 2]$使$f(x_1) \geq g(x_2)$,求$a$的取值范围.
    \begin{itemize}
        \item 变式1:$\forall x_1 \in [0, 2], \forall x_2 \in [0, 2]$
        \item 变式2:$\exists x_1 \in [0, 2], \forall x_2 \in [0, 2]$
        \item 变式3:$\exists x_1 \in [0, 2], \exists x_2 \in [0, 2]$
    \end{itemize}
    \item 已知$f(x) = x^2$,$g(x) = 4 - x^2$,若对$\forall x_1 \in [0, 2]$,$\exists x_2 \in [0, 2]$使$f(x_1) + g(x_2) \geq c$,求$c$的取值范围.
\end{enumerate}
\end{question}

\begin{question}
已知对于$\forall x_1 \in (2, 3)$,$\exists x_2 \in (1, 2)$使$f(x_2) = g(x_1)$成立,其中$f(x) = -2x + 8 - 4a$,$g(x) = x^2 - 2ax$,求$a$的取值范围.
\end{question}

\section{函数性质应用压轴题}
\begin{question}
已知$f(x)$,$x \in \mathbb{R}$,$\forall x_1,x_2 \in (0,+\infty)$且$x_1 \neq x_2$,$\dfrac{x_2f(x_1)-x_1f(x_2)}{x_1-x_2} > 0$,$f(x)$为奇函数,$f(2025)=2025$,则不等式$f(x) > x$的解集为$\underline{\qquad\qquad}$.
\end{question}

\begin{question}
已知$f(x)$,$x \in \mathbb{R}$,$\forall x,y \in \mathbb{R}$,$f(x)f(y)=f(x+y)$,且$f(1)=\dfrac{1}{2}$,则\paren
\begin{choices}
\item  $f(0)=0$
\item  $f(-1)=2$
\item  $f(x+1)=f(x)$
\item  $f(x+2)-f(x+11) < f(x+1)-f(x)$
\end{choices}
\end{question}


\begin{question}
已知$f(x)$为$\mathbb{R}$上的奇函数,且$f(1-x)=f(1+x)$,若$f(1)=2$,则$f(1)+f(2)+\dots+f(2025)=$\paren
\begin{choices}
\item $0$
\item  $2025$
\item  $2024$
\item  $2$
\end{choices}
\end{question}

\begin{question}
已知$f(x)$为$\mathbb{R}$上的奇函数,当$x_1,x_2 \in (0,+\infty)$,$\dfrac{x_2f(x_1)-x_1f(x_2)}{x_1-x_2} > 0$,$f(6)=6$,则不等式$f(x) > x$的解集为\paren[B]
\begin{choices}
\item  $(-\infty,-6)\cup(0,6)$
\item  $(-6,0)\cup(6,+\infty)$
\item  $(-6,0)\cup(0,6)$
\item  $(-\infty,-6)\cup(6,+\infty)$
\end{choices}
\end{question}

\begin{question}
已知$f(x)$,$x \in \mathbb{R}$,$f(x+4)$为偶函数,$f(-x+2)$为奇函数,且$f(x)$在$[0,2]\uparrow$,则下列错误的是\paren

\begin{choices}
\item  $f(2)=0$
\item  $x=4$是$f(x)$的一条对称轴
\item  $f(x)$在$[4,8]\downarrow$
\item  $f(1) < f(7)$
\end{choices}
\end{question}

\begin{question}
$f(x)$,$x \in (0,+\infty)$,当$x_1 \neq x_2$时,$\dfrac{x_2f(x_1)-x_1f(x_2)}{x_1-x_2} > 0$,则\paren
\begin{choices}
\item $3f(4) > 4f(3)$
\item  $y=\dfrac{f(x)}{x}$在$(0,+\infty)\uparrow$
\item  $y=xf(x)$在$(0,+\infty)\uparrow$
\item  $f(3x_1+x_2)+f(x_1+3x_2) > 4f(x_1+x_2)$
\end{choices}
\end{question}

\begin{question}
若定义在$(-1,1)$上的$f(x)$不恒为$0$,对$\forall x,y \in (-1,1)$都有$f(x)+f(y)=f\left(\dfrac{x+y}{1+xy}\right)$,且当$x \in (-1,0)$时,$f(x) > 0$,则\paren
\begin{choices}
\item  $f(0)=0$
\item  $f(x)$为奇函数
\item  $f\left(\dfrac{1}{3}\right)+f\left(\dfrac{1}{4}\right) > f\left(\dfrac{1}{2}\right)$
\item  $f(x)$在$(0,1)\downarrow$
\end{choices}
\end{question}

\begin{question}
已知$f(x)=\begin{cases} x^2+(1-a)x-2a+b,\ &x < 1 \\ \dfrac{a+1}{x}+2x,\ &x \geq 1 \end{cases}$,对于$\forall x_1,x_2 \in \mathbb{R}$且$x_1 \neq x_2$,$\dfrac{f(x_1)-f(x_2)}{x_1-x_2} < 2$成立,则$a$的取值范围为$\underline{\quad\quad\quad}$.
\end{question}

\begin{question}
设$f(x)$为$\mathbb{R}$上的奇函数,$\forall x_1,x_2 \in (0,+\infty)$,$x_1 \neq x_2$,$\dfrac{x_2f(x_1)-x_1f(x_2)}{x_1-x_2} > 0$,若$f(2)=4$,则$f(x)-2x \leq 0$的解集为$\underline{\quad\quad\quad}$.
\end{question}

\begin{question}
$f(x)$,$x \in \mathbb{R}$,$\forall x,y \in \mathbb{R}$,$f(xy)=f(x)f(y)$,且$f(-1)=-1$,当$0 < x < 1$时$f(x) \in (0,1)$.
    \begin{enumerate}
        \item 判断$f(x)$的奇偶性;
        \item 判断$f(x)$在$(0,+\infty)$的单调性并证明;
        \item $\forall x_1,x_2 \in [-1,1]$,$a \in [-1,5]$,总有$2\big|f(x_1)-f(x_2)\big| \leq m^2-am-2$恒成立,求$m$的取值范围.
    \end{enumerate}
\end{question}    
\end{document}