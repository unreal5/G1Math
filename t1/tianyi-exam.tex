\documentclass{exam-zh}

\begin{document}


\section{单项选择题:本题共8小题,每小题5分,共40分.在每小题给出的四个选项中,只有一项是符合题目要求的.}

\begin{question}
若集合\( A = \{ x | x < 1 \text{ 或 } x > 2 \} \),\( B = \{ x | -1 < x < 3 \} \),则\( A \cap B = \)\paren

\begin{choices}
\item \( \{ x | -1 < x < 3 \} \)
\item \( \{ x | -1 < x < 1 \} \)
\item \( \{ x | x < 1 \text{ 或 } x > 2 \} \)
\item \( \{ x | -1 < x < 1 \text{ 或 } 2 < x < 3 \} \)
\end{choices}
\end{question}

\begin{question} “\( x = 2 \)”是“\( x^2 - 3x + 2 = 0 \)”的\paren
\begin{choices}
\item 充分不必要条件
\item 必要不充分条件
\item 充要条件
\item 既不充分也不必要条件
\end{choices}
\end{question}

\begin{question} 
若一元二次不等式\( x^2 + bx + c < 0 \)的解集为\( (-3, 4) \),则\( b + c = \)\paren
\begin{choices}
\item \( -13 \)
\item \( -12 \)
\item \( 12 \)
\item \( 13 \)
\end{choices}
\end{question}

\begin{question}
已知\( p = 2^{1.1} \),\( q = \log_4 8 \),\( r = 0.3^2 \),则\( p, q, r \)的大小关系为\paren
\begin{choices}
\item \( p > r > q \)
\item \( p > q > r \)
\item \( q > r > p \)
\item \( q > p > r \)
\end{choices}
\end{question}

\begin{question}
下列函数在定义域上既是奇函数又是增函数的是\paren
\begin{choices}
\item \( f(x) = -x \)
\item \( f(x) = -\frac{1}{x} \)
\item \( f(x) = x^{\frac{1}{2}} \)
\item \( f(x) = \begin{cases} -x^2, & x < -1 \\ x, & -1 \leq x \leq 1 \\ x^2, & x > 1 \end{cases} \)
\end{choices}
\end{question}

\begin{question}
 已知正数\( a, b \)满足\( a + b = 4 \),则\( (1 + a)(1 + b) \)的最大值为\paren
\begin{choices}
\item \( 4 \)
\item \( 5 \)
\item \( 8 \)
\item \( 9 \)
\end{choices}
\end{question}

\begin{question} 
已知函数\( f(x) = x + \frac{b}{x} \)在\( (1, +\infty) \)上单调递增,则实数\( b \)的取值范围是\paren
\begin{choices}
\item \( (-\infty, -1] \)
\item \( (-\infty, 1] \)
\item \( [-1, +\infty) \)
\item \( [1, +\infty) \)
\end{choices}
\end{question}

\begin{question}
\relax 已知函数\( f(x) \)的定义域为\( (0, +\infty) \),对任意的\( x_1, x_2 \in (0, +\infty) \),且\( x_1 \neq x_2 \),都有\\ \( \dfrac{x_2 f(x_1) - x_1 f(x_2)}{x_1 - x_2} < 0 \),则不等式\( f(x + 5) > \dfrac{f(x^2 - 25)}{x - 5} \)的解集为\paren
\begin{choices}
\item \( (5, 6) \)
\item \( (5, +\infty) \)
\item \( (6, +\infty) \)
\item \( (6, 10) \)
\end{choices}
\end{question}


\section{多项选择题:本题共3小题,每小题6分,共18分.在每小题给出的选项中,有多项符合题目要求,全部选对的得6分,部分选对的得部分分,有选错的得0分.}

\begin{question}
已知\( a, b \neq 0 \),且\( a > b \),则下列不等式一定成立的是\paren
\begin{choices}
\item \( a^2 > b^2 \)
\item \( a^3 > b^3 \)
\item \( \dfrac{1}{a} < \dfrac{1}{b} \)
\item \( 2^a > 2^b \)
\end{choices}
\end{question}

\begin{question} 已知\( a > 0 \)且\( a \neq 1 \),\( b \in \mathbf{R} \),则函数\( f(x) = bx - a \)与\( g(x) = b \cdot a^x \)在同一坐标系内的图象可能是\paren
\begin{center}
\includegraphics[width=.9\linewidth]{10t.png} % 实际使用时需插入对应图像
\end{center}
\end{question}

\begin{question}
已知函数\( f(x) = a^x - a^{-x} (a > 0 \text{ 且 } a \neq 1) \),则\paren
\begin{choices}
\item \( f(x) \)的图象过定点\( (0, 0) \)
\item \( f(x) \)在\( \mathbf{R} \)上单调递增
\item \( y = \frac{f(2x)}{2f(x)} \)为偶函数
\item 当\( a > 1 \)时,函数\( f(|x|) \)的最小值是0
\end{choices}
\end{question}


\section{填空题:本题共3小题,每小题5分,共15分.}

\begin{question}
已知函数\( f(x) = \begin{cases} x + a, & x < 0 \\ \left( \dfrac{1}{2} \right)^x, & x \geq 0 \end{cases} \),若\( f(-1) = 0 \),则\( f(x) \)的最大值为\(\underline{\quad\quad\quad\quad\quad}\).
\end{question}

\begin{question}{ 设集合\( A = \{ x \in \mathbf{N}^* | x \leq 100 \} \),\( B = \{ x^3 + 3^x | x \in A \} \),则\( A \cup B \)中的元素个数为\(\underline{\quad\quad\quad\quad\quad}\).}
\end{question}

\begin{question}{ 若不等式\( (|x| - a)(3 - x^2) \leq 0 \)对任意\( x \in \mathbf{R} \)恒成立,则实数\( a = \underline{\quad\quad\quad\quad\quad}\).}
\end{question}



\section{解答题:本题共5小题,共77分.解答应写出文字说明、证明过程或演算步骤.}


\begin{question}
\begin{enumerate}
\item 求值:\( \sqrt{\dfrac{25}{4}} - 4^0 \times 2^{-1} - \left( \dfrac{8}{27} \right)^{-\dfrac{1}{3}} + 0.125^{\dfrac{1}{3}} \);

\item 求值:\( 10^{\lg 3} + \log_2(\log_2 16) - \log_4 1 + \ln e^2 \);

\item 已知\( x + x^{-1} = 3 \),求\( \dfrac{x^{\dfrac{1}{2}} + x^{-\dfrac{1}{2}}}{x^2 + x^{-2} - 2} \)的值.
\end{enumerate}
\end{question}
%
\begin{question}
已知集合\( M = \{ x | 2m - 1 < x < m + 1 \} \),\( N = \{ x | 3^x \geq 9 \} \).
\begin{enumerate}
\item 若\( m = \dfrac{4}{3} \),求\( M \cap (\complement_{\mathbf{R}} N) \);

\item 若\( M \subseteq N \),求实数\( m \)的取值范围.
\end{enumerate}
\end{question}


\begin{question}
已知\( f(x) \)是定义在\( \mathbf{R} \)上的奇函数,且当\( x \leq 0 \)时\( f(x) = x^2 + 2x \).
\begin{enumerate}
\item  求\( f(x) \)的解析式;

\item  求\( f(x) \)的单调递增区间;

\item  若关于\( x \)的方程\( f(x) = t \)有3个不相等的实数根,求实数\( t \)的取值范围.
\end{enumerate}
\end{question}
%
\begin{question}
已知二次函数\( f(x) \)的图象经过\( (0, -3) \),\( (-1, 0) \)和\( (3, 0) \)三点.
\begin{enumerate}
\item  求\( f(x) \)的解析式;

\item  若当\( x \in [-2, 2] \)时\( f(x) > -x + 2m - 1 \)恒成立,求实数\( m \)的取值范围;

\item  求\( f(x) \)在区间\( [t - 1, t] \)上的最小值\( g(t) \).
\end{enumerate}
\end{question}

\begin{question}
已知函数\( f(x) \)对任意的实数\( a, b \),都有\( f(a + b) = f(a) + f(b) - 1 \),且当\( x > 0 \)时\( f(x) > 1 \).
\begin{enumerate}
\item  求\( f(0) \)的值;

\item  判断并证明\( f(x) \)的单调性;

\item  若存在实数\( x \),使得不等式\( f(t \cdot 4^x) + f(3 \times 2^x - 1) > 2 \)成立,求实数\( t \)的取值范围.
\end{enumerate}

\end{question}

\end{document}