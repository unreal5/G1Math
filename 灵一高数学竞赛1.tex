\documentclass[CJKmath,a4paper,10pt]{ctexart}
\usepackage{geometry}
\geometry{inner=1.5cm,outer=1.5cm,top=2.45cm,bottom=1cm}
\usepackage[dvipsnames,svgnames,x11names,table]{xcolor}


\RequirePackage{amssymb} % Must be loaded before unicode-math
\RequirePackage{unicode-math} % Math fonts in xetexorluatex
%\setmathfont{texgyrepagella-math.otf}
%\setmainfont{STIXTwoText}[
%	Path=./fonts/STIXTwoText/,
%  Extension = .otf,
%  UprightFont = STIXTwoMath-Regular,
%  BoldFont = STIXTwoText-SemiBold,
%  ItalicFont = STIXTwoText-Italic,
%  BoldItalicFont = STIXTwoText-BoldItalic,
%]
%\setmathfont{STIXTwoMath-Regular.otf}

\usepackage{tikz,calc}
\usetikzlibrary{calc,arrows,shadows.blur,tikzmark}
% By default all math in TikZ nodes are set in inline mode. Change this to
% displaystyle so that we don't get small fractions.
\everymath{\displaystyle}

\usepackage[most]{tcolorbox}
\tcbuselibrary{breakable, skins,theorems}


\usepackage{xkeyval}
\makeatletter 
\usepackage{amsmath,amssymb,amsthm}
\usepackage{xkeyval} 
\usepackage{zhlipsum}
\tcbset{
    common/.style={
      fontupper=\rmfamily,
      lower separated=true,
      coltitle=black,
      colback=gray!5,
      boxrule=0.5pt,
      fonttitle=\bfseries,
      enhanced,
      breakable,
      top=8pt,
      before skip=8pt,
      attach boxed title to top left={yshift=-0.05in,xshift=0.15in},
      boxed title style={
      	boxrule=0pt,
        colframe=black,
        arc=0pt,
        outer arc=0pt,
        drop fuzzy shadow
     	},
      separator sign={.},
      drop fuzzy shadow
     },
     litistyle/.style={  %例题风格
     	common,
      colframe=gray,% 
      colback=white,
      colbacktitle=Gold,
      sharp corners,rounded corners=southeast,
      overlay unbroken and last={
      \node[anchor=south east, outer sep=0pt] at (\linewidth-width,0){\textcolor{black!60}{$\blacksquare$}};}
         },
% -----------------------------------------------------------------------------
% Explanation: BreakableBox@title key
% The following defines a pgfkeys/tcolorbox key named
%   "BreakableBox@title" which accepts 2 arguments (see 
%   "/.code n args={2}"). When the key is used like
%   "BreakableBox@title={<env>}{<optTitle>}" the code below is executed
%   with #1=<env> and #2=<optTitle>.
%
% Purpose:
% - If the second argument (#2) is empty, set the tcolorbox title to
%     \csname <env>name\endcsname~\thetcbcounter
%   which expands to the environment name (e.g. "liti" -> \litiname)
%   followed by the current counter value.
% - If the second argument is provided, append " (#2)" to the title.
%
% Notes:
% - This relies on \makeatletter being active so that keys with '@'
%   are allowed in control sequence names.
% - \ifblank is used to test whether #2 is empty; ensure the package
%   that provides \ifblank (for example, etoolbox or xparse) is loaded
%   earlier if necessary.
% - Example usage inside the code: \tcbset{title={...}} sets the
%   tcolorbox title key accordingly.
%
         BreakableBox@title/.code n args={2}
              {
                \ifblank{#2}
                  {\tcbset{title={\csname #1name\endcsname~\thetcbcounter}}}
                  {\tcbset{title={\csname #1name\endcsname~\thetcbcounter\ (#2)}}}
              },
}      
  % define an internal control sequence \newbreakablebox for fancy mode's newtheorem
  % #1 is the environment name, #2 is the prefix of label, #3 is the style
  % style: thmstyle, defstyle, prostyle
  % e.g. \newbreakablebox{theorem}{thm}{thmstyle}
  % will define two environments: numbered ``theorem'' and no-numbered ``theorem*''
  % WARNING FOR MULTILINGUAL: this cs will automatically find \theoremname's definition,
  % WARNING FOR MULTILINGUAL: it should be defined in language settings.
  \newcommand{\newbreakablebox}[3]{
    \ifcsundef{#1name}{%
      % define a default name if not defined before
      \tcbset{BreakableBox@title/.code n args={2}{\tcbset{title={use newcommand define #1name}}} }
    }{\relax}
    \DeclareTColorBox[auto counter,number within=section]{#1}{ g o t\label g }{
        common, % use common style
        #3, % use the style passed in
        IfValueTF={##1}
          {BreakableBox@title={#1}{##1}}
          {
            IfValueTF={##2}
            {BreakableBox@title={#1}{##2}}
            {BreakableBox@title={#1}{}}
          },
        IfValueT={##4}
          {
            IfBooleanTF={##3}
              {label={##4}}
              {VIVID@label={#2}{##4}}
          }
      }
    \DeclareTColorBox{#1*}{ g o }{
        common,#3,
        IfValueTF={##1}
          {BreakableBox@title={#1}{##1}}
          {
            IfValueTF={##2}
            {BreakableBox@title={#1}{##2}}
            {BreakableBox@title={#1}{}}
          },
      }
  }
  % define several environment 
  % we define headers like \definitionname before
  \newcommand{\litiname}{例题}
  \newbreakablebox{liti}{def}{litistyle}
  
  
%补充内容
%%设置新字体
%%定义带圈数字命令
\newfontfamily{\nmfont}{circlenumber}
[%
Extension=.otf,
Path=./fonts/]

\newcommand{\quan}[1]{{\nmfont \symbol{#1}}}
\newcommand{\kk}[1]{\quan{\numexpr32+#1}}%\kk{<参数范围1-95>}96、97、98、99分别用\quan{196} \quan{197} \quan{199} \quan{201}

%脚注使用带圈数字
\newcommand*\kkctr[1]{%
  \protect\kk{\number\numexpr\value{#1}\relax}}
\renewcommand*\thefootnote{\textcolor{black}{\kkctr{footnote}}}

%%无悬挂脚注格式
\renewcommand\@makefntext[1]{%
  \setlength\parindent{2\ccwd}\selectfont
  \@thefnmark\ #1}

%修改\part,使其不分页
\def\@endpart{%
	\thispagestyle{empty}
  \vskip40\p@%
   \@afterheading}



\RequirePackage{enumitem}
%\newenvironment{myenum}{\begin{enumerate}[label=\protect\kk{\arabic*}]\small}{\end{enumerate}}%
\setlist{noitemsep}
\setlist[enumerate, 1]{label=\protect\kk{\arabic*},itemsep=0.5ex}  
\makeatother



%%%marker环境
\newtcolorbox{marker}[1][]{enhanced,before skip=2mm,
	after skip=3mm,fontupper=\rmfamily,
	boxrule=0.4pt,left=5mm,right=2mm,top=1mm,bottom=1mm,
	colback=yellow!50,colframe=yellow!20!black,
	sharp corners,rounded corners=southeast,
	arc is angular,arc=3mm,underlay={%
		\path[fill=tcbcolback!80!black] ([yshift=3mm]interior.south east)--++(-0.4,-0.1)--++(0.1,-0.2);
		\path[draw=tcbcolframe,shorten <=-0.05mm,shorten >=-0.05mm] ([yshift=3mm]interior.south east)--++(-0.4,-0.1)--++(0.1,-0.2);
		\path[fill=yellow!50!black,draw=none] (interior.south west) rectangle node[white]{\Huge\bfseries !} ([xshift=4mm]interior.north west);
	},
	drop fuzzy shadow,#1
}
\begin{document}



\section{解答题(本题共5小题,共77分)}
\begin{liti}
已知命题 \(p\):实数 \(x\) 满足 \(x^2-10x+16 \leq 0\),命题 \(q\):实数 \(x\) 满足 \(x^2-4mx+3m^2 \leq 0\)(其中 \(m>0\))。
\begin{enumerate}
	\item 若 \(m=1\),且命题 \(p\) 和 \(q\) 中至少有一个为真命题,求实数 \(x\) 的取值范围;
	\item 若 \(q\) 是 \(p\) 的充分条件,求实数 \(m\) 的取值范围。
\end{enumerate}
\tcblower
\textbf{解:}

命题 \(p\):由 \(x^2-10x+16 \leq 0\) 得 \((x-2)(x-8) \leq 0\),解得 \(2 \leq x \leq 8\)。

命题 \(q\): \(x^2-4mx+3m^2 \leq 0\) 得 \((x-m)(x-3m) \leq 0\)。因为 \(m>0\),所以 \(m < 3m\)。\(\therefore 命题q\):\(m \leq x \leq 3m\)。

\begin{enumerate}
    \item 当 \(m=1\) 时,\(q\):\(1 \leq x \leq 3\)。
    
    因为命题 \(p\) 和 \(q\) 中至少有一个为真命题,所以 \(x\) 的取值范围是两个不等式解集的并集。
    
    即 \(\{x \mid 2 \leq x \leq 8\} \cup \{x \mid 1 \leq x \leq 3\} = \{x \mid 1 \leq x \leq 8\}\)。
    
    所以实数 \(x\) 的取值范围是 \([1, 8]\)。
    
    \item “\(q\) 是 \(p\) 的充分条件”等价于 \(q \implies p\),即命题 \(q\) 成立的 \(x\) 集合是命题 \(p\) 成立的 \(x\) 集合的\textbf{子集}。
    
    即 \([m, 3m] \subseteq [2, 8]\)。
    
    \begin{flalign*}
    &   所以需要满足:
    \begin{cases}
        m \geq 2 \\
        3m \leq 8
    \end{cases}&
    \end{flalign*}
    解得 \(2 \leq m \leq \dfrac{8}{3}\)。
    
    这个结果满足 \(m>0\) 的前提。
    
    所以实数 \(m\) 的取值范围是 \(\left[2, \dfrac{8}{3}\right]\)。
\end{enumerate}
\end{liti}



\begin{liti}
已知函数 \(f(x)=ax^2-(a+2)x+b\)。
    \begin{enumerate}
        \item[(1)] 若 \(f(x) \leq 0\) 的解集为 \(\{x \mid 1 \leq x \leq 2\}\),求 \(a,b\) 的值;
        \item[(2)] 若 \(b=2\),求不等式 \(f(x) \leq 0\) 的解集;
        \item[(3)] 在(1)的条件下,若对任意 \(x>1\),不等式 \(\dfrac{f(x)+1}{ax-1} \geq 2k^2+k\) 恒成立,求实数 \(k\) 的取值范围。
    \end{enumerate}
\tcblower
\textbf{解:}
\begin{enumerate}
    \item[(1)]
    $\because$不等式 \(f(x) \leq 0\) 的解集为 \(\{x \mid 1 \leq x \leq 2\}\),这说明抛物线 \(y=f(x)\) 开口向上,且方程 \(ax^2-(a+2)x+b=0\) 的两个根是 1 和 2。\\
    \(\therefore a>0\)。
    \begin{flalign*}
    &根据韦达定理:
    \begin{cases}
        1+2 = \dfrac{a+2}{a} \\
        1 \times 2 = \dfrac{b}{a}
    \end{cases},\quad
     求解,得:\begin{cases}
    a=1\\
    b=2
    \end{cases}\quad&
     \end{flalign*}
    \(\because\quad a=1>0\),符合开口向上的条件\\
    \(\therefore\quad a=1, b=2\)。
    
    \item[(2)]
    当 \(b=2\) 时,不等式为 $ax^2-(a+2)x+2 \leq 0 \iff   (ax-2)(x-1) \leq 0$。\\
    方程的根为 \(x=1\) 和 \(x=\dfrac{2}{a}\) (当 \(a \neq 0\) 时)。\\
    分情况讨论 \(a\) 的取值:
    \begin{itemize}
        \item 若 \(a>2\),则 \(\dfrac{2}{a} < 1\)。抛物线开口向上,解集为 \(\left[\dfrac{2}{a}, 1\right]\)。
        \item 若 \(a=2\),不等式为 \(2(x-1)^2 \leq 0\),解为 \(x=1\)。
        \item 若 \(0<a<2\),则 \(\dfrac{2}{a} > 1\)。抛物线开口向上,解集为 \(\left[1, \dfrac{2}{a}\right]\)。
        \item 若 \(a=0\),不等式为 \(-2x+2 \leq 0\),解得 \(x \geq 1\)。解集为 \([1, +\infty)\)。
        \item 若 \(a<0\),则 \(\dfrac{2}{a} < 0 < 1\)。抛物线开口向下,解集为 \(\left(-\infty, \dfrac{2}{a}\right] \cup [1, +\infty)\)。
    \end{itemize}
    
    \item[(3)]    根据 (1) 的条件,\(a=1, b=2\), \(\therefore\quad f(x)=x^2-3x+2\)。\\
    不等式为 \(\dfrac{x^2-3x+2+1}{x-1} \geq 2k^2+k\),即 \(\dfrac{x^2-3x+3}{x-1} \geq 2k^2+k\)。
    
    设 \(g(x) = \dfrac{x^2-3x+3}{x-1}\)。因为 \(x>1\),令 \(t=x-1\),则 \(t>0\),且 \(x=t+1\),则:\\
    $
    g(x) = \dfrac{(t+1)^2-3(t+1)+3}{t} = \dfrac{t^2+2t+1-3t-3+3}{t} = \dfrac{t^2-t+1}{t} = t+\dfrac{1}{t}-1
    $
    
    根据基本不等式,当 \(t>0\) 时,\(t+\dfrac{1}{t} \geq 2\sqrt{t \cdot \dfrac{1}{t}} = 2\),当且仅当 \(t=1\) 时取等号。
    所以 \(g(x)\) 的最小值为 \(2-1=1\)。
    
    因为不等式对任意 \(x>1\) 恒成立,所以 \(2k^2+k\) 不能超过 \(g(x)\) 的最小值。
    \[
    2k^2+k \leq 1 \implies 2k^2+k-1 \leq 0 \implies (2k-1)(k+1) \leq 0
    \]
    解得 \(-1 \leq k \leq \dfrac{1}{2}\)。
    
    所以实数 \(k\) 的取值范围是 \(\left[-1, \dfrac{1}{2}\right]\)。
\end{enumerate}
\end{liti}


\begin{liti}
某学校要建造一个长方体形体育馆,其地面面积为 \(240\ \text{m}^2\),体育馆高 \(5\ \text{m}\)。甲工程队报价为:馆顶每平方米造价100元,前后两侧墙壁平均造价每平方米150元,左右两侧墙壁平均造价每平方米250元,设体育馆前墙长为 \(x\) 米。
    \begin{enumerate}
        \item[(1)] 当 \(x \in (0,50)\) 时,体育馆前墙长度为多少时,甲工程队报价最低?
        \item[(2)] 当 \(x \in (0,t]\)(\(0<t<50\))时,现有乙工程队也参与该校体育馆建造竞标,其给出的整体报价为 \(12000+500\left(\dfrac{a+1152}{x}+a\right)\) 元(\(a>0\))。若无论体育馆前墙长度 \(x\) 为多少米,乙工程队都能中标,试求 \(a\) 的取值范围。
    \end{enumerate}
\tcblower
\textbf{解:}
\begin{enumerate}
    \item[(1)]
    设体育馆前墙长为 \(x\) 米,则侧墙长为 \(y = \dfrac{240}{x}\) 米。
    体育馆的建造成本 \(C_A(x)\) 包括馆顶、前后墙和左右墙的费用。
    \begin{itemize}
        \item 馆顶面积为 \(240\ \text{m}^2\),造价为 \(240 \times 100 = 24000\) 元。
        \item 前后两侧墙壁的总面积为 \(2 \times x \times 5 = 10x\ \text{m}^2\),造价为 \(10x \times 150 = 1500x\) 元。
        \item 左右两侧墙壁的总面积为 \(2 \times y \times 5 = 10y = 10 \times \dfrac{240}{x} = \dfrac{2400}{x}\ \text{m}^2\),造价为 \(\dfrac{2400}{x} \times 250 = \dfrac{600000}{x}\) 元。
    \end{itemize}
    所以,甲工程队的总报价函数为:
    \[ C_A(x) = 24000 + 1500x + \dfrac{600000}{x} \]
    其中 \(x \in (0, 50)\)。
    
    为了求报价的最小值,我们对函数中与 \(x\) 相关的部分使用基本不等式:
    \[ 1500x + \dfrac{600000}{x} \geq 2\sqrt{1500x \cdot \dfrac{600000}{x}} = 2\sqrt{9 \times 10^8} = 2 \times 30000 = 60000 \]
    当且仅当 \(1500x = \dfrac{600000}{x}\) 时,等号成立。
    解得 \(x^2 = \dfrac{600000}{1500} = 400\),所以 \(x=20\)(因为 \(x>0\))。
    
    由于 \(x=20\) 在定义域 \((0, 50)\) 内,所以当体育馆前墙长度为20米时,甲工程队报价最低。
    
    \item[(2)]
    乙工程队的报价为 \(C_B(x) = 12000+500\left(\dfrac{a+1152}{x}+a\right)\)。
    乙工程队能中标,意味着对于任意给定的 \(x\),其报价 \(C_B(x)\) 总是低于或等于甲工程队的报价 \(C_A(x)\)。
    \[ C_B(x) \leq C_A(x) \]
    \[ 12000+500\left(\dfrac{a+1152}{x}+a\right) \leq 24000 + 1500x + \dfrac{600000}{x} \]
    化简得:
    \[ a-24 \leq 3x + \dfrac{48-a}{x} \]
    这个不等式需要在 \(x \in (0, t]\) 上恒成立。
    
    设 \(h(x) = 3x + \dfrac{48-a}{x}\)。我们需要 \(a-24 \leq \min_{x \in (0, t]} h(x)\)。
    
    对 \(h(x)\) 的单调性进行讨论:
    \begin{itemize}
        \item 如果 \(48-a \leq 0\),即 \(a \geq 48\)。\(h(x)\) 在 \((0, \infty)\) 上是增函数。最小值为 \(h(t) = 3t + \dfrac{48-a}{t}\)。
        \(a-24 \leq 3t + \dfrac{48-a}{t} \implies a \leq \dfrac{3t^2+24t+48}{t+1}\)。
        令 \(g(t) = \dfrac{3t^2+24t+48}{t+1}\),其在 \((0,50)\) 上的最小值为36。
        所以 \(a \leq 36\)。与 \(a \geq 48\) 矛盾,无解。
        
        \item 如果 \(0 < a < 48\)。\(h(x)\) 是对勾函数,在 \(x_0 = \sqrt{\dfrac{48-a}{3}}\) 处取得最小值。
        \begin{itemize}
            \item 若 \(x_0 \geq t\),即 \(a \leq 48-3t^2\)。\(h(x)\) 在 \((0, t]\) 上单调递减,最小值为 \(h(t)\)。
            得到 \(a \leq g(t)\)。
            \item 若 \(x_0 < t\),即 \(a > 48-3t^2\)。\(h(x)\) 在 \((0, t]\) 上的最小值为 \(h(x_0) = 2\sqrt{3(48-a)}\)。
            不等式变为 \(a-24 \leq 2\sqrt{3(48-a)}\)。
            若 \(a \leq 24\),显然成立。
            若 \(24 < a < 48\),平方解得 \(a \leq 36\)。
            所以此情况下要求 \(a \leq 36\)。
        \end{itemize}
    \end{itemize}
    
    “无论... \(x\) 为多少米”意味着不等式对所有 \(x \in (0, t]\) 成立。
    “无论... \(t\) 为多少”意味着对所有 \(t \in (0, 50)\) 成立。
    
    综合所有情况,要使不等式对所有 \(t \in (0,50)\) 和 \(x \in (0,t]\) 恒成立,\(a\) 必须小于等于36。
    因为 \(a>0\),所以 \(a\) 的取值范围是 \((0, 36]\)。
\end{enumerate}
\end{liti}

\begin{liti}
问题:已知 \(a,b,c\) 均为正实数,且 \(\dfrac{1}{a}+\dfrac{1}{b}+\dfrac{1}{c}=1\),求证:\(a+b+c \geq 9\)(当且仅当 \(a=b=c=3\) 时,等号成立)。\\
    证明:\(a+b+c=(a+b+c)\left(\dfrac{1}{a}+\dfrac{1}{b}+\dfrac{1}{c}\right)=3+\left(\dfrac{b}{a}+\dfrac{a}{b}\right)+\left(\dfrac{c}{a}+\dfrac{a}{c}\right)+\left(\dfrac{c}{b}+\dfrac{b}{c}\right) \geq 3+2+2+2=9\)。\\
    学习上述解法并解决下列问题:
    \begin{enumerate}
        \item[(1)] 已知 \(a,b,c\) 均为正实数,且 \(a+b+c=4\),求 \(\dfrac{1}{a}+\dfrac{4}{b}+\dfrac{9}{c}\) 的最小值;
        \item[(2)] 已知 \(a,b,x,y\) 均为正实数,且 \(\dfrac{x^2}{a^2}+\dfrac{y^2}{b^2}=1\),求证:\(a^2+b^2 \geq (x+y)^2\);
        \item[(3)] 求 \(T=\sqrt{3-2t}+\sqrt{t-1}\) 的最大值,并求出使得 \(T\) 取得最大值时 \(t\) 的值。
    \end{enumerate}
\tcblower
\textbf{解:}
\begin{enumerate}
    \item[(1)]
    \begin{flalign*}
    &\dfrac{1}{a}+\dfrac{4}{b}+\dfrac{9}{c}=\left(\dfrac{1}{a}+\dfrac{4}{b}+\dfrac{9}{c}\right)(a+b+c)\times\dfrac{1}{4}=\dfrac{1}{4}\times\left(1+\dfrac{4a}{b}+\dfrac{9a}{c}+\dfrac{b}{a}+4+\dfrac{9b}{c}+\dfrac{c}{a}+\dfrac{4c}{b}+9\right)&\\
    &=\dfrac{1}{4}\times\left[14+\left(\dfrac{b}{a}+\dfrac{4a}{b}\right)+\left(\dfrac{c}{a}+\dfrac{9a}{c}\right)+\left(\dfrac{4c}{b}+\dfrac{9b}{c}\right)\right]\geq\dfrac{1}{4}\left(14+4+6+12\right)=9\\
    &当且仅当\begin{cases}
    	\qquad\dfrac{b}{a}=\dfrac{4a}{b}\\
    	\qquad\dfrac{c}{a}=\dfrac{9a}{c}\\
    	\qquad\dfrac{4c}{b}+\dfrac{9b}{c}\\
    	\qquad a+b+c=4\\
    	\qquad a>0,b>0,c>0
    \end{cases}时取等,解得:\begin{cases}
    a= \dfrac{2}{3}\\
    b= \dfrac{4}{3}\\
    c= 2
    \end{cases}
    \end{flalign*}
       
    $\therefore$最小值为9。

    \item[(2)]
    $ a^2+b^2 = (a^2+b^2)\left(\dfrac{x^2}{a^2}+\dfrac{y^2}{b^2}\right) =x^2+\dfrac{a^2y^2}{b^2}+\dfrac{b^2y^2}{a^2}+y^2 \geq x^2 +y^2+2\sqrt{\dfrac{a^2y^2}{b^2}\cdot\dfrac{b^2y^2}{a^2}}=x^2+y^2+2xy=\left(x+y\right)^2 $\\
    当且仅当 \(\dfrac{a^2}{x^2/a^2} = \dfrac{b^2}{y^2/b^2}\iff \dfrac{a^2}{x} = \dfrac{b^2}{y}\) 时取等号。
    证毕。

    \item[(3)]
    定义域为 \(1 \leq t \leq \dfrac{3}{2}\)。
    
  	令$x=\sqrt{3-2t},y=\sqrt{t-1},a=1,b=\dfrac{\sqrt{2}}{2}$ :\\
    则$\dfrac{\sqrt{3-2t}^2}{1^2}+\dfrac{\sqrt{t-1}^2}{\left(\dfrac{\sqrt{2}}{2}\right)^2}=3-2t+2t-2=1$,满足(2)的条件:\\
    $\therefore\quad a^2+b^2\ge \left(x+y\right)^2\iff 1+\dfrac{1}{2}\geq \left(\sqrt{3-2t}+\sqrt{t-2}\right)^2$。\\
    即:$T\leq\sqrt{\dfrac{3}{2}}=\dfrac{\sqrt{6}}{2}$,当且仅当:$\dfrac{1}{\sqrt{3-2t}}=\dfrac{\dfrac{1}{2}}{\sqrt{t-1}}\iff 2\sqrt{t-1}=\sqrt{3-2t}\iff 4t-4=3-2t\iff t=\dfrac{7}{6}$时取等。
    
    \(\therefore\quad T 的最大值为 \dfrac{\sqrt{6}}{2}\),此时 \(t=\dfrac{7}{6}\)。
\end{enumerate}
\end{liti}

\begin{liti}
已知函数 \(f(x)=|3x^2-ax|+x^2-x+1\)(\(x>0\)),\(g(x)=\dfrac{f(x)}{x}\)。
    \begin{enumerate}
        \item[(1)] 若 \(a=1\),求函数 \(f(x)\) 的值域;
        \item[(2)] 若 \(a \leq 0\),试判断 \(g(x)\) 的单调性并证明;
        \item[(3)] 对 \(\forall t \in [3,4]\),\(\exists x_1,x_2 \in [\dfrac{1}{8},2]\)(\(x_1 \neq x_2\)),使得 \(t=g(x_1)=g(x_2)\),求实数 \(a\) 的取值范围。
    \end{enumerate}
\tcblower
\textbf{解:}
\begin{enumerate}
    \item[(1)] 当 \(a=1\) 时,\(f(x) = |3x^2-x| + x^2-x+1\)。函数有两个0点:$0,\dfrac{1}{3}$。又因为$x>0$:
    \begin{itemize}
        \item 当 \(0 < x \leq \frac{1}{3}\) 时,\(f(x) = -(3x^2-x) + x^2-x+1 = -2x^2+1\)。
        在 \((0, \frac{1}{3}]\) 上单调递减,值域为 \([\frac{7}{9}, 1)\)。
        \item 当 \(x > \frac{1}{3}\) 时,\(f(x) = (3x^2-x) + x^2-x+1 = 4x^2-2x+1\)。
        在 \((\frac{1}{3}, \infty)\) 上单调递增,值域为 \((\frac{7}{9}, +\infty)\)。
    \end{itemize}
    所以,函数 \(f(x)\) 的值域是 \(\left[\frac{7}{9}, +\infty\right)\)。

    \item[(2)] 当 \(a \leq 0\) 时,$3x^2-ax \geq 0$,所以 \(|3x^2-ax| = 3x^2-ax\)。\\
    \(\therefore\quad f(x) = 4x^2-(a+1)x+1\)。\\
    \(\therefore\quad g(x) = \frac{f(x)}{x} = 4x - (a+1) + \frac{1}{x}=4x + \frac{1}{x} - (a+1)\)。这是对勾函数,当$x>0$时,有最小值$g(x)\geq 2\sqrt{4x\cdot\dfrac{1}{x}-(a+1)}$;当且仅当$x=\dfrac{1}{2}$时取等。
    
    
    当 \(0 < x < \frac{1}{2}\) 时,单调递减。
    当 \(x > \frac{1}{2}\) 时,单调递增。

    \item[(3)]
    条件等价于 \([3,4] \subseteq g\left(\left[\frac{1}{8}, 2\right]\right)\) 且函数在 \([\frac{1}{8}, 2]\) 上的最小值点不是区间的端点。
    \(g(x) = |3x-a| + x-1+\frac{1}{x}\)。
    
    \textbf{超纲}
\end{enumerate}
\end{liti}
\end{document}