\subsection{待定系数法}
\begin{exer}
\\
已知$a,b>0,则\dfrac{ab+b}{a^2+b^2+1}$的最大值?

\itshape\small\noindent
对于$a^2+b^2+c^2$形式,可以利用待定系数法配凑,例如:
\[a^2+b^2+c^2=a^2+\lambda b^2 +(1-\lambda)b^2+c^2\ge2\sqrt{\lambda}ab+2\sqrt{1-\lambda}bc\]
这样就会让a,b,c三者联系起来。把例中的1当作$c^2=1$,则:
\begin{flalign*}
&原式分母\ge 2\sqrt{\lambda}ab+2\sqrt{1-\lambda}b\cdot1&\\
&分子=ab+b,ab项系数为1,b项系数也为1。我们需要对应项成比例:\\
&\dfrac{1}{2\sqrt{\lambda}}=\dfrac{1}{2\sqrt{1-\lambda}}\\
&解得\lambda=\dfrac{1}{2}\\
&\therefore 原式\le\dfrac{ab+b}{\sqrt{2}(ab+b)}=\dfrac{\sqrt{2}}{2}\\
&\therefore 最大值为\dfrac{\sqrt{2}}{2}
\end{flalign*}
\end{exer}

\begin{exer}
已知$a>0,b>0,求\dfrac{ab+b}{a^2+b^2+1}$的最大值。
\end{exer}
利用上例的配凑思想可解。
\begin{exer}
$已知x,y>0,且x+2y+\sqrt{xy}=2,求x+3y的最大值。$
\end{exer}
\kaishu
利用基本不等式可知:$\sqrt{xy}\le\dfrac{x+y}{2}$,原式变为:$2\le x+2y+\dfrac{x+y}{2}$。

此时对比题目要求的结果是x+3y。思考:如何才能让 $x+2y+\dfrac{x+y}{2}$(通过某种变换)$\rightarrow\quad x+3y?$显然$\dfrac{x+y}{2}$中,分子x,y前面有某种系数就可以。

但是的$x+y$是由$\sqrt{xy}$得来的,这是题目给出的条件,我们无法修改,我们应该如何做?

没办法改变条件,但可以对条件进行变形,利用$a\times\dfrac{1}{a}$的特性,把$\sqrt{xy}\rightarrow\sqrt{(a\cdot x) \times\dfrac{y}{a}}$; 由基本不等式得:$\sqrt{(ax) \times\dfrac{y}{a}}\le\dfrac{ax+\dfrac{y}{a}}{2}$

现在的问题就变成如何求出a的值。

原式变形为:$2=x+2y+\sqrt{(ax) \times\dfrac{y}{a}}\le x+2y +\dfrac{ax+\dfrac{y}{a}}{2}=(\dfrac{a}{2}+1)x+(2+\dfrac{1}{2a})y$
对比要求的结果:$x+3y$,$x,y的系数比1:3$

$\therefore \dfrac{\dfrac{a}{2}+1}{2+\dfrac{1}{2a}}=\dfrac{1}{3}\iff 3a^2+2a-1=0$

解得:$a=\begin{cases}
\dfrac{1}{3}\\
-\dfrac{4}{3}
\end{cases}
$

其实这二个解均可使用,但对于基本不等式,如果配的系数为负数,不满足使用条件,麻烦,取$\dfrac{1}{3}$即可。

$\therefore 2= x+2y+\sqrt{(\dfrac{1}{3}x)\times\dfrac{y}{\frac{1}{3}}}=x+2y+\sqrt{\dfrac{x}{3}\times 3y}\le x+2y+\dfrac{\dfrac{x}{3}+3y}{2}=\dfrac{7}{6}x+\dfrac{7y}{2}=\dfrac{7}{6}(x+3y)$

$\therefore x+3y\ge 2\times\dfrac{6}{7}=\dfrac{12}{7}$
%\end{flalign*}
\rmfamily
\begin{exer}
$设x,y\in \mathbf{R},若4x^2+y^2+xy=1,则2x+y的最大值为?$
\end{exer}
解:\begin{enumerate}
\item \textbf{方法一(常规万能K值法):}
\begin{flalign*}
&\textbf{求谁设谁}:设2x+y=K,则y=k-2x,因为条件简单,代入化简得:&\\
&6x^2-3kx+k^2-1=0\\
&\because 条件成立,必定存在x,即原方程有解\\
&\therefore \Delta \ge 0,即 (-3k)^2-4\times 6\times (k^2-1)\ge 0\\
&化简得:9k^2-
\end{flalign*}
\item \textbf{方法二:}利用前述配凑思想,解出配的系数为$\dfrac{\sqrt{6}}{3}$
\end{enumerate}
\begin{exercise}
$已知a,b>0,a+2b=2,则\dfrac{a^2+1}{a}+\dfrac{2b^2}{b+1}$的最小值?
\end{exercise}
第二项不好折,但看到$b^2和(b+1),在拆分的情景下,应考虑(b+1)(b-1)=b^2-1$。原式=$a+\dfrac{1}{a}+\dfrac{2(b^2-1)+2}{b+1}=a+\dfrac{1}{a}+2(b-1)+\dfrac{2}{b+1}=a+2b-2+\dfrac{1}{a}+\dfrac{2}{b+1}$,由条件知前三项为0,则变成$\dfrac{1}{a}+\dfrac{2}{b+1}$,应用1的代换或权方和速解$=\dfrac{9}{4}$,……略。
\newpage