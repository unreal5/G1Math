\begin{tcolorbox}[breakable,title=测试]%[skin=enhancedmiddle jigsaw, leftrule=5mm,rightrule=5mm,  boxsep=0mm,top=0mm,bottom=0mm,  frame style={top color=blue,bottom color=red},
%enhanced jigsaw, breakable,  size=title,  colback=gray!5!white, colframe=red!75!black,  fonttitle=\bfseries, title=Mybreakablebox,pad at break=2mm,break at=-\baselineskip/0pt,  height fixed for=middle]
已知实数$m,n,p且m>1,mn-n-2m^2=0,m^2+n+4=4m+p,则下列结论正确的是:\\
A.\quad n>m\quad B.\quad p>m \quad C.\quad p<n\quad D.\quad n<p$
\tcblower
\begin{flalign*}
&1. \textbf{分析第一个方程}: mn - n - 2m^2 = 0&\\
&n(m-1) = 2m^2&\\
&\text{因为 } m>1, \text{所以 } m-1>0, \text{我们可以得到:}&\\
&n = \dfrac{2m^2}{m-1}&\\
\\
&2. \textbf{比较 n 和 m (选项A)}&\\
&n-m = \dfrac{2m^2}{m-1} - m = \dfrac{2m^2 - m(m-1)}{m-1} = \dfrac{m^2+m}{m-1}&\\
&\text{因为 } m>1, \text{所以分子 } m^2+m > 0, \text{分母 } m-1 > 0。&\\
&\therefore n-m > 0 \implies n > m. \text{ 所以选项A正确。}&\\
\\
&3. \textbf{分析第二个方程}: m^2+n+4=4m+p&\\
&p = m^2-4m+4+n = (m-2)^2+n&\\
\\
&4. \textbf{比较 p 和 n (选项C, D)}&\\
&p-n = (m-2)^2 \ge 0&\\
&\text{这意味着 } p \ge n. \text{ 仅当 m=2 时 p=n。}&\\
&\text{所以选项C (p<n) 错误,选项D (n<p) 不总是成立。}&\\
\\
&5. \textbf{比较 p 和 m (选项B)}&\\
&\text{我们已经证明了 } p \ge n \text{ 且 } n > m。&\\
&\text{根据不等式的传递性,} p \ge n > m \implies p > m.&\\
&\text{所以选项B也正确。}&\\
&\textbf{结论}: \text{如果这是单选题,题目可能存在问题,因为A和B都正确。如果这是多选题,则选A和B。}&
\end{flalign*}

重复求解:
$1. \textbf{分析第一个方程}: mn - n - 2m^2 = 0\\
n(m-1) = 2m^2 
\text{因为 } m>1, \text{所以 } m-1>0, \text{我们可以得到:}\\
n = \dfrac{2m^2}{m-1}\\
\\
2. \textbf{比较 n 和 m (选项A)}\\
n-m = \dfrac{2m^2}{m-1} - m = \dfrac{2m^2 - m(m-1)}{m-1} = \dfrac{m^2+m}{m-1}\\
\text{因为 } m>1, \text{所以分子 } m^2+m > 0, \text{分母 } m-1 > 0。\\
\therefore n-m > 0 \implies n > m. \text{ 所以选项A正确。}\\
\\
3. \textbf{分析第二个方程}: m^2+n+4=4m+p\\
p = m^2-4m+4+n = (m-2)^2+n\\
\\
4. \textbf{比较 p 和 n (选项C, D)}\\
p-n = (m-2)^2 \ge 0\\
\text{这意味着 } p \ge n. \text{ 仅当 m=2 时 p=n。}\\
\text{所以选项C (p<n) 错误,选项D (n<p) 不总是成立。}\\
\\
5. \textbf{比较 p 和 m (选项B)}\\
\text{我们已经证明了 } p \ge n \text{ 且 } n > m。\\
\text{根据不等式的传递性,} p \ge n > m \implies p > m.\\
\text{所以选项B也正确。}\\
\textbf{结论}: \text{如果这是单选题,题目可能存在问题,因为A和B都正确。如果这是多选题,则选A和B。}$
\end{tcolorbox}

1. $已知集合 A=\{ x|-3<x\leq4\} , B=\{ x|m<x<n\} ,若 A\cap B=\{ x|-3<x<-1\} 或 3<x<4 ,则集合 \{ x\in Z|m\leq x\leq n\} 的元素的个数为()$
A. 16
B. 30
C. 31
D. 32

2. $已知集合 A 满足 \{ x\in Z|\frac{4}{3-x}\in Z\} \subseteq A\subseteq \{ -3,-2,0,1,2,3,4\} ,则满足条件的集合 A 的个数为()$
A. 16
B. 15
C. 8
D. 7

3. 下列说法正确的是()
A. $“ x>1 ”是“ \frac{1}{x}<1 ”的充要条件$
B. $“ x\neq y ”是“ xy\neq0 ”的充分不必要条件$
C. $命题“ \forall x>1,\ln x>1 ”的否定是“ \exists x>1,\ln x\leq1 ”$
D. $命题“ \exists x\in R,x^{2}-x+1=0 ”的否定是假命题$

4. $已知 \exists x>0 ,使得 \frac{x+1}{x^{2}}-3m+m^{2}<0 成立,则实数 m 的取值范围是()$
A.  1<m<2 
B.  0<m<2 
C.  m<1 或 m>2 
D.  0<m<3 
5. 已知 a>0,b>0,3a+3b=1 ,则 $\frac{1}{a+1}+\frac{1}{b+1}$ 的取值范围是()
A.  $3+2\sqrt{2}\leq x<6 $
B.  $3+2\sqrt{2}<x<6 $
C.  $2\leq x<6 $
D.  $3\leq x<6 $
6. $已知函数 y=f(x) 的定义域为 [-2,4] ,则 y=\frac{f(2x)}{x-2} 的定义域为()$
A. [1,2)
B. [2,3)
C. (2,4]
D. [2,3]

7. 某装修公司需要用每米12元的铁管制成直角三角形的铁支架,其面积为 $1m^{2}$ ,做一个这样的支架最少约花( $\sqrt{2}\approx1.41$ ,结果取整数)()
A. 36元
B. 48元
C. 58元
D. 72元

8. $若 -5\leq a<2 ,则 -3a-3 的取值范围是()$
A.  $-5\leq -3a-3<2 $
B.  -5<-3a-3<6 
C.  -7<-3a-3<5 
D.  -8<-3a-3<2 

9. 若关于 x 的不等式 $4-x^{2}+ax+a^{2} 在 -1<x<2 内有解,则 a 的取值范围为()$
A.  -2<a<6 
B.  -3<a<5 
C.  -1<a<4 
D.  0<a<6 

10. 若正实数 a,b 满足 a+b=1 ,则 $a^{2}x+3+\frac{1}{b}x-x=0 的根的最大值为()$
A.  $\frac{42-8\sqrt{2}}{7} $
B.  $\frac{-2+8\sqrt{2}}{7} $
C.  $\frac{-9+8\sqrt{2}}{7} $

11. $已知集合 A=\{ x|0\leq x\leq a\} ,集合 B=\{ x|\ln(5+x)<x+6\} ,若命题“ \exists x\in R,A\cap B\neq\varnothing ”为假命题,则实数 a 的取值范围为()$
A.  $\{ a|a<5\} $
B.  $\{ a|a\leq6\}$ 
C.  $\{ a|5\leq a\leq6\} $
D.  $\{ a|0\leq a\leq6\} $

12. $已知函数 f(x) 的定义域为 R , f(x+y)=f(x)+f(y) ,且 f(1)=1 ,则 f(2024)= ()$
A. 0
B. 2024
C. 2025
D. 2026