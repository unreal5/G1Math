\documentclass[CJKmath,a4paper,10pt]{ctexart}
\usepackage{geometry}
\geometry{inner=1.5cm,outer=1.5cm,top=2.45cm,bottom=1cm}
\usepackage[dvipsnames,svgnames,x11names,table]{xcolor}


\RequirePackage{amssymb} % Must be loaded before unicode-math
\RequirePackage{unicode-math} % Math fonts in xetexorluatex
%\setmathfont{texgyrepagella-math.otf}
%\setmainfont{STIXTwoText}[
%	Path=./fonts/STIXTwoText/,
%  Extension = .otf,
%  UprightFont = STIXTwoMath-Regular,
%  BoldFont = STIXTwoText-SemiBold,
%  ItalicFont = STIXTwoText-Italic,
%  BoldItalicFont = STIXTwoText-BoldItalic,
%]
%\setmathfont{STIXTwoMath-Regular.otf}

\usepackage{tikz,calc}
\usetikzlibrary{calc,arrows,shadows.blur,tikzmark}
% By default all math in TikZ nodes are set in inline mode. Change this to
% displaystyle so that we don't get small fractions.
\everymath{\displaystyle}

\usepackage[most]{tcolorbox}
\tcbuselibrary{breakable, skins,theorems}


\usepackage{xkeyval}
\makeatletter 
\usepackage{amsmath,amssymb,amsthm}
\usepackage{xkeyval} 
\usepackage{zhlipsum}
\tcbset{
    common/.style={
      fontupper=\rmfamily,
      lower separated=true,
      coltitle=black,
      colback=gray!5,
      boxrule=0.5pt,
      fonttitle=\bfseries,
      enhanced,
      breakable,
      top=8pt,
      before skip=8pt,
      attach boxed title to top left={yshift=-0.05in,xshift=0.15in},
      boxed title style={
      	boxrule=0pt,
        colframe=black,
        arc=0pt,
        outer arc=0pt,
        drop fuzzy shadow
     	},
      separator sign={.},
      drop fuzzy shadow
     },
     litistyle/.style={  %例题风格
     	common,
      colframe=gray,% 
      colback=white,
      colbacktitle=Gold,
      sharp corners,rounded corners=southeast,
      overlay unbroken and last={
      \node[anchor=south east, outer sep=0pt] at (\linewidth-width,0){\textcolor{black!60}{$\blacksquare$}};}
         },
% -----------------------------------------------------------------------------
% Explanation: BreakableBox@title key
% The following defines a pgfkeys/tcolorbox key named
%   "BreakableBox@title" which accepts 2 arguments (see 
%   "/.code n args={2}"). When the key is used like
%   "BreakableBox@title={<env>}{<optTitle>}" the code below is executed
%   with #1=<env> and #2=<optTitle>.
%
% Purpose:
% - If the second argument (#2) is empty, set the tcolorbox title to
%     \csname <env>name\endcsname~\thetcbcounter
%   which expands to the environment name (e.g. "liti" -> \litiname)
%   followed by the current counter value.
% - If the second argument is provided, append " (#2)" to the title.
%
% Notes:
% - This relies on \makeatletter being active so that keys with '@'
%   are allowed in control sequence names.
% - \ifblank is used to test whether #2 is empty; ensure the package
%   that provides \ifblank (for example, etoolbox or xparse) is loaded
%   earlier if necessary.
% - Example usage inside the code: \tcbset{title={...}} sets the
%   tcolorbox title key accordingly.
%
         BreakableBox@title/.code n args={2}
              {
                \ifblank{#2}
                  {\tcbset{title={\csname #1name\endcsname~\thetcbcounter}}}
                  {\tcbset{title={\csname #1name\endcsname~\thetcbcounter\ (#2)}}}
              },
}      
  % define an internal control sequence \newbreakablebox for fancy mode's newtheorem
  % #1 is the environment name, #2 is the prefix of label, #3 is the style
  % style: thmstyle, defstyle, prostyle
  % e.g. \newbreakablebox{theorem}{thm}{thmstyle}
  % will define two environments: numbered ``theorem'' and no-numbered ``theorem*''
  % WARNING FOR MULTILINGUAL: this cs will automatically find \theoremname's definition,
  % WARNING FOR MULTILINGUAL: it should be defined in language settings.
  \newcommand{\newbreakablebox}[3]{
    \ifcsundef{#1name}{%
      % define a default name if not defined before
      \tcbset{BreakableBox@title/.code n args={2}{\tcbset{title={use newcommand define #1name}}} }
    }{\relax}
    \DeclareTColorBox[auto counter,number within=section]{#1}{ g o t\label g }{
        common, % use common style
        #3, % use the style passed in
        IfValueTF={##1}
          {BreakableBox@title={#1}{##1}}
          {
            IfValueTF={##2}
            {BreakableBox@title={#1}{##2}}
            {BreakableBox@title={#1}{}}
          },
        IfValueT={##4}
          {
            IfBooleanTF={##3}
              {label={##4}}
              {VIVID@label={#2}{##4}}
          }
      }
    \DeclareTColorBox{#1*}{ g o }{
        common,#3,
        IfValueTF={##1}
          {BreakableBox@title={#1}{##1}}
          {
            IfValueTF={##2}
            {BreakableBox@title={#1}{##2}}
            {BreakableBox@title={#1}{}}
          },
      }
  }
  % define several environment 
  % we define headers like \definitionname before
  \newcommand{\litiname}{例题}
  \newbreakablebox{liti}{def}{litistyle}
  
  
%补充内容
%%设置新字体
%%定义带圈数字命令
\newfontfamily{\nmfont}{circlenumber}
[%
Extension=.otf,
Path=./fonts/]

\newcommand{\quan}[1]{{\nmfont \symbol{#1}}}
\newcommand{\kk}[1]{\quan{\numexpr32+#1}}%\kk{<参数范围1-95>}96、97、98、99分别用\quan{196} \quan{197} \quan{199} \quan{201}

%脚注使用带圈数字
\newcommand*\kkctr[1]{%
  \protect\kk{\number\numexpr\value{#1}\relax}}
\renewcommand*\thefootnote{\textcolor{black}{\kkctr{footnote}}}

%%无悬挂脚注格式
\renewcommand\@makefntext[1]{%
  \setlength\parindent{2\ccwd}\selectfont
  \@thefnmark\ #1}

%修改\part,使其不分页
\def\@endpart{%
	\thispagestyle{empty}
  \vskip40\p@%
   \@afterheading}



\RequirePackage{enumitem}
%\newenvironment{myenum}{\begin{enumerate}[label=\protect\kk{\arabic*}]\small}{\end{enumerate}}%
\setlist{noitemsep}
\setlist[enumerate, 1]{label=\protect\kk{\arabic*},itemsep=0.5ex}  
\makeatother



%%%marker环境
\newtcolorbox{marker}[1][]{enhanced,before skip=2mm,
	after skip=3mm,fontupper=\rmfamily,
	boxrule=0.4pt,left=5mm,right=2mm,top=1mm,bottom=1mm,
	colback=yellow!50,colframe=yellow!20!black,
	sharp corners,rounded corners=southeast,
	arc is angular,arc=3mm,underlay={%
		\path[fill=tcbcolback!80!black] ([yshift=3mm]interior.south east)--++(-0.4,-0.1)--++(0.1,-0.2);
		\path[draw=tcbcolframe,shorten <=-0.05mm,shorten >=-0.05mm] ([yshift=3mm]interior.south east)--++(-0.4,-0.1)--++(0.1,-0.2);
		\path[fill=yellow!50!black,draw=none] (interior.south west) rectangle node[white]{\Huge\bfseries !} ([xshift=4mm]interior.north west);
	},
	drop fuzzy shadow,#1
}
\begin{document}
\section{函数单调性九大题型}
\subsection{考点一: 单调性判定、证明}
\begin{liti}
已知函数 \(f(x) = ax - \frac{1}{x}\),且 \(f(-2) = -\frac{3}{2}\)。
\begin{enumerate}
    \item 求函数 \(f(x)\) 的解析式;
    \item 判断函数在区间 \((0, +\infty)\) 上的单调性并用定义法加以证明。
\end{enumerate}
\small
\tcblower
解:
\begin{enumerate}
    \item 由 \(f(-2) = -2a + \frac{1}{2} = -\frac{3}{2}\),解得 \(a = 1\),故 \(f(x) = x - \frac{1}{x}\);
    \item 函数在 \((0, +\infty)\) 上单调递增。证明如下:
          设 \(0 < x_1 < x_2\),则
          \[
          f(x_1) - f(x_2) = \left(x_1 - \frac{1}{x_1}\right) - \left(x_2 - \frac{1}{x_2}\right) = (x_1 - x_2) + \frac{x_1 - x_2}{x_1x_2} = (x_1 - x_2)\left(1 + \frac{1}{x_1x_2}\right)
          \]
          因 \(x_1 - x_2 < 0\),\(1 + \frac{1}{x_1x_2} > 0\),故 \(f(x_1) - f(x_2) < 0\),即 \(f(x_1) < f(x_2)\),因此 \(f(x)\) 在 \((0, +\infty)\) 上单调递增。
\end{enumerate}
\end{liti}

\begin{liti}
已知函数 \(f(x) = \frac{1}{x^2 - 4}\)。
\begin{enumerate}
    \item 求函数 \(f(x)\) 的定义域;
    \item 判断函数 \(f(x)\) 在 \((2, +\infty)\) 上的单调性,并用定义加以证明。
\end{enumerate}
\tcblower 解:
\begin{enumerate}
    \item 定义域为 \(x^2 - 4 \neq 0\),得$x\neq \pm 2$,即 \(x \in (-\infty, -2) \cup (-2, 2) \cup (2, +\infty)\);
    \item 函数在 \((2, +\infty)\) 上单调递减。证明如下:
          设 \(x_2 > x_1 > 2\),\\
          则:
          $
          f(x_1) - f(x_2) = \frac{1}{x_1^2 - 4} - \frac{1}{x_2^2 - 4} = \frac{(x_2 - x_1)(x_2 + x_1)}{(x_1^2 - 4)(x_2^2 - 4)}
          $\\
          \(\because x_2 - x_1 > 0\),\(x_2 + x_1 > 0\),\(x_1^2 - 4 > 0\),\(x_2^2 - 4 > 0\),故 \(f(x_1) - f(x_2) > 0\),即 \(f(x_1) > f(x_2)\),因此 \(f(x)\) 在 \((2, +\infty)\) 上单调递减。
\end{enumerate}
\end{liti}

\begin{liti}
判断并证明函数 \(f(x) = ax^2 + \frac{1}{x}\)(其中 \(1 < a < 3\))在 \([1, 2]\) 上的单调性。

\tcblower 解:函数在 \([1, 2]\) 上单调递增。证明如下:
设 \(1 \leq x_1 < x_2 \leq 2\),则
\[
f(x_1) - f(x_2) = a(x_1^2 - x_2^2) + \left(\frac{1}{x_1} - \frac{1}{x_2}\right) = (x_1 - x_2)\left[a(x_1 + x_2) - \frac{1}{x_1x_2}\right]
\]
因 \(x_1 - x_2 < 0\),\(a(x_1 + x_2) > 2\),\(\frac{1}{x_1x_2} < 1\),故 \(a(x_1 + x_2) - \frac{1}{x_1x_2} > 0\),因此 \(f(x_1) - f(x_2) < 0\),即 \(f(x_1) < f(x_2)\),故 \(f(x)\) 在 \([1, 2]\) 上单调递增。
\end{liti}
\subsection{考点二: 求函数单调区间}
\begin{liti}
求函数 \(f(x) = \frac{x}{1 - x}\) 的单调区间。

\tcblower 解析:化简 \(f(x) = -1 + \frac{1}{1 - x}\),由复合函数单调性可知,单调递增区间为 \((-\infty, 1)\) 和 \((1, +\infty)\)。
\end{liti}

\begin{liti}
(2021·陕西省咸阳中学模拟)求函数 \(f(x) = -x^2 + 2|x| + 1\) 的单调区间。

\tcblower 解析:
- 当 \(x \geq 0\) 时,\(f(x) = -x^2 + 2x + 1\),开口向下,对称轴为 \(x = 1\),增区间为 \((0, 1)\),减区间为 \((1, +\infty)\);
- 当 \(x < 0\) 时,\(f(x) = -x^2 - 2x + 1\),开口向下,对称轴为 \(x = -1\),增区间为 \((-\infty, -1)\),减区间为 \((-1, 0)\);
综上,增区间为 \((-\infty, -1)\) 和 \((0, 1)\),减区间为 \((-1, 0)\) 和 \((1, +\infty)\)。
\end{liti}

\begin{liti}
(2022·四川省通宁中学模拟)求函数 \(f(x) = |x^2 - 3x + 2|\) 的单调递增区间。

\tcblower 解析:令 \(t = x^2 - 3x + 2 = (x - 1)(x - 2)\),画图可知,单调递增区间为 \((1, \frac{3}{2})\) 和 \((2, +\infty)\)。
\end{liti}

\subsection{考点三: 求复合函数的单调区间}
\begin{liti}
求函数 \(y = \frac{1}{4 + 3x - x^2}\) 的单调递增区间。

\tcblower 解析:令 \(t = -x^2 + 3x + 4\)(定义域 \(t \neq 0\),即 \(x \in (-\infty, -1) \cup (-1, 4) \cup (4, +\infty)\)),\(y = \frac{1}{t}\) 单调递减。由“同增异减”:
- \(t = -x^2 + 3x + 4\) 的增区间为 \((-\infty, \frac{3}{2})\),减区间为 \((\frac{3}{2}, +\infty)\);
故 \(y\) 的单调递增区间为 \((\frac{3}{2}, 4)\)。
\end{liti}

\begin{liti}
求函数 \(f(x) = \sqrt{x^2 - 2x - 3}\) 的单调递增区间。

\tcblower 解析:令 \(t = x^2 - 2x - 3\)(定义域 \(t \geq 0\),即 \(x \in (-\infty, -1] \cup [3, +\infty)\)),\(y = \sqrt{t}\) 单调递增。由“同增异减”:
- \(t = x^2 - 2x - 3\) 的增区间为 \([1, +\infty)\);
故 \(f(x)\) 的单调递增区间为 \([3, +\infty)\)。
\end{liti}

\subsection{考点四: 利用单调性确定参数取值范围}
\begin{liti}
若函数 \(f(x) = x^2 - 2mx + 1\) 在 \((2, +\infty)\) 上是增函数,则实数 \(m\) 的取值范围是\_\_\_。

\tcblower 解析:二次函数对称轴为 \(x = m\),增区间为 \([m, +\infty)\),故 \(m \leq 2\),答案:\(m \leq 2\)。
\end{liti}

\begin{liti}
已知函数 \(f(x) = -mx^2 + 3x + 1\) 在区间 \((-1, +\infty)\) 上是增函数,求实数 \(m\) 的取值范围。

\tcblower 解析:
- 当 \(m = 0\) 时,\(f(x) = 3x + 1\) 在 \(\mathbb{R}\) 上递增,满足条件;
- 当 \(m \neq 0\) 时,二次函数开口向下(\(-m > 0 \Rightarrow m < 0\)),对称轴 \(x = \frac{3}{2m}\),需 \(\frac{3}{2m} \leq -1\),解得 \(-\frac{3}{2} \leq m < 0\);
综上,\(m \in [-\frac{3}{2}, 0]\)。
\end{liti}

\begin{liti}
函数 \(f(x) = \frac{ax + 1}{x + 2}\) 在区间 \((-2, +\infty)\) 上单调递增,求实数 \(a\) 的取值范围。

\tcblower 解析:化简 \(f(x) = a + \frac{1 - 2a}{x + 2}\),由反比例函数单调性,需 \(1 - 2a < 0\),解得 \(a > \frac{1}{2}\),答案:\(a > \frac{1}{2}\)。
\end{liti}

\begin{liti}
已知函数 \(f(x) = |-3x + a|\) 的单调递减区间是 \((-\infty, 4]\),则实数 \(a\) 的值为\_\_\_。

\tcblower 解析:\(f(x) = 3|x - \frac{a}{3}|\),单调递减区间为 \((-\infty, \frac{a}{3}]\),故 \(\frac{a}{3} = 4 \Rightarrow a = 12\),答案:\(12\)。
\end{liti}
\subsection{考点五: 利用单调性比较大小}
\begin{liti}
定义在 \(\mathbb{R}\) 上的函数 \(f(x)\) 满足:对任意的 \(x_1, x_2 \in [0, +\infty)\)(\(x_1 \neq x_2\)),有 \(\frac{f(x_2) - f(x_1)}{x_2 - x_1} < 0\)。则下列选项正确的是(  )
A. \(f(3) < f(2) < f(4)\)  B. \(f(1) < f(2) < f(3)\)  C. \(f(-2) < f(1) < f(3)\)  D. \(f(3) < f(1) < f(0)\)

\tcblower 解析:由条件知 \(f(x)\) 在 \([0, +\infty)\) 上单调递减,且 \(f(x)\) 为偶函数(隐含性质),故 \(f(3) < f(1) < f(0)\),答案:D。
\end{liti}

\begin{liti}
(2022·全国·高一专题练习)函数 \(y = f(x)\) 在 \(\mathbb{R}\) 上是增函数,若 \(a + b \leq 0\),则有(  )
A. \(f(a) + f(b) \leq -f(a) - f(b)\)  B. \(f(a) + f(b) \geq -f(a) - f(b)\)  C. \(f(a) + f(b) \leq f(-a) + f(-b)\)  D. \(f(a)f(b) > f(-a)f(-b)\)

\tcblower 解析:由 \(a \leq -b\) 得 \(f(a) \leq f(-b)\),由 \(b \leq -a\) 得 \(f(b) \leq f(-a)\),故 \(f(a) + f(b) \leq f(-a) + f(-b)\),答案:C。
\end{liti}

\subsection{考点六: 利用函数的单调性解决不等式问题}
\begin{liti}
已知函数 \(f(x)\) 在 \([-2, 2]\) 上单调递减,且 \(f(a^2 - a) > f(2a - 2)\),则实数 \(a\) 的取值范围为\_\_\_。

\tcblower 解析:由单调性得 \(\begin{cases} -2 \leq a^2 - a \leq 2 \\ -2 \leq 2a - 2 \leq 2 \\ a^2 - a < 2a - 2 \end{cases}\),解得 \(a \in [0, 1)\),答案:\([0, 1)\)。
\end{liti}


\begin{liti}
(2021·湖南省娄底市二中模拟)已知函数 \(f(x)\) 为 \((0, +\infty)\) 上的增函数,若 \(f(a^2 - a) > f(a + 3)\),则实数 \(a\) 的取值范围为\_\_\_。

\tcblower 解析:由单调性得 \(\begin{cases} a^2 - a > 0 \\ a + 3 > 0 \\ a^2 - a > a + 3 \end{cases}\),解得 \(a \in (-3, -1) \cup (3, +\infty)\),答案:\((-3, -1) \cup (3, +\infty)\)。
\end{liti}


\begin{liti}
已知函数 \(f(x)\) 对任意的 \(a, b \in \mathbb{R}\),都有 \(f(a + b) = f(a) + f(b) - 2\),且当 \(x > 0\) 时,\(f(x) > 2\)。
\begin{enumerate}
    \item 证明:\(f(x)\) 是 \(\mathbb{R}\) 上的增函数;
    \item 若 \(f(4) = 5\),解不等式 \(f(m^2 - m) < \frac{7}{2}\)。
\end{enumerate}

\tcblower 解析:
\begin{enumerate}
    \item 设 \(x_1 > x_2\),则 \(x_1 - x_2 > 0\),\(f(x_1 - x_2) > 2\),故 \(f(x_1) = f(x_2 + (x_1 - x_2)) = f(x_2) + f(x_1 - x_2) - 2 > f(x_2)\),即 \(f(x)\) 单调递增;
    \item 令 \(a = b = 2\),得 \(f(4) = 2f(2) - 2 = 5 \Rightarrow f(2) = \frac{7}{2}\),不等式化为 \(f(m^2 - m) < f(2)\),故 \(m^2 - m < 2\),解得 \(-1 < m < 2\)。
\end{enumerate}
\end{liti}

\subsection{考点七: 最值(值域)问题}
\begin{liti}
当 \(-3 \leq x \leq -1\) 时,函数 \(f(x) = \frac{5x - 1}{4x + 2}\) 的最小值为\_\_\_。

\tcblower 解析:分离常数得 \(f(x) = \frac{5}{4} - \frac{7}{2(4x + 2)}\),在 \([-3, -1]\) 上单调递增,最小值为 \(f(-3) = \frac{8}{5}\),答案:\(\frac{8}{5}\)。
\end{liti}

\begin{liti}
函数 \(y = x^2 - x + 1\) 在区间 \([-1, 1]\) 上的最大值与最小值之和为\_\_\_。

\tcblower 解析:对称轴为 \(x = \frac{1}{2}\),最小值 \(f(\frac{1}{2}) = \frac{3}{4}\),最大值 \(f(-1) = 3\),和为 \(\frac{15}{4}\),答案:\(\frac{15}{4}\)。
\end{liti}

\begin{liti}
画出函数 \(y = -x(|x - 2| - 2)\)(\(x \in [-1, 5]\))的图象,并根据图象指出函数的单调区间和最大、最小值。

\tcblower 解析:
- 当 \(x \leq 2\) 时,\(y = x^2\);当 \(x > 2\) 时,\(y = -x^2 + 4x\);
- 增区间:\((0, 2)\),减区间:\([-1, 0)\) 和 \((2, 5]\);
- 最大值 \(f(2) = 4\),最小值 \(f(5) = -5\)。
\end{liti}

\subsection{考点八: 恒成立与能成立问题}
\begin{liti}
已知函数 \(f(x) = \frac{2x + 1}{x + 1}\)(\(x \geq 0\))。
\begin{enumerate}
    \item 证明:\(f(x)\) 在区间 \([0, +\infty)\) 上为增函数;
    \item 若在 \([0, 2]\) 上存在实数 \(x_0\),使得 \(f(x_0) > \frac{m}{3} + 1\) 成立,求正数 \(m\) 的取值范围。
\end{enumerate}

\tcblower 解析:
\begin{enumerate}
    \item 设 \(0 \leq x_1 < x_2\),则 \(f(x_1) - f(x_2) = \frac{x_1 - x_2}{(x_1 + 1)(x_2 + 1)} < 0\),故 \(f(x)\) 单调递增;
    \item \(f(x)\) 在 \([0, 2]\) 上的最大值为 \(f(2) = \frac{5}{3}\),由题意得 \(\frac{5}{3} > \frac{m}{3} + 1 \Rightarrow m < 2\),故 \(0 < m < 2\)。
\end{enumerate}
\end{liti}

\begin{liti}
已知函数 \(f(x) = (x - 2)|x - a| + 1\)。
\begin{enumerate}
    \item 当 \(a = 4\) 时,写出 \(f(x)\) 的单调区间;
    \item 若存在 \(x \in [3, 5]\),使得 \(f(x) > 5\),求实数 \(a\) 的取值范围。
\end{enumerate}

\tcblower 解析:
\begin{enumerate}
    \item 当 \(a = 4\) 时,\(f(x) = \begin{cases} (x - 2)(x - 4) + 1, x \geq 4 \\ (x - 2)(4 - x) + 1, x < 4 \end{cases}\),增区间:\((-\infty, 3)\) 和 \([4, +\infty)\),减区间:\([3, 4)\);
    \item 不等式化为 \((x - 2)|x - a| > 4\),即 \(a < x - \frac{4}{x - 2}\) 或 \(a > x + \frac{4}{x - 2}\)。由最值得 \(a < 4\) 或 \(a > 6\)。
\end{enumerate}
\end{liti}
\end{document}