\documentclass[CJKmath,a4paper,10pt]{ctexart}
\usepackage{geometry}
\geometry{inner=1.5cm,outer=1.5cm,top=2.45cm,bottom=1cm}
\usepackage[dvipsnames,svgnames,x11names,table]{xcolor}
\usepackage{exam-zh-choices}

\RequirePackage{amssymb} % Must be loaded before unicode-math
\RequirePackage{unicode-math} % Math fonts in xetexorluatex
\setmathfont{texgyrepagella-math.otf}
\setmainfont{STIXTwoText}[
	Path=./fonts/STIXTwoText/,
  Extension = .otf,
  UprightFont = STIXTwoMath-Regular,
  BoldFont = STIXTwoText-SemiBold,
  ItalicFont = STIXTwoText-Italic,
  BoldItalicFont = STIXTwoText-BoldItalic,
]
%\setmathfont{STIXTwoMath-Regular.otf}

\usepackage{tikz,calc}
\usetikzlibrary{calc,arrows,shadows.blur,tikzmark}
% By default all math in TikZ nodes are set in inline mode. Change this to
% displaystyle so that we don't get small fractions.
\everymath{\displaystyle}

\usepackage[most]{tcolorbox}
\tcbuselibrary{breakable, skins,theorems}


\usepackage{xkeyval}
\makeatletter 
\usepackage{amsmath,amssymb,amsthm}
\usepackage{xkeyval} 
\usepackage{zhlipsum}
\tcbset{
    common/.style={
      fontupper=\rmfamily,
      lower separated=true,
      coltitle=black,
      colback=gray!5,
      boxrule=0.5pt,
      fonttitle=\bfseries,
      enhanced,
      breakable,
      top=8pt,
      before skip=8pt,
      attach boxed title to top left={yshift=-0.05in,xshift=0.15in},
      boxed title style={
      	boxrule=0pt,
        colframe=black,
        arc=0pt,
        outer arc=0pt,
        drop fuzzy shadow
     	},
      separator sign={.},
      drop fuzzy shadow
     },
     litistyle/.style={  %例题风格
     	common,
      colframe=gray,% 
      colback=white,
      colbacktitle=Gold,
      sharp corners,rounded corners=southeast,
      overlay unbroken and last={
      \node[anchor=south east, outer sep=0pt] at (\linewidth-width,0){\textcolor{black!60}{$\blacksquare$}};}
         },
% -----------------------------------------------------------------------------
% Explanation: BreakableBox@title key
% The following defines a pgfkeys/tcolorbox key named
%   "BreakableBox@title" which accepts 2 arguments (see 
%   "/.code n args={2}"). When the key is used like
%   "BreakableBox@title={<env>}{<optTitle>}" the code below is executed
%   with #1=<env> and #2=<optTitle>.
%
% Purpose:
% - If the second argument (#2) is empty, set the tcolorbox title to
%     \csname <env>name\endcsname~\thetcbcounter
%   which expands to the environment name (e.g. "liti" -> \litiname)
%   followed by the current counter value.
% - If the second argument is provided, append " (#2)" to the title.
%
% Notes:
% - This relies on \makeatletter being active so that keys with '@'
%   are allowed in control sequence names.
% - \ifblank is used to test whether #2 is empty; ensure the package
%   that provides \ifblank (for example, etoolbox or xparse) is loaded
%   earlier if necessary.
% - Example usage inside the code: \tcbset{title={...}} sets the
%   tcolorbox title key accordingly.
%
         BreakableBox@title/.code n args={2}
              {
                \ifblank{#2}
                  {\tcbset{title={\csname #1name\endcsname~\thetcbcounter}}}
                  {\tcbset{title={\csname #1name\endcsname~\thetcbcounter\ (#2)}}}
              },
}      
  % define an internal control sequence \newbreakablebox for fancy mode's newtheorem
  % #1 is the environment name, #2 is the prefix of label, #3 is the style
  % style: thmstyle, defstyle, prostyle
  % e.g. \newbreakablebox{theorem}{thm}{thmstyle}
  % will define two environments: numbered ``theorem'' and no-numbered ``theorem*''
  % WARNING FOR MULTILINGUAL: this cs will automatically find \theoremname's definition,
  % WARNING FOR MULTILINGUAL: it should be defined in language settings.
  \newcommand{\newbreakablebox}[3]{
    \ifcsundef{#1name}{%
      % define a default name if not defined before
      \tcbset{BreakableBox@title/.code n args={2}{\tcbset{title={use newcommand define #1name}}} }
    }{\relax}
    \DeclareTColorBox[auto counter,number within=section]{#1}{ g o t\label g }{
        common, % use common style
        #3, % use the style passed in
        IfValueTF={##1}
          {BreakableBox@title={#1}{##1}}
          {
            IfValueTF={##2}
            {BreakableBox@title={#1}{##2}}
            {BreakableBox@title={#1}{}}
          },
        IfValueT={##4}
          {
            IfBooleanTF={##3}
              {label={##4}}
              {VIVID@label={#2}{##4}}
          }
      }
    \DeclareTColorBox{#1*}{ g o }{
        common,#3,
        IfValueTF={##1}
          {BreakableBox@title={#1}{##1}}
          {
            IfValueTF={##2}
            {BreakableBox@title={#1}{##2}}
            {BreakableBox@title={#1}{}}
          },
      }
  }
  % define several environment 
  % we define headers like \definitionname before
  \newcommand{\litiname}{例题}
  \newbreakablebox{liti}{def}{litistyle}
  
  
%补充内容
%%设置新字体
%%定义带圈数字命令
\newfontfamily{\nmfont}{circlenumber}
[%
Extension=.otf,
Path=./fonts/]

\newcommand{\quan}[1]{{\nmfont \symbol{#1}}}
\newcommand{\kk}[1]{\quan{\numexpr32+#1}}%\kk{<参数范围1-95>}96、97、98、99分别用\quan{196} \quan{197} \quan{199} \quan{201}

%脚注使用带圈数字
\newcommand*\kkctr[1]{%
  \protect\kk{\number\numexpr\value{#1}\relax}}
\renewcommand*\thefootnote{\textcolor{black}{\kkctr{footnote}}}

%%无悬挂脚注格式
\renewcommand\@makefntext[1]{%
  \setlength\parindent{2\ccwd}\selectfont
  \@thefnmark\ #1}

%修改\part,使其不分页
\def\@endpart{%
	\thispagestyle{empty}
  \vskip40\p@%
   \@afterheading}



\RequirePackage{enumitem}
%\newenvironment{myenum}{\begin{enumerate}[label=\protect\kk{\arabic*}]\small}{\end{enumerate}}%
\setlist{noitemsep}
\setlist[enumerate, 1]{label=\protect\kk{\arabic*},itemsep=0.5ex}  
\makeatother



%%%marker环境
\newtcolorbox{marker}[1][]{enhanced,before skip=2mm,
	after skip=3mm,fontupper=\rmfamily,
	boxrule=0.4pt,left=5mm,right=2mm,top=1mm,bottom=1mm,
	colback=yellow!50,colframe=yellow!20!black,
	sharp corners,rounded corners=southeast,
	arc is angular,arc=3mm,underlay={%
		\path[fill=tcbcolback!80!black] ([yshift=3mm]interior.south east)--++(-0.4,-0.1)--++(0.1,-0.2);
		\path[draw=tcbcolframe,shorten <=-0.05mm,shorten >=-0.05mm] ([yshift=3mm]interior.south east)--++(-0.4,-0.1)--++(0.1,-0.2);
		\path[fill=yellow!50!black,draw=none] (interior.south west) rectangle node[white]{\Huge\bfseries !} ([xshift=4mm]interior.north west);
	},
	drop fuzzy shadow,#1
}
\begin{document}
\section{恒成立、存在性问题}

\begin{liti}
已知关于$x$的不等式$2x-1 > m(x^2-1)$
\begin{enumerate}
    \item 是否存在实数$m$,使不等式对任意$x \in \mathbb{R}$恒成立?
    \item 若不等式对于$x \in (1, +\infty)$恒成立,求$m$的取值范围.
    \item 若不等式对于$m \in [-2, 2]$恒成立,求实数$x$的取值范围.
\end{enumerate}
\tcblower
原不等式等价于 $-mx^2+2x+(m-1)>0$。
\begin{enumerate}
\item  若要对 $\forall x\in\mathbb{R}$ 恒成立:
\begin{itemize}
\item 若 $m>0$,当 $\big|x\big|\to\infty$,右侧 $m(x^2-1)\sim m x^2$,线性项 $2x-1$ 无法压过二次项,故不等式不可能恒真;
\item 若 $m\le 0$,代入 $x=0$ 得 $-1>-m\iff m>1$,与 $m\le 0$ 矛盾。
\end{itemize}
综上,\textbf{不存在}这样的 $m$。

\item 若对 $x\in(1,+\infty)$ 恒成立:当 $x>1$ 时 $x^2-1>0$,于是
\[
m<\frac{2x-1}{x^2-1}=: \varphi(x).
\]
在 $(1,+\infty)$ 上,$\displaystyle \varphi'(x)=-\frac{2(x^2-x+1)}{(x^2-1)^2}<0$,故 $\varphi$ 严格递减,
\[
\inf_{x>1}\varphi(x)=\lim_{x\to\infty}\varphi(x)=0,\quad \text{且 } \varphi(x)>0.
\]
因此充要条件为 $m\le 0$。

\item 若对所有 $m\in[-2,2]$ 恒成立:
\[
2x-1>\max_{m\in[-2,2]} m(x^2-1)
=\begin{cases}
2(x^2-1), & \big|x\big|>1,\\[2pt]
-2(x^2-1), & \big|x\big|<1,\\[2pt]
0, & \big|x\big|=1.
\end{cases}
\]
分情况解:
\begin{itemize}
\item  $\big|x\big|>1$: $2x-1>2(x^2-1)\iff 2x^2-2x-1<0$,
  根为 $\displaystyle \frac{1\pm\sqrt{3}}{2}$,取两根之间并与 $\big|x\big|>1$ 交,得
  $\displaystyle x\in\Big(1,\frac{1+\sqrt{3}}{2}\Big)$;
\item  $\big|x\big|<1$: $2x-1>-2(x^2-1)\iff 2x^2+2x-3>0$,
  根为 $\displaystyle \frac{-1\pm\sqrt{7}}{2}$,在 $\big|x\big|<1$ 内取右侧外区间,得
  $\displaystyle x\in\Big(\frac{-1+\sqrt{7}}{2},\,1\Big)$;
\item  $\big|x\big|=1$: $x=1$ 成立,$x=-1$ 不成立。
\end{itemize}
合并可得$x\in\Big(\tfrac{-1+\sqrt{7}}{2},\,\tfrac{1+\sqrt{3}}{2}\Big)$.
\end{enumerate}
综上: (1) 不存在;(2) $m\le 0$;(3) $x\in\big(\tfrac{-1+\sqrt{7}}{2},\,\tfrac{1+\sqrt{3}}{2}\big)$。
\end{liti}

\begin{liti}
$f(x) = \sqrt{(m^2 - m - 6)x^2 + (m + 2)x + 8}$
\begin{enumerate}
    \item 若$f(x)$的定义域为$[-1, 2]$,求$m = \underline{\qquad\qquad}$
    \item 若$f(x)$的定义域为$\mathbb{R}$,求实数$m$的取值范围.
    \item 若$f(x)$的值域为$[0, +\infty)$,求实数$m$的取值范围.
    \item 若$f(x)$的值域为$[1, +\infty)$,则$m = \underline{\quad}$
\end{enumerate}
\end{liti}

\begin{liti}
\begin{enumerate}
    \item 若不等式$\dfrac{1}{x} + \dfrac{1}{1 - 4x} - m \geq 0$对$x \in (0, \dfrac{1}{4})$恒成立,则$m$的最大值为$\underline{\qquad 9 \qquad}$
    \item 若存在$x \in \mathbb{R}$使$|x - 1| + |x + 2| \leq m$,求$m$的取值范围.
    \item 若存在$x \in [1, 3]$使$x^2 - 2mx + m^2 - 1 < 0$成立,求$m$的取值范围.
\end{enumerate}
\tcblower
\begin{enumerate}
    \item 不等式恒成立 $\iff m \le \dfrac{1}{x} + \dfrac{1}{1 - 4x}$ 对 $\forall x \in (0, \frac{1}{4})$ 成立。
    	设 $f(x) = \dfrac{1}{x} + \dfrac{1}{1 - 4x}$,则 $m$ 的最大值为 $f(x)$ 在 $(0, \frac{1}{4})$ 上的最小值。
    	
    	$\dfrac{1}{x} + \dfrac{1}{1 - 4x},令1-4x=t,(0<t<1),则x=\dfrac{1-t}{4}。原式等于:\dfrac{4}{1-t}+\dfrac{1}{t}$
    	\\
    	
    	乘以“1”:$\big[(1-t)+t\big]\big(\dfrac{4}{1-t}+\dfrac{1}{t}\big)=4+\dfrac{1-t}{t}+\dfrac{4t}{1-t}+1\geq 5+2\sqrt{4}=9$.
    	
    	当且仅当:$\big(1-t\big)^2=4t^2,即t=\dfrac{1}{3},x=\dfrac{1-\dfrac{1}{3}}{4}=\dfrac{1}{6}$时成立。$\dfrac{1}{6}\in(0,\dfrac{1}{4})$
    	
    	$\therefore m_{max}= 9$


    \item “存在$x$”成立,等价于 $m$ 大于或等于函数 $g(x)=|x-1|+|x+2|$ 的最小值。
    由绝对值的几何意义,$g(x)$ 表示数轴上的点 $x$ 到 $1$ 和 $-2$ 的距离之和,其最小值为 $|1-(-2)|=3$。
    故 $m \ge 3$。$m$ 的取值范围为 $[3, +\infty)$。

    \item 不等式为 $(x-m)^2 < 1 \iff |x-m|<1 \iff m-1 < x < m+1$。
    “存在 $x \in [1,3]$”成立,等价于区间 $(m-1, m+1)$ 与 $[1,3]$ 有交集。
    其充要条件是 $m+1 > 1$ 且 $m-1 < 3$,解得 $0 < m < 4$。
    $m$ 的取值范围为 $(0, 4)$。
\end{enumerate}
\end{liti}

\begin{liti}{(口诀):“\textbf{任意}”即“扛住最坏情况”,“\textbf{存在}”即“只要有一个即可”}
\begin{enumerate}
    \item 已知$f(x) = x^2 - x + 3$,$g(x) = 2x - a$,若对$\forall x_1 \in [0, 2]$,$\exists x_2 \in [0, 2]$使$f(x_1) \geq g(x_2)$,求$a$的取值范围.
    \begin{itemize}
        \item 变式1:$\forall x_1 \in [0, 2], \forall x_2 \in [0, 2]$
        \item 变式2:$\exists x_1 \in [0, 2], \forall x_2 \in [0, 2]$
        \item 变式3:$\exists x_1 \in [0, 2], \exists x_2 \in [0, 2]$
    \end{itemize}
    \item 已知$f(x) = x^2$,$g(x) = 4 - x^2$,若对$\forall x_1 \in [0, 2]$,$\exists x_2 \in [0, 2]$使$f(x_1) + g(x_2) \geq c$,求$c$的取值范围.
\end{enumerate}
\end{liti}

\begin{liti}
已知对于$\forall x_1 \in (2, 3)$,$\exists x_2 \in (1, 2)$使$f(x_2) = g(x_1)$成立,其中$f(x) = -2x + 8 - 4a$,$g(x) = x^2 - 2ax$,求$a$的取值范围.
\tcblower
\textbf{分析:}
条件“对于$\forall x_1 \in (2, 3)$,$\exists x_2 \in (1, 2)$使$f(x_2) = g(x_1)$成立”的含义是,函数 $g(x)$ 在区间 $(2,3)$ 上的值域,必须是函数 $f(x)$ 在区间 $(1,2)$ 上的值域的\textbf{子集}。

设 $R_f = \{f(x_2) \mid x_2 \in (1,2)\}$,$R_g = \{g(x_1) \mid x_1 \in (2,3)\}$。则条件等价于 $R_g \subseteq R_f$。

\textbf{解:}
\begin{enumerate}
    \item \textbf{求函数 $f(x)$ 的值域 $R_f$}
    
    函数 $f(x) = -2x + 8 - 4a$ 是关于 $x$ 的一次函数,斜率为 $-2$,因此在区间 $(1,2)$ 上单调递减。
    其值域为一个开区间:
    \[ R_f = (f(2), f(1)) \]
    其中 $f(1) = -2(1) + 8 - 4a = 6 - 4a$,
    $f(2) = -2(2) + 8 - 4a = 4 - 4a$。
    所以,$R_f = (4 - 4a, 6 - 4a)$。

    \item \textbf{求函数 $g(x)$ 的值域 $R_g$}
    
    函数 $g(x) = x^2 - 2ax$ 是开口向上的二次函数,对称轴为 $x=a$。我们需要根据对称轴 $a$ 相对于区间 $(2,3)$ 的位置来讨论其值域。
    \begin{itemize}
        \item \textbf{情况一:$a \le 2$}
        
        对称轴在区间 $(2,3)$ 的左侧或恰好在左端点。此时 $g(x)$ 在 $(2,3)$ 上单调递增。
        值域为 $(g(2), g(3))$。
        $g(2) = 2^2 - 2a(2) = 4 - 4a$。
        $g(3) = 3^2 - 2a(3) = 9 - 6a$。
        所以 $R_g = (4 - 4a, 9 - 6a)$。
        
        \item \textbf{情况二:$2 < a < 3$}
        
        对称轴在区间 $(2,3)$ 内部。$g(x)$ 在 $x=a$ 处取得最小值。
        最小值为 $g(a) = a^2 - 2a(a) = -a^2$。
        区间的两个端点值分别为 $g(2)=4-4a$ 和 $g(3)=9-6a$。
        由于 $2<a<3$,有 $g(3)-g(2) = (9-6a)-(4-4a) = 5-2a < 0$,所以 $g(3) < g(2)$。
        因此,值域为 $[g(a), g(2)) = [-a^2, 4-4a)$。
        
        \item \textbf{情况三:$a \ge 3$}
        
        对称轴在区间 $(2,3)$ 的右侧或恰好在右端点。此时 $g(x)$ 在 $(2,3)$ 上单调递减。
        值域为 $(g(3), g(2))$。
        所以 $R_g = (9 - 6a, 4 - 4a)$。
    \end{itemize}

    \item \textbf{利用 $R_g \subseteq R_f$ 求解 $a$}
    
    将上述三种情况得到的值域 $R_g$ 分别代入 $R_g \subseteq (4-4a, 6-4a)$ 求解。
    \begin{itemize}
        \item 当 $a \le 2$ 时,$R_g = (4-4a, 9-6a)$。
        
        $R_g \subseteq R_f$ 要求 $9-6a \le 6-4a$。
        解得 $3 \le 2a \implies a \ge \frac{3}{2}$。
        结合条件 $a \le 2$,得 $\frac{3}{2} \le a \le 2$。
        
        \item 当 $2 < a < 3$ 时,$R_g = [-a^2, 4-4a)$。
        
        $R_g \subseteq R_f$ 要求 $-a^2 \ge 4-4a$。
        即 $a^2 - 4a + 4 \le 0 \implies (a-2)^2 \le 0$。
        这只有在 $a=2$ 时成立,但这与本情况的条件 $2 < a < 3$ 矛盾。故此情况下无解。
        
        \item 当 $a \ge 3$ 时,$R_g = (9-6a, 4-4a)$。
        
        $R_g \subseteq R_f$ 要求 $9-6a \ge 4-4a$ 且 $4-4a \le 6-4a$。
        $9-6a \ge 4-4a \implies 5 \ge 2a \implies a \le \frac{5}{2}$。
        $4-4a \le 6-4a \implies 4 \le 6$,恒成立。
        结合条件 $a \ge 3$,有 $a \ge 3$ 且 $a \le \frac{5}{2}$,无解。
    \end{itemize}
\end{enumerate}
\textbf{结论:}
综上所述,只有第一种情况有解,实数 $a$ 的取值范围是 $\left[\dfrac{3}{2}, 2\right]$。
\end{liti}

\section{函数性质应用压轴题}
\begin{liti}
已知$f(x)$,$x \in \mathbb{R}$,$\forall x_1,x_2 \in (0,+\infty)$且$x_1 \neq x_2$,$\dfrac{x_2f(x_1)-x_1f(x_2)}{x_1-x_2} > 0$,$f(x)$为奇函数,$f(2025)=2025$,则不等式$f(x) > x$的解集为$\underline{\quad (-2025, 0) \cup (2025, +\infty) \quad}$.
\tcblower
	\textbf{解题过程:}\quad 关键在于由所给分式不等式推出 $\dfrac{f(x)}{x}$ 在 $(0,+\infty)$ 上严格递增,再结合奇函数性质把正区间的比较结果“翻转”到负区间。

设 $g(x)=\dfrac{f(x)}{x}$($x\ne 0$)。取任意 $0<x_1<x_2$,则分母 $x_1-x_2<0$。因
\[
\frac{x_2f(x_1)-x_1f(x_2)}{x_1-x_2}>0 \quad (x_1\ne x_2,\ x_1,x_2>0),
\]
得分子 $x_2f(x_1)-x_1f(x_2)<0$,即 $x_2f(x_1)<x_1f(x_2)$,整理
\[
\frac{f(x_1)}{x_1}<\frac{f(x_2)}{x_2} \iff g(x_1)<g(x_2).
\]
故 $g$ 在 $(0,+\infty)$ 上严格递增。

又因 $f$ 为奇函数,$f(-x)=-f(x)$,于是
\[
g(-x)=\frac{f(-x)}{-x}=\frac{-f(x)}{-x}=\frac{f(x)}{x}=g(x),
\]
说明 $g$ 是偶函数(而且只需在正半轴知道它的单调性)。

已知 $f(2025)=2025 \Rightarrow g(2025)=\dfrac{2025}{2025}=1$。由严格递增性:
\[
0<x<2025 \Rightarrow g(x)<1; \qquad x>2025 \Rightarrow g(x)>1.
\]

现在求 $f(x)>x$ 的解:
\begin{itemize}
  \item 当 $x>0$,$f(x)>x \iff \dfrac{f(x)}{x}>1 \iff g(x)>1 \iff x>2025$。
  \item 当 $x<0$,写成 $x=-t$ ($t>0$)。条件 $f(-t)>-t \iff -f(t)>-t \iff f(t)<t \iff g(t)<1 \iff t<2025$,即 $x\in(-2025,0)$。
  \item $x=0$ 时,因奇函数有 $f(0)=0$,不满足 $f(0)>0$;$x=\pm 2025$ 时 $f(\pm 2025)=\pm 2025$ 为等号,不取。
\end{itemize}

综上解集为 $(-2025,0)\cup(2025,+\infty)$。

	\textbf{检验:} 若取 $x=-3000(<-2025)$,因为 $3000>2025$,有 $g(3000)>1 \Rightarrow f(3000)>3000 \Rightarrow f(-3000)= -f(3000)<-3000$,不满足;与解集外一致。若取 $x=-1000$,$1000<2025$,则 $f(1000)<1000 \Rightarrow f(-1000)>-1000$,满足;与解集内一致。正确。

\end{liti}

\begin{liti}
已知$f(x)$,$x \in \mathbb{R}$,$\forall x,y \in \mathbb{R}$,$f(x)f(y)=f(x+y)$,且$f(1)=\dfrac{1}{2}$,则()
\begin{choices}
\item  $f(0)=0$
\item  $f(-1)=2$
\item  $f(x+1)=f(x)$
\item  $f(x+2)-f(x+11) < f(x+1)-f(x)$
\end{choices}
\end{liti}


\begin{liti}
已知$f(x)$为$\mathbb{R}$上的奇函数,且$f(1-x)=f(1+x)$,若$f(1)=2$,则$f(1)+f(2)+\dots+f(2025)=$()
\begin{choices}
\item $0$
\item  $2025$
\item  $2024$
\item  $2$
\end{choices}
\end{liti}

\begin{liti}
已知$f(x)$为$\mathbb{R}$上的奇函数,当$x_1,x_2 \in (0,+\infty)$,$\dfrac{x_2f(x_1)-x_1f(x_2)}{x_1-x_2} > 0$,$f(6)=6$,则不等式$f(x) > x$的解集为()[B]
\begin{choices}
\item  $(-\infty,-6)\cup(0,6)$
\item  $(-6,0)\cup(6,+\infty)$
\item  $(-6,0)\cup(0,6)$
\item  $(-\infty,-6)\cup(6,+\infty)$
\end{choices}
\end{liti}

\begin{liti}
已知$f(x)$,$x \in \mathbb{R}$,$f(x+4)$为偶函数,$f(-x+2)$为奇函数,且$f(x)$在$[0,2]\uparrow$,则下列错误的是()

\begin{choices}
\item  $f(2)=0$
\item  $x=4$是$f(x)$的一条对称轴
\item  $f(x)$在$[4,8]\downarrow$
\item  $f(1) < f(7)$
\end{choices}
\end{liti}

\begin{liti}
$f(x)$,$x \in (0,+\infty)$,当$x_1 \neq x_2$时,$\dfrac{x_2f(x_1)-x_1f(x_2)}{x_1-x_2} > 0$,则()
\begin{choices}
\item $3f(4) > 4f(3)$
\item  $y=\dfrac{f(x)}{x}$在$(0,+\infty)\uparrow$
\item  $y=xf(x)$在$(0,+\infty)\uparrow$
\item  $f(3x_1+x_2)+f(x_1+3x_2) > 4f(x_1+x_2)$
\end{choices}
\end{liti}

\begin{liti}
若定义在$(-1,1)$上的$f(x)$不恒为$0$,对$\forall x,y \in (-1,1)$都有$f(x)+f(y)=f\left(\dfrac{x+y}{1+xy}\right)$,且当$x \in (-1,0)$时,$f(x) > 0$,则()
\begin{choices}
\item  $f(0)=0$
\item  $f(x)$为奇函数
\item  $f\left(\dfrac{1}{3}\right)+f\left(\dfrac{1}{4}\right) > f\left(\dfrac{1}{2}\right)$
\item  $f(x)$在$(0,1)\downarrow$
\end{choices}
\end{liti}

\begin{liti}
已知$f(x)=\begin{cases} x^2+(1-a)x-2a+b,\ &x < 1 \\ \dfrac{a+1}{x}+2x,\ &x \geq 1 \end{cases}$,对于$\forall x_1,x_2 \in \mathbb{R}$且$x_1 \neq x_2$,$\dfrac{f(x_1)-f(x_2)}{x_1-x_2} < 2$成立,则$a$的取值范围为$\underline{\quad\quad\quad}$.
\end{liti}

\begin{liti}
设$f(x)$为$\mathbb{R}$上的奇函数,$\forall x_1,x_2 \in (0,+\infty)$,$x_1 \neq x_2$,$\dfrac{x_2f(x_1)-x_1f(x_2)}{x_1-x_2} > 0$,若$f(2)=4$,则$f(x)-2x \leq 0$的解集为$\underline{\quad\quad\quad}$.
\end{liti}

\begin{liti}
$f(x)$,$x \in \mathbb{R}$,$\forall x,y \in \mathbb{R}$,$f(xy)=f(x)f(y)$,且$f(-1)=-1$,当$0 < x < 1$时$f(x) \in (0,1)$.
    \begin{enumerate}
        \item 判断$f(x)$的奇偶性;
        \item 判断$f(x)$在$(0,+\infty)$的单调性并证明;
        \item $\forall x_1,x_2 \in [-1,1]$,$a \in [-1,5]$,总有$2\big|f(x_1)-f(x_2)\big| \leq m^2-am-2$恒成立,求$m$的取值范围.
    \end{enumerate}
\end{liti}  
\end{document}